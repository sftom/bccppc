\begin{figure*}
    \centering
    \includegraphics[width=2.57cm]{./images/brasao_da_republica.jpeg}
    
    \textsf{Ministério da Educação}\\
    \textsf{Universidade Federal do Agreste de Pernambuco}\\
    \textsf{Bacharelado em Ciência da Computação}
\end{figure*}

\vspace{0.5cm}

\begin{scriptsize}
\noindent \fbox{
\begin{minipage}[t]{\textwidth}
\textsf{\scriptsize COMPONENTE CURRICULAR:} \\
\textbf{COMPUTAÇÃO GRÁFICA} \\
\textsf{\scriptsize CÓDIGO:} \textbf{CCMP3019}
\end{minipage}}

\noindent \fbox{
\begin{minipage}[t]{0.293\textwidth}
\textsf{\scriptsize PERÍODO A SER OFERTADO:} \\
\textbf{6}
\end{minipage}
%
\begin{minipage}[t]{0.7\textwidth}
\textsf{\scriptsize NÚCLEO DE FORMAÇÃO:} \\
\textbf{CICLO PROFISSIONAL OU TRONCO COMUM}
\end{minipage}}

\noindent \fbox{
\begin{minipage}[t]{0.493\textwidth}
\textsf{\scriptsize TIPO:} \textbf{OBRIGATÓRIO}
\end{minipage}
%
\begin{minipage}[t]{0.5\textwidth}
\textsf{\scriptsize CRÉDITOS:} \textbf{4}
\end{minipage}}

\noindent \fbox{
\begin{minipage}[t]{0.3\textwidth}
\textsf{\scriptsize CARGA HORÁRIA TOTAL:} 
\textbf{60}
\end{minipage}
%
\begin{minipage}[t]{0.19\textwidth}
\textsf{\scriptsize TEÓRICA:} 
\textbf{60}
\end{minipage}
%
\begin{minipage}[t]{0.19\textwidth}
\textsf{\scriptsize PRÁTICA:} 
\textbf{0}
\end{minipage}
%
\begin{minipage}[t]{0.30\textwidth}
\textsf{\scriptsize EAD-SEMIPRESENCIAL:} 
\textbf{0}
\end{minipage}}

\noindent \fbox{
\begin{minipage}[t]{\textwidth}
\textsf{\scriptsize PRÉ-REQUISITOS:}
\begin{itemize}
\item
  CCMP3057 INTRODUÇÃO À PROGRAMAÇÃO
\item
  MATM3019 ÁLGEBRA LINEAR I
\item
  MATM3021 GEOMETRIA ANALÍTICA A
\end{itemize}
\end{minipage}}

\noindent \fbox{
\begin{minipage}[t]{\textwidth}
\textsf{\scriptsize CORREQUISITOS:} 
Não há.
\end{minipage}}

\noindent \fbox{
\begin{minipage}[t]{\textwidth}
\textsf{\scriptsize REQUISITO DE CARGA HORÁRIA:} 
Não há.
\end{minipage}}

\noindent \fbox{
\begin{minipage}[t]{\textwidth}
\textsf{\scriptsize EMENTA:} \\
Introdução à computação gráfica. Biblioteca gráfica OpenGL.
Representação de objetos. Dispositivos periféricos gráficos. Processo de
visualização. Curvas e superfícies paramétricas. Eliminação de
superfícies ocultas. Geração de imagens com realismo. Tópicos
complementares em computação gráfica.
\end{minipage}}

\noindent \fbox{
\begin{minipage}[t]{\textwidth}
\textsf{\scriptsize BIBLIOGRAFIA BÁSICA:}
\begin{enumerate}
\def\labelenumi{\arabic{enumi}.}
\item
  Computação Gráfica: Teoria e Prática, AZEVEDO, E. e Conci, A. Editora
  Campus, Elsevier, 2003. Rio de Janeiro.
\item
  Fundamentos de Computação Gráfica, Gomes, J. e Velho, L. IMPA, 2003.
\item
  Geometric Algebra for Computer Graphics, John A. Vince, Springer,
  2008.
\end{enumerate}
\end{minipage}}

\noindent \fbox{
\begin{minipage}[t]{\textwidth}
\textsf{\scriptsize BIBLIOGRAFIA COMPLEMENTAR:}
\begin{enumerate}
\def\labelenumi{\arabic{enumi}.}
\item
  Fundamentals of Computer Graphics, Second Ed. Peter Shirley, et al.~A
  K Peters Ltd, 2005.
\item
  OpenGL® Programming Guide, Shreiner D., et al.~Addison-Wesley, 5th
  Edition, 2005.
\item
  Foley, J.D. van Dam, A. Feiner K.S., Jughes, J.F., ``Computer
  Graphics: Principles And Practice'', Addison Wesley, 1993.
\end{enumerate}
\end{minipage}}
\end{scriptsize}

\newpage