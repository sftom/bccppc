\begin{figure*}
    \centering
    \includegraphics[width=2.57cm]{./images/brasao_da_republica.jpeg}
    
    \textsf{Ministério da Educação}\\
    \textsf{Universidade Federal do Agreste de Pernambuco}\\
    \textsf{Bacharelado em Ciência da Computação}
\end{figure*}

\vspace{2cm}

\begin{scriptsize}
\noindent \fbox{
\begin{minipage}[t]{\textwidth}
\textsf{\scriptsize COMPONENTE CURRICULAR:} \\
\textbf{SEGURANÇA DE REDES DE COMPUTADORES} \\
\textsf{\scriptsize CÓDIGO:} \\
\textbf{CCMP3079}
\end{minipage}}

\noindent \fbox{
\begin{minipage}[t]{0.293\textwidth}
\textsf{\scriptsize PERÍODO A SER OFERTADO:} \\
\textbf{0}
\end{minipage}
%
\begin{minipage}[t]{0.7\textwidth}
\textsf{\scriptsize NÚCLEO DE FORMAÇÃO:} \\
\textbf{COMPONENTES OPTATIVOS ÁREA TEMÁTICA REDES E SISTEMAS
DISTRIBUÍDOS}
\end{minipage}}

\noindent \fbox{
\begin{minipage}[t]{0.493\textwidth}
\textsf{\scriptsize TIPO:} \\
\textbf{OPTATIVO}
\end{minipage}
%
\begin{minipage}[t]{0.5\textwidth}
\textsf{\scriptsize CRÉDITOS:} \\
\textbf{4}
\end{minipage}}

\noindent \fbox{
\begin{minipage}[t]{0.3\textwidth}
\textsf{\scriptsize CARGA HORÁRIA TOTAL:} \\
\textbf{60}
\end{minipage}
%
\begin{minipage}[t]{0.19\textwidth}
\textsf{\scriptsize TEÓRICA:} \\
\textbf{60}
\end{minipage}
%
\begin{minipage}[t]{0.19\textwidth}
\textsf{\scriptsize PRÁTICA:} \\
\textbf{0}
\end{minipage}
%
\begin{minipage}[t]{0.30\textwidth}
\textsf{\scriptsize EAD-SEMIPRESENCIAL:} \\
\textbf{0}
\end{minipage}}

\noindent \fbox{
\begin{minipage}[t]{\textwidth}
\textsf{\scriptsize PRÉ-REQUISITOS:}
\begin{itemize}
\item
  CCMP3006 ALGORITMOS E ESTRUTURA DE DADOS I
\item
  CCMP3016 ALGORITMOS E ESTRUTURA DE DADOS II
\item
  CCMP3056 INTRODUÇÃO À COMPUTAÇÃO C
\item
  CCMP3057 INTRODUÇÃO À PROGRAMAÇÃO
\end{itemize}
\end{minipage}}

\noindent \fbox{
\begin{minipage}[t]{\textwidth}
\textsf{\scriptsize CORREQUISITOS:} 
Não há.
\end{minipage}}

\noindent \fbox{
\begin{minipage}[t]{\textwidth}
\textsf{\scriptsize REQUISITO DE CARGA HORÁRIA:} 
Não há.
\end{minipage}}

\noindent \fbox{
\begin{minipage}[t]{\textwidth}
\textsf{\scriptsize EMENTA:} \\
Introdução aos fundamentos básicos em segurança de redes de
computadores. Legislação, políticas, normas e ética em segurança de
redes de computadores. Princípios de criptografia. Estudo dos mecanismos
e ferramentas para segurança de redes de computadores. Tecnologias
emergentes em segurança de redes de computadores.
\end{minipage}}

\noindent \fbox{
\begin{minipage}[t]{\textwidth}
\textsf{\scriptsize BIBLIOGRAFIA BÁSICA:}
\begin{enumerate}
\def\labelenumi{\arabic{enumi}.}
\item
  Stallings, William. ``Criptografia e segurança de redes: princípios e
  práticas''. 4a ed.~São Paulo: Pearson, 2011.
\item
  Nakamura, Emilio T.; Geus, Paulo L. Segurança de Redes em Ambientes
  Corporativos. São Paulo: Editora Novatec, 2007.
\item
  Carvalho, Luciano Gonçalves de. ``Segurança de redes''. Rio de
  Janeiro: Ciência Moderna, 2005.
\end{enumerate}
\end{minipage}}

\noindent \fbox{
\begin{minipage}[t]{\textwidth}
\textsf{\scriptsize BIBLIOGRAFIA COMPLEMENTAR:}
\begin{enumerate}
\def\labelenumi{\arabic{enumi}.}
\item
  Burnett, Steve; Paine, Stephen. ``Criptografia e Segurança''. O Guia
  Oficial RSA. 1ª ed.~Rio de Janeiro: Editora Campos, 2002.
\item
  HOWARD, Michael; LEBLANC, David. ``Escrevendo código seguro:
  estratégias e técnicas práticas para codificação segura de aplicativos
  em um mundo em rede''. Porto Alegre: Bookman, 2005. 701 p.
\item
  TRIGO, Clodonil Honorio; MELO, Sandro Pereira de. ``Projeto de
  segurança em software livre''. Rio de Janeiro: Alta Books, 2004. 193
  p.
\item
  WADLOW, Thomas. ``Segurança de Redes: Projeto e Gerenciamento de Redes
  Seguras''. Rio de Janeiro, Ed. Campus, 2000.
\item
  IEEE Transactions on Information Forensics and Security (portal
  CAPES).
\end{enumerate}
\end{minipage}}
\end{scriptsize}

\newpage