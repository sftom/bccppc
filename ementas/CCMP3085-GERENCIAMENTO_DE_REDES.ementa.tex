\begin{figure*}
    \centering
    \includegraphics[width=2.57cm]{./images/brasao_da_republica.jpeg}
    
    \textsf{Ministério da Educação}\\
    \textsf{Universidade Federal do Agreste de Pernambuco}\\
    \textsf{Bacharelado em Ciência da Computação}
\end{figure*}

\vspace{2cm}

\begin{scriptsize}
\noindent \fbox{
\begin{minipage}[t]{\textwidth}
\textsf{\scriptsize COMPONENTE CURRICULAR:} \\
\textbf{GERENCIAMENTO DE REDES} \\
\textsf{\scriptsize CÓDIGO:} \\
\textbf{CCMP3085}
\end{minipage}}

\noindent \fbox{
\begin{minipage}[t]{0.293\textwidth}
\textsf{\scriptsize PERÍODO A SER OFERTADO:} \\
\textbf{0}
\end{minipage}
%
\begin{minipage}[t]{0.7\textwidth}
\textsf{\scriptsize NÚCLEO DE FORMAÇÃO:} \\
\textbf{COMPONENTES OPTATIVOS ÁREA TEMÁTICA REDES E SISTEMAS
DISTRIBUÍDOS}
\end{minipage}}

\noindent \fbox{
\begin{minipage}[t]{0.493\textwidth}
\textsf{\scriptsize TIPO:} \\
\textbf{OPTATIVO}
\end{minipage}
%
\begin{minipage}[t]{0.5\textwidth}
\textsf{\scriptsize CRÉDITOS:} \\
\textbf{4}
\end{minipage}}

\noindent \fbox{
\begin{minipage}[t]{0.3\textwidth}
\textsf{\scriptsize CARGA HORÁRIA TOTAL:} \\
\textbf{60}
\end{minipage}
%
\begin{minipage}[t]{0.19\textwidth}
\textsf{\scriptsize TEÓRICA:} \\
\textbf{60}
\end{minipage}
%
\begin{minipage}[t]{0.19\textwidth}
\textsf{\scriptsize PRÁTICA:} \\
\textbf{0}
\end{minipage}
%
\begin{minipage}[t]{0.30\textwidth}
\textsf{\scriptsize EAD-SEMIPRESENCIAL:} \\
\textbf{0}
\end{minipage}}

\noindent \fbox{
\begin{minipage}[t]{\textwidth}
\textsf{\scriptsize PRÉ-REQUISITOS:}
\begin{itemize}
\item
  CCMP3006 ALGORITMOS E ESTRUTURA DE DADOS I
\item
  CCMP3016 ALGORITMOS E ESTRUTURA DE DADOS II
\item
  CCMP3023 REDE DE COMPUTADORES
\item
  CCMP3056 INTRODUÇÃO À COMPUTAÇÃO C
\item
  CCMP3057 INTRODUÇÃO À PROGRAMAÇÃO
\end{itemize}
\end{minipage}}

\noindent \fbox{
\begin{minipage}[t]{\textwidth}
\textsf{\scriptsize CORREQUISITOS:} 
Não há.
\end{minipage}}

\noindent \fbox{
\begin{minipage}[t]{\textwidth}
\textsf{\scriptsize REQUISITO DE CARGA HORÁRIA:} 
Não há.
\end{minipage}}

\noindent \fbox{
\begin{minipage}[t]{\textwidth}
\textsf{\scriptsize EMENTA:} \\
Princípios, organização e métodos de administração de rede; Tecnologias
para operação e gerência de rede; Rede de gerência de telecomunicações
TMN; Recursos humanos para administração de rede; Plataformas de
gerência de redes; e Aplicações de gerência de rede.
\end{minipage}}

\noindent \fbox{
\begin{minipage}[t]{\textwidth}
\textsf{\scriptsize BIBLIOGRAFIA BÁSICA:}
\begin{enumerate}
\def\labelenumi{\arabic{enumi}.}
\item
  Stallings,William. SNMP, SNMPv2, SNMPv3, and RMON 1 and RMON 2. Third
  Edition. Pearson, 1999.
\item
  Burgess, Mark. Princípios de Administração de redes e sistemas. 3ª
  ed.~São Paulo: LTC, 2006.
\item
  Kusore, James.; Ross, Keith w. Redes de Computadores e a Internet: uma
  abordagem top-down. 3ª Edição. São Paulo: Peason Addison Wesleey,
  2008.
\item
  Tenenbaum, Andrew S. Redes de Computadores. 4ª Edição. Rio de
  Jeaneiro: Editora Campus, 2002.
\item
  Comer, Douglas E. Interligação de Redes com TCP/IP, Volumes I e II. 5ª
  Edição. Prentice Hall, 2006.
\end{enumerate}
\end{minipage}}

\noindent \fbox{
\begin{minipage}[t]{\textwidth}
\textsf{\scriptsize BIBLIOGRAFIA COMPLEMENTAR:}
\begin{enumerate}
\def\labelenumi{\arabic{enumi}.}
\item
  SOARES, LEMOS e COLCHER -- Redes Locais -- Das LANs, MANs e WANs às
  redes ATM. 2° Edição. Ed. Campus, 1995
\item
  Anderson, Al; Benedetti, Ryan. Redes de Computadores - Use a Cabeça!
  Alta Books.
\item
  Torres, Gabriel. Redes De Computadores. Novaterra.
\item
  Cardoso, Fernanda Caetano. Servidores De Internet Embarcada. Ciência
  Moderna.
\item
  Gast, Matthew S. 802.11 Wireless Networks - The Definitive Guide.
  Oreilly \& Assoc
\end{enumerate}
\end{minipage}}
\end{scriptsize}

\newpage