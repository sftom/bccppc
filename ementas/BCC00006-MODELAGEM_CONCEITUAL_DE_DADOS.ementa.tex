\begin{figure*}
    \centering
    \includegraphics[width=2.57cm]{./images/brasao_da_republica.jpeg}
    
    \textsf{Ministério da Educação}\\
    \textsf{Universidade Federal do Agreste de Pernambuco}\\
    \textsf{Bacharelado em Ciência da Computação}
\end{figure*}

\vspace{0.5cm}

\begin{scriptsize}
\noindent \fbox{
\begin{minipage}[t]{\textwidth}
\textsf{\scriptsize COMPONENTE CURRICULAR:} \\
\textbf{MODELAGEM CONCEITUAL DE DADOS} \\
\textsf{\scriptsize CÓDIGO:} \textbf{BCC00006}
\end{minipage}}

\noindent \fbox{
\begin{minipage}[t]{0.293\textwidth}
\textsf{\scriptsize PERÍODO A SER OFERTADO:} \\
\textbf{0}
\end{minipage}
%
\begin{minipage}[t]{0.7\textwidth}
\textsf{\scriptsize NÚCLEO DE FORMAÇÃO:} \\
\textbf{COMPONENTES OPTATIVOS ÁREA TEMÁTICA BANCO DE DADOS}
\end{minipage}}

\noindent \fbox{
\begin{minipage}[t]{0.493\textwidth}
\textsf{\scriptsize TIPO:} \textbf{OPTATIVO}
\end{minipage}
%
\begin{minipage}[t]{0.5\textwidth}
\textsf{\scriptsize CRÉDITOS:} \textbf{4}
\end{minipage}}

\noindent \fbox{
\begin{minipage}[t]{0.3\textwidth}
\textsf{\scriptsize CARGA HORÁRIA TOTAL:} 
\textbf{60}
\end{minipage}
%
\begin{minipage}[t]{0.19\textwidth}
\textsf{\scriptsize TEÓRICA:} 
\textbf{30}
\end{minipage}
%
\begin{minipage}[t]{0.19\textwidth}
\textsf{\scriptsize PRÁTICA:} 
\textbf{30}
\end{minipage}
%
\begin{minipage}[t]{0.30\textwidth}
\textsf{\scriptsize EAD-SEMIPRESENCIAL:} 
\textbf{0}
\end{minipage}}

\noindent \fbox{
\begin{minipage}[t]{\textwidth}
\textsf{\scriptsize PRÉ-REQUISITOS:}
CCMP3066 BANCO DE DADOS I
\end{minipage}}

\noindent \fbox{
\begin{minipage}[t]{\textwidth}
\textsf{\scriptsize CORREQUISITOS:} 
Não há.
\end{minipage}}

\noindent \fbox{
\begin{minipage}[t]{\textwidth}
\textsf{\scriptsize REQUISITO DE CARGA HORÁRIA:} 
Não há.
\end{minipage}}

\noindent \fbox{
\begin{minipage}[t]{\textwidth}
\textsf{\scriptsize EMENTA:} \\
Conceitos Básicos. Análise de Requisitos para Projeto Conceitual do
Banco de Dados. Verificação do Projeto Conceitual do Banco de Dados.
Estratégias para Especificação do Projeto Conceitual de Banco de Dados.
Aspectos Avançados de Projeto Conceitual de Banco de Dados com o Modelo
Entidade-Relacionamento (MER) e com a Linguagem de Modelagem Unificada
(UML). Metamodelos e UML Profile. Ferramentas CASE. Ontologias e Modelos
Conceituais de Banco de dados. Projeto Conceitual de Data Warehouse.
Tópicos Especiais. Projeto Prático.
\end{minipage}}

\noindent \fbox{
\begin{minipage}[t]{\textwidth}
\textsf{\scriptsize BIBLIOGRAFIA BÁSICA:}
\begin{enumerate}
\def\labelenumi{\arabic{enumi}.}
\item
  SILBERSCHATZ, Abraham et al.~Sistemas de bancos de dados. 5. ed.~778
  p.~São Paulo : Makron Books, 2006
\item
  TEOREY et. al.~- Projeto e Modelagem de Banco De Dados, Elsevier.
  2007.
\item
  ELSMARI \& NAVATHE. Sistemas de Banco de Dados. 6. Ed. 744p. Pearson.
  2011.
\end{enumerate}
\end{minipage}}

\noindent \fbox{
\begin{minipage}[t]{\textwidth}
\textsf{\scriptsize BIBLIOGRAFIA COMPLEMENTAR:}
\begin{enumerate}
\def\labelenumi{\arabic{enumi}.}
\item
  REUSER, Carlos Alberto. Projeto de Banco de Dados. 6. Ed. 282p. Ed.
  Bookman 2009.
\item
  RAMAKRISHNAN, GEHRKE. Sistema de Gerenciamento de Banco de Dados. 3.
  Ed. 884 páginas, Mcgraw Hill 2008.
\item
  DATE, C. J. Introdução a Sistemas de Banco de dados. 8. Ed. 900 p.~Rio
  de Janeiro : Campus, 2004
\end{enumerate}
\end{minipage}}
\end{scriptsize}

\newpage