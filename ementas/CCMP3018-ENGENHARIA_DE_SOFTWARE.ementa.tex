\begin{figure*}
    \centering
    \includegraphics[width=2.57cm]{./images/brasao_da_republica.jpeg}
    
    \textsf{Ministério da Educação}\\
    \textsf{Universidade Federal do Agreste de Pernambuco}\\
    \textsf{Bacharelado em Ciência da Computação}
\end{figure*}

\vspace{0.5cm}

\begin{scriptsize}
\noindent \fbox{
\begin{minipage}[t]{\textwidth}
\textsf{\scriptsize COMPONENTE CURRICULAR:} \\
\textbf{ENGENHARIA DE SOFTWARE} \\
\textsf{\scriptsize CÓDIGO:} \textbf{CCMP3018}
\end{minipage}}

\noindent \fbox{
\begin{minipage}[t]{0.293\textwidth}
\textsf{\scriptsize PERÍODO A SER OFERTADO:} \\
\textbf{4}
\end{minipage}
%
\begin{minipage}[t]{0.7\textwidth}
\textsf{\scriptsize NÚCLEO DE FORMAÇÃO:} \\
\textbf{CICLO PROFISSIONAL OU TRONCO COMUM}
\end{minipage}}

\noindent \fbox{
\begin{minipage}[t]{0.493\textwidth}
\textsf{\scriptsize TIPO:} \textbf{OBRIGATÓRIO}
\end{minipage}
%
\begin{minipage}[t]{0.5\textwidth}
\textsf{\scriptsize CRÉDITOS:} \textbf{4}
\end{minipage}}

\noindent \fbox{
\begin{minipage}[t]{0.3\textwidth}
\textsf{\scriptsize CARGA HORÁRIA TOTAL:} 
\textbf{60}
\end{minipage}
%
\begin{minipage}[t]{0.19\textwidth}
\textsf{\scriptsize TEÓRICA:} 
\textbf{60}
\end{minipage}
%
\begin{minipage}[t]{0.19\textwidth}
\textsf{\scriptsize PRÁTICA:} 
\textbf{0}
\end{minipage}
%
\begin{minipage}[t]{0.30\textwidth}
\textsf{\scriptsize EAD-SEMIPRESENCIAL:} 
\textbf{0}
\end{minipage}}

\noindent \fbox{
\begin{minipage}[t]{\textwidth}
\textsf{\scriptsize PRÉ-REQUISITOS:}
\begin{itemize}
\item
  CCMP3017 PROGRAMAÇÃO ORIENTADA AO OBJETO
\item
  CCMP3057 INTRODUÇÃO À PROGRAMAÇÃO
\end{itemize}
\end{minipage}}

\noindent \fbox{
\begin{minipage}[t]{\textwidth}
\textsf{\scriptsize CORREQUISITOS:} 
Não há.
\end{minipage}}

\noindent \fbox{
\begin{minipage}[t]{\textwidth}
\textsf{\scriptsize REQUISITO DE CARGA HORÁRIA:} 
Não há.
\end{minipage}}

\noindent \fbox{
\begin{minipage}[t]{\textwidth}
\textsf{\scriptsize EMENTA:} \\
Contextualização da Engenharia de Software; Fundamentação dos Princípios
da Engenharia de Software; Conceituação de Produto e Processo de
Software; Tipos de Processos de Software; Comparação entre os Paradigmas
de Desenvolvimento Software; Caracterização do Projeto de Software; UML;
Gerenciamento de Projetos; Gerenciamento Ágil; Processo de Engenharia de
Requisito; Requisitos; Testes de Software; Estilos Arquiteturais;
Evolução e Refatoração; Definição de Qualidade de Software.
\end{minipage}}

\noindent \fbox{
\begin{minipage}[t]{\textwidth}
\textsf{\scriptsize BIBLIOGRAFIA BÁSICA:}
\begin{enumerate}
\def\labelenumi{\arabic{enumi}.}
\item
  Armando Fox e David Patterson. Construindo Software como Serviço: Uma
  Abordagem Ágil Usando Computação em Nuvem. Strawberry Canyon LLC.
  2015.
\item
  Marco Tulio Valente. Engenharia de Software Moderna. Editora
  Independente. 2020.
\item
  FOX, Armando, PATTERSON, David. Construindo Software como Serviço
  (SaaS): Uma Abordagem Ágil Usando Computação em Nuvem. Strawberry
  Canyon LCC, 2015.
\end{enumerate}
\end{minipage}}

\noindent \fbox{
\begin{minipage}[t]{\textwidth}
\textsf{\scriptsize BIBLIOGRAFIA COMPLEMENTAR:}
\begin{enumerate}
\def\labelenumi{\arabic{enumi}.}
\item
  Scott Chacon e Ben Straub. Pro Git. Apress. Second Edition.
\item
  Robert C. Martin. Clean Code: A Handbook of Agile Software
  Craftsmanship. First Edition.
\item
  Obie Fernandez. The Rails 5 Way. Addison-Wesley.
\item
  Dave Thomas. Programming Ruby 1.9 e 2.0 the Pragmatic Programmers
  Guide. 2013.
\item
  Obie Fernandez. The Rails 5 Way. Addison-Wesley.
\end{enumerate}
\end{minipage}}
\end{scriptsize}

\newpage