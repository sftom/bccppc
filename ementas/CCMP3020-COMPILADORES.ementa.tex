\begin{figure*}
    \centering
    \includegraphics[width=2.57cm]{./images/brasao_da_republica.jpeg}
    
    \textsf{Ministério da Educação}\\
    \textsf{Universidade Federal do Agreste de Pernambuco}\\
    \textsf{Bacharelado em Ciência da Computação}
\end{figure*}

\vspace{0.5cm}

\begin{scriptsize}
\noindent \fbox{
\begin{minipage}[t]{\textwidth}
\textsf{\scriptsize COMPONENTE CURRICULAR:} \\
\textbf{COMPILADORES} \\
\textsf{\scriptsize CÓDIGO:} \textbf{CCMP3020}
\end{minipage}}

\noindent \fbox{
\begin{minipage}[t]{0.293\textwidth}
\textsf{\scriptsize PERÍODO A SER OFERTADO:} \\
\textbf{6}
\end{minipage}
%
\begin{minipage}[t]{0.7\textwidth}
\textsf{\scriptsize NÚCLEO DE FORMAÇÃO:} \\
\textbf{CICLO PROFISSIONAL OU TRONCO COMUM}
\end{minipage}}

\noindent \fbox{
\begin{minipage}[t]{0.493\textwidth}
\textsf{\scriptsize TIPO:} \textbf{OBRIGATÓRIO}
\end{minipage}
%
\begin{minipage}[t]{0.5\textwidth}
\textsf{\scriptsize CRÉDITOS:} \textbf{4}
\end{minipage}}

\noindent \fbox{
\begin{minipage}[t]{0.3\textwidth}
\textsf{\scriptsize CARGA HORÁRIA TOTAL:} 
\textbf{60}
\end{minipage}
%
\begin{minipage}[t]{0.19\textwidth}
\textsf{\scriptsize TEÓRICA:} 
\textbf{60}
\end{minipage}
%
\begin{minipage}[t]{0.19\textwidth}
\textsf{\scriptsize PRÁTICA:} 
\textbf{0}
\end{minipage}
%
\begin{minipage}[t]{0.30\textwidth}
\textsf{\scriptsize EAD-SEMIPRESENCIAL:} 
\textbf{0}
\end{minipage}}

\noindent \fbox{
\begin{minipage}[t]{\textwidth}
\textsf{\scriptsize PRÉ-REQUISITOS:}
\begin{itemize}
\item
  CCMP3006 ALGORITMOS E ESTRUTURA DE DADOS I
\item
  CCMP3016 ALGORITMOS E ESTRUTURA DE DADOS II
\item
  CCMP3057 INTRODUÇÃO À PROGRAMAÇÃO
\item
  CCMP3059 MATEMÁTICA DISCRETA
\item
  CCMP3068 TEORIA DA COMPUTAÇÃO
\item
  MATM3008 LÓGICA MATEMÁTICA
\end{itemize}
\end{minipage}}

\noindent \fbox{
\begin{minipage}[t]{\textwidth}
\textsf{\scriptsize CORREQUISITOS:} 
Não há.
\end{minipage}}

\noindent \fbox{
\begin{minipage}[t]{\textwidth}
\textsf{\scriptsize REQUISITO DE CARGA HORÁRIA:} 
Não há.
\end{minipage}}

\noindent \fbox{
\begin{minipage}[t]{\textwidth}
\textsf{\scriptsize EMENTA:} \\
Processadores de linguagem: compilador e interpretador. Introdução à
compilação.Fases da compilação. Ambiguidade. Relações sobre gramáticas.
Análise léxica. Análise sintática ascendente e descendente. Tabelas de
símbolos. Esquemas de tradução. Análise semântica. Geração de código
intermediário. Introdução à otimização de código. Ambientes de execução.
Projeto de um compilador simplificado.
\end{minipage}}

\noindent \fbox{
\begin{minipage}[t]{\textwidth}
\textsf{\scriptsize BIBLIOGRAFIA BÁSICA:}
\begin{enumerate}
\def\labelenumi{\arabic{enumi}.}
\item
  AHO, A. V.; SETHI, R.; ULLMAN, J. D.
  \textbf{Compiladores: Princípios, Técnicas e Ferramentas}. 2ª edição -
  São Paulo: Pearson Addison --Wesley, 2008.
\item
  PRICE, A. M., TOSCANI, S. S.
  \textbf{Implementação de Linguagens de Programação: Compiladores}. 3.
  Ed. -- Porto Alegre: Bookman: Instituto de Informática da UFRGS, 2008.
\item
  DELAMARO, M. E.
  \textbf{Como Construir um Compilador Utilizando Ferramentas Java}.
  Novatec, 2004.
\end{enumerate}
\end{minipage}}

\noindent \fbox{
\begin{minipage}[t]{\textwidth}
\textsf{\scriptsize BIBLIOGRAFIA COMPLEMENTAR:}
\begin{enumerate}
\def\labelenumi{\arabic{enumi}.}
\item
  LOUDEN, Kenneth. C. \textbf{Compiladores: princípios e práticas}. São
  Paulo: Cengage Learning.
\item
  KEITH, Cooper. \textbf{Construindo Compiladores}. GEN LTC, 1ª edição,
  2013.
\item
  SANTOS, Pedro Reis; LANGLOIS, Thibault.
  \textbf{Compiladores: Da Teoria à Prática}. Lisboa: FCA, 1ª edição,
  2014.
\item
  SETZER, V. \textbf{A Construção de Um Compilador}. Rio de Janeiro:
  Campus, 1986.
\end{enumerate}
\end{minipage}}
\end{scriptsize}

\newpage