\begin{figure*}
    \centering
    \includegraphics[width=2.57cm]{./images/brasao_da_republica.jpeg}
    
    \textsf{Ministério da Educação}\\
    \textsf{Universidade Federal do Agreste de Pernambuco}\\
    \textsf{Bacharelado em Ciência da Computação}
\end{figure*}

\vspace{0.5cm}

\begin{scriptsize}
\noindent \fbox{
\begin{minipage}[t]{\textwidth}
\textsf{\scriptsize COMPONENTE CURRICULAR:} \\
\textbf{CÁLCULO III} \\
\textsf{\scriptsize CÓDIGO:} \textbf{MATM3007}
\end{minipage}}

\noindent \fbox{
\begin{minipage}[t]{0.293\textwidth}
\textsf{\scriptsize PERÍODO A SER OFERTADO:} \\
\textbf{0}
\end{minipage}
%
\begin{minipage}[t]{0.7\textwidth}
\textsf{\scriptsize NÚCLEO DE FORMAÇÃO:} \\
\textbf{COMPONENTES OPTATIVOS ÁREA TEMÁTICA MATEMÁTICA E SIMULAÇÃO
COMPUTACIONAL}
\end{minipage}}

\noindent \fbox{
\begin{minipage}[t]{0.493\textwidth}
\textsf{\scriptsize TIPO:} \textbf{OPTATIVO}
\end{minipage}
%
\begin{minipage}[t]{0.5\textwidth}
\textsf{\scriptsize CRÉDITOS:} \textbf{4}
\end{minipage}}

\noindent \fbox{
\begin{minipage}[t]{0.3\textwidth}
\textsf{\scriptsize CARGA HORÁRIA TOTAL:} 
\textbf{60}
\end{minipage}
%
\begin{minipage}[t]{0.19\textwidth}
\textsf{\scriptsize TEÓRICA:} 
\textbf{60}
\end{minipage}
%
\begin{minipage}[t]{0.19\textwidth}
\textsf{\scriptsize PRÁTICA:} 
\textbf{0}
\end{minipage}
%
\begin{minipage}[t]{0.30\textwidth}
\textsf{\scriptsize EAD-SEMIPRESENCIAL:} 
\textbf{0}
\end{minipage}}

\noindent \fbox{
\begin{minipage}[t]{\textwidth}
\textsf{\scriptsize PRÉ-REQUISITOS:}
\begin{itemize}
\item
  MATM3031 CÁLCULO PARA COMPUTAÇÃO I
\item
  MATM3032 CÁLCULO PARA COMPUTAÇÃO II
\end{itemize}
\end{minipage}}

\noindent \fbox{
\begin{minipage}[t]{\textwidth}
\textsf{\scriptsize CORREQUISITOS:} 
Não há.
\end{minipage}}

\noindent \fbox{
\begin{minipage}[t]{\textwidth}
\textsf{\scriptsize REQUISITO DE CARGA HORÁRIA:} 
Não há.
\end{minipage}}

\noindent \fbox{
\begin{minipage}[t]{\textwidth}
\textsf{\scriptsize EMENTA:} \\
Conceitos introdutórios e classificação das equações diferenciais.
Equações diferenciais de primeira ordem. Obtenção de soluções de
equações lineares, separáveis, exatas, não exatas com fatores
integrantes simples, etc\ldots{} Algumas aplicações das equações de
primeira ordem. Equações diferenciais de segunda ordem, propriedades
gerais das soluções, soluções das homogêneas com coeficientes
constantes. Equações lineares não homogêneas, método dos coeficientes a
determinar e método da variação dos parâmetros. Estudo introdutório das
oscilações lineares livres e forçadas. Transformada de Laplace,
propriedades fundamentais, e utilização para resolução de equações
diferenciais. Equação do calor. Método de separação de variáveis. Séries
de Fourier, propriedades básicas e aplicações. Equação da onda,
vibrações em uma corda elástica. Equação de Laplace.
\end{minipage}}

\noindent \fbox{
\begin{minipage}[t]{\textwidth}
\textsf{\scriptsize BIBLIOGRAFIA BÁSICA:}
\begin{enumerate}
\def\labelenumi{\arabic{enumi}.}
\item
  STEWART, James. Cálculo V.2. 2ed. São Paulo: Cengage Learning, 2010.
\item
  LEITHOLD, Louis. O Cálculo com Geometria Analítica Volume 2. 3ed. São
  Paulo: Harbra, 1994.
\item
  SIMMONS, George F. Cálculo com Geometria Analítica V. 2. São Paulo:
  Pearson Makron Books, 1988.
\end{enumerate}
\end{minipage}}

\noindent \fbox{
\begin{minipage}[t]{\textwidth}
\textsf{\scriptsize BIBLIOGRAFIA COMPLEMENTAR:}
\begin{enumerate}
\def\labelenumi{\arabic{enumi}.}
\item
  MUNEM, Foulis. Cálculo V.2. Rio de Janeiro: LTC, 1982.
\item
  GUIDORIZZI, Hamilton Luiz. Um Curso de Cálculo V.2. 5ed. Rio de
  Janeiro: LTC, 2008.
\item
  GUIDORIZZI, Hamilton Luiz. Um Curso de Cálculo V.3. 5ed. Rio de
  Janeiro: LTC, 2008.
\item
  ÁVILA, Geraldo. Cálculo das Funções de uma Variável V.2. 7ed. Rio de
  Janeiro: LTC, 2003.
\item
  ANTON, Howard. Cálculo um Novo Horizonte V.2. 6ed. Porto alegre:
  Bookman, 2000.
\end{enumerate}
\end{minipage}}
\end{scriptsize}

\newpage