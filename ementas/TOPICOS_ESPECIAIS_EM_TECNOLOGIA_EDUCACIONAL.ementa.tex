\begin{figure*}
    \centering
    \includegraphics[width=2.57cm]{./images/brasao_da_republica.jpeg}
    
    \textsf{Ministério da Educação}\\
    \textsf{Universidade Federal do Agreste de Pernambuco}\\
    \textsf{Bacharelado em Ciência da Computação}
\end{figure*}

\vspace{0.5cm}

\begin{scriptsize}
\noindent \fbox{
\begin{minipage}[t]{\textwidth}
\textsf{\scriptsize COMPONENTE CURRICULAR:} \\
\textbf{TÓPICOS ESPECIAIS EM TECNOLOGIA EDUCACIONAL} \\
\textsf{\scriptsize CÓDIGO:} \textbf{}
\end{minipage}}

\noindent \fbox{
\begin{minipage}[t]{0.293\textwidth}
\textsf{\scriptsize PERÍODO A SER OFERTADO:} \\
\textbf{0}
\end{minipage}
%
\begin{minipage}[t]{0.7\textwidth}
\textsf{\scriptsize NÚCLEO DE FORMAÇÃO:} \\
\textbf{COMPONENTES OPTATIVOS ÁREA TEMÁTICA TECNOLOGIA EDUCACIONAL}
\end{minipage}}

\noindent \fbox{
\begin{minipage}[t]{0.493\textwidth}
\textsf{\scriptsize TIPO:} \textbf{OPTATIVO}
\end{minipage}
%
\begin{minipage}[t]{0.5\textwidth}
\textsf{\scriptsize CRÉDITOS:} \textbf{4}
\end{minipage}}

\noindent \fbox{
\begin{minipage}[t]{0.3\textwidth}
\textsf{\scriptsize CARGA HORÁRIA TOTAL:} 
\textbf{60}
\end{minipage}
%
\begin{minipage}[t]{0.19\textwidth}
\textsf{\scriptsize TEÓRICA:} 
\textbf{60}
\end{minipage}
%
\begin{minipage}[t]{0.19\textwidth}
\textsf{\scriptsize PRÁTICA:} 
\textbf{0}
\end{minipage}
%
\begin{minipage}[t]{0.30\textwidth}
\textsf{\scriptsize EAD-SEMIPRESENCIAL:} 
\textbf{0}
\end{minipage}}

\noindent \fbox{
\begin{minipage}[t]{\textwidth}
\textsf{\scriptsize PRÉ-REQUISITOS:}
CCMP3056 INTRODUÇÃO A COMPUTAÇÃO C
\end{minipage}}

\noindent \fbox{
\begin{minipage}[t]{\textwidth}
\textsf{\scriptsize CORREQUISITOS:} 
Não há.
\end{minipage}}

\noindent \fbox{
\begin{minipage}[t]{\textwidth}
\textsf{\scriptsize REQUISITO DE CARGA HORÁRIA:} 
Não há.
\end{minipage}}

\noindent \fbox{
\begin{minipage}[t]{\textwidth}
\textsf{\scriptsize EMENTA:} \\
Estudo de teorias de aprendizagem multimídia. Elaboração de projetos
pedagógicos utilizando recursos digitais. Novas tecnologias da
informação aplicadas ao ensino formal e informal.
\end{minipage}}

\noindent \fbox{
\begin{minipage}[t]{\textwidth}
\textsf{\scriptsize BIBLIOGRAFIA BÁSICA:}
\begin{enumerate}
\def\labelenumi{\arabic{enumi}.}
\item
  FRANCO, Sergio R.K. Informática na Educação: Estudos
  Interdisciplinares. Editora da UFRGS, 2004.
\item
  LEVY, Pierre - As tecnologias da Inteligência- O futuro do pensamento
  na era da informática. São Paulo: Editora 34, 2004, 13a. Edição.
\item
  PAPERT, Seymour. A máquina das crianças: repensando a escola na era da
  informática. Porto Alegre: Artes Médicas, 1994.
\end{enumerate}
\end{minipage}}

\noindent \fbox{
\begin{minipage}[t]{\textwidth}
\textsf{\scriptsize BIBLIOGRAFIA COMPLEMENTAR:}
\begin{enumerate}
\def\labelenumi{\arabic{enumi}.}
\item
  ARAUJO, Alberto.E.P. Apostila de Informática na Educação I. UFRPE.
  2008.
\item
  COX, Kenia. Informática na Educação Escolar, Autores Associados, 2003.
\item
  LEVY, Pierre. Cibercultura, Editora 34, 1999.
\item
  RBIE - Revista Brasileira de Informática na Educação ISSN 1414-5685.
\item
  RENOTE - Revista Novas Tecnologias na Educação ISSN 1679-1916.
\end{enumerate}
\end{minipage}}
\end{scriptsize}

\newpage