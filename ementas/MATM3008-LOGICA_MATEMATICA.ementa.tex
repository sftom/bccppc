\begin{figure*}
    \centering
    \includegraphics[width=2.57cm]{./images/brasao_da_republica.jpeg}
    
    \textsf{Ministério da Educação}\\
    \textsf{Universidade Federal do Agreste de Pernambuco}\\
    \textsf{Bacharelado em Ciência da Computação}
\end{figure*}

\vspace{2cm}

\begin{scriptsize}
\noindent \fbox{
\begin{minipage}[t]{\textwidth}
\textsf{\scriptsize COMPONENTE CURRICULAR:} \\
\textbf{LÓGICA MATEMÁTICA} \\
\textsf{\scriptsize CÓDIGO:} \\
\textbf{MATM3008}
\end{minipage}}

\noindent \fbox{
\begin{minipage}[t]{0.293\textwidth}
\textsf{\scriptsize PERÍODO A SER OFERTADO:} \\
\textbf{1}
\end{minipage}
%
\begin{minipage}[t]{0.7\textwidth}
\textsf{\scriptsize NÚCLEO DE FORMAÇÃO:} \\
\textbf{CICLO GERAL OU CICLO BÁSICO}
\end{minipage}}

\noindent \fbox{
\begin{minipage}[t]{0.493\textwidth}
\textsf{\scriptsize TIPO:} \\
\textbf{OBRIGATÓRIO}
\end{minipage}
%
\begin{minipage}[t]{0.5\textwidth}
\textsf{\scriptsize CRÉDITOS:} \\
\textbf{4}
\end{minipage}}

\noindent \fbox{
\begin{minipage}[t]{0.3\textwidth}
\textsf{\scriptsize CARGA HORÁRIA TOTAL:} \\
\textbf{60}
\end{minipage}
%
\begin{minipage}[t]{0.19\textwidth}
\textsf{\scriptsize TEÓRICA:} \\
\textbf{60}
\end{minipage}
%
\begin{minipage}[t]{0.19\textwidth}
\textsf{\scriptsize PRÁTICA:} \\
\textbf{0}
\end{minipage}
%
\begin{minipage}[t]{0.30\textwidth}
\textsf{\scriptsize EAD-SEMIPRESENCIAL:} \\
\textbf{0}
\end{minipage}}

\noindent \fbox{
\begin{minipage}[t]{\textwidth}
\textsf{\scriptsize PRÉ-REQUISITOS:}
Não há.
\end{minipage}}

\noindent \fbox{
\begin{minipage}[t]{\textwidth}
\textsf{\scriptsize CORREQUISITOS:} 
Não há.
\end{minipage}}

\noindent \fbox{
\begin{minipage}[t]{\textwidth}
\textsf{\scriptsize REQUISITO DE CARGA HORÁRIA:} 
Não há.
\end{minipage}}

\noindent \fbox{
\begin{minipage}[t]{\textwidth}
\textsf{\scriptsize EMENTA:} \\
Proposições e conectivos; Operações lógicas sobre proposições;
Construção de tabelas-verdade; Tautologias, contradições e
contingencias; Implicação lógica; Equivalência lógica; Álgebra das
proposições; Método dedutivo; Argumentos, regras de inferência; Validade
mediante tabela verdade; Validade mediante regras de inferência; Métodos
de demonstrações; Sentenças abertas; Operações lógicas sobre sentenças
abertas; Quantificadores.
\end{minipage}}

\noindent \fbox{
\begin{minipage}[t]{\textwidth}
\textsf{\scriptsize BIBLIOGRAFIA BÁSICA:}
\begin{enumerate}
\def\labelenumi{\arabic{enumi}.}
\item
  GERSTING, JUDITH L. Fundamentos matemáticos para a ciência da
  computação: um tratamento moderno de matemática discreta. LTC, 2008.
\item
  ALENCAR FILHO, Edgard de, Iniciação à Lógica Matemática. 18. ed.~203
  p, São Paulo : Nobel, 2000.
\item
  DAGHLIAN, Jacob ; Lógica e Álgebra de Boole, 4. ed.~167 p., São Paulo
  : Atlas, 1995.
\end{enumerate}
\end{minipage}}

\noindent \fbox{
\begin{minipage}[t]{\textwidth}
\textsf{\scriptsize BIBLIOGRAFIA COMPLEMENTAR:}
\begin{enumerate}
\def\labelenumi{\arabic{enumi}.}
\item
  DEL PICCHIA, Walter; Métodos Numéricos para Resolução de Problemas
  Lógicos. São Paulo : Edgard Blücher, 1993.
\item
  SALMON, Wesley C. Lógica, 3a Edição. Rio de Janeiro: LTC -- Editora,
  2002
\item
  NOLT, John, ROHATYN, Dennis. Lógica. São Paulo: Schaum McGraw-Hill,
  1991
\item
  OLIVEIRA, A. J. F. de. Lógica e aritmética. Brasília: Editora UnB,
  2004
\item
  SOARES, Edvaldo. Fundamentos de Lógica. Elementos de Lógica Formal e
  Teoria da Argumentação. São Paulo: Atlas S. A., 2003.
\end{enumerate}
\end{minipage}}
\end{scriptsize}

\newpage