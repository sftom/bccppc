\begin{figure*}
    \centering
    \includegraphics[width=2.57cm]{./images/brasao_da_republica.jpeg}
    
    \textsf{Ministério da Educação}\\
    \textsf{Universidade Federal do Agreste de Pernambuco}\\
    \textsf{Bacharelado em Ciência da Computação}
\end{figure*}

\vspace{0.5cm}

\begin{scriptsize}
\noindent \fbox{
\begin{minipage}[t]{\textwidth}
\textsf{\scriptsize COMPONENTE CURRICULAR:} \\
\textbf{TEORIA DA COMPUTAÇÃO} \\
\textsf{\scriptsize CÓDIGO:} \textbf{CCMP3068}
\end{minipage}}

\noindent \fbox{
\begin{minipage}[t]{0.293\textwidth}
\textsf{\scriptsize PERÍODO A SER OFERTADO:} \\
\textbf{5}
\end{minipage}
%
\begin{minipage}[t]{0.7\textwidth}
\textsf{\scriptsize NÚCLEO DE FORMAÇÃO:} \\
\textbf{CICLO GERAL OU CICLO BÁSICO}
\end{minipage}}

\noindent \fbox{
\begin{minipage}[t]{0.493\textwidth}
\textsf{\scriptsize TIPO:} \textbf{OBRIGATÓRIO}
\end{minipage}
%
\begin{minipage}[t]{0.5\textwidth}
\textsf{\scriptsize CRÉDITOS:} \textbf{4}
\end{minipage}}

\noindent \fbox{
\begin{minipage}[t]{0.3\textwidth}
\textsf{\scriptsize CARGA HORÁRIA TOTAL:} 
\textbf{60}
\end{minipage}
%
\begin{minipage}[t]{0.19\textwidth}
\textsf{\scriptsize TEÓRICA:} 
\textbf{60}
\end{minipage}
%
\begin{minipage}[t]{0.19\textwidth}
\textsf{\scriptsize PRÁTICA:} 
\textbf{0}
\end{minipage}
%
\begin{minipage}[t]{0.30\textwidth}
\textsf{\scriptsize EAD-SEMIPRESENCIAL:} 
\textbf{0}
\end{minipage}}

\noindent \fbox{
\begin{minipage}[t]{\textwidth}
\textsf{\scriptsize PRÉ-REQUISITOS:}
CCMP3059 MATEMÁTICA DISCRETA
\end{minipage}}

\noindent \fbox{
\begin{minipage}[t]{\textwidth}
\textsf{\scriptsize CORREQUISITOS:} 
Não há.
\end{minipage}}

\noindent \fbox{
\begin{minipage}[t]{\textwidth}
\textsf{\scriptsize REQUISITO DE CARGA HORÁRIA:} 
Não há.
\end{minipage}}

\noindent \fbox{
\begin{minipage}[t]{\textwidth}
\textsf{\scriptsize EMENTA:} \\
Conceitos Básicos. Alfabetos e Linguagens. Gramáticas. Linguagens
Regulares; Autômatos Finitos; Gramáticas Lineares. Linguagens Livres de
Contexto; Autômato de Pilhas; Gramáticas Livre de Contexto Ambígua.
Formas Normais. Linguagens Recursivamente Enumeráveis e Sensíveis ao
Contexto; Máquina de Turing. Hierarquia de Chomsky. Indecidibilidade.
\end{minipage}}

\noindent \fbox{
\begin{minipage}[t]{\textwidth}
\textsf{\scriptsize BIBLIOGRAFIA BÁSICA:}
\begin{enumerate}
\def\labelenumi{\arabic{enumi}.}
\item
  SIPSER, M. \textbf{Introdução à Teoria da Computação}. São Paulo
  Thomson Learning, 2007.
\item
  HOPCROFT, J. E.; MOTWANI, R.; ULLMAN, J. D.
  \textbf{Introdução à Teoria dos Autômatos, Linguagens e Computação}.
  Editora Campus, 2002.
\item
  MENEZES, P. B. \textbf{Linguagens Formais e Autômatos}. Série Livros
  Didáticos, 6ª Ed. Porto Alegre Sagra Luzzatto, 2008.
\end{enumerate}
\end{minipage}}

\noindent \fbox{
\begin{minipage}[t]{\textwidth}
\textsf{\scriptsize BIBLIOGRAFIA COMPLEMENTAR:}
\begin{enumerate}
\def\labelenumi{\arabic{enumi}.}
\item
  DIVERIO, Tiaraju A.; MENEZES, Paulo F. Blauth.
  \textbf{Teoria da Computação – Máquinas Universais e Computabilidade}.
  Porto Alegre: Bookman, 3ª edição, 2011.
\item
  COELHO, F.; PEDRO NETO, J.
  \textbf{Teoria da Computação - Computabilidade e Complexidade}, 1ª
  Edição, Editora: Escolar Editora/Zamboni, 2010.
\item
  LEWIS, H. R.; PAPPADIMITRIOU, C. H.
  \textbf{Elementos de Teoria da Computação}. Bookman, 2ª edição, 2000.
\item
  SHIELDS, M. W. \textbf{An Introduction to Automata Theory}. Oxford
  Blackwell Scientific Publications, 1987.
\item
  SALOMA, A. \textbf{Formal Languages}. New York Academic Press, 1973.
\end{enumerate}
\end{minipage}}
\end{scriptsize}

\newpage