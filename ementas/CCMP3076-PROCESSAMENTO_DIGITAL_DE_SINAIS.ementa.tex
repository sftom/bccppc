\begin{figure*}
    \centering
    \includegraphics[width=2.57cm]{./images/brasao_da_republica.jpeg}
    
    \textsf{Ministério da Educação}\\
    \textsf{Universidade Federal do Agreste de Pernambuco}\\
    \textsf{Bacharelado em Ciência da Computação}
\end{figure*}

\vspace{2cm}

\begin{scriptsize}
\noindent \fbox{
\begin{minipage}[t]{\textwidth}
\textsf{\scriptsize COMPONENTE CURRICULAR:} \\
\textbf{PROCESSAMENTO DIGITAL DE SINAIS} \\
\textsf{\scriptsize CÓDIGO:} \\
\textbf{CCMP3076}
\end{minipage}}

\noindent \fbox{
\begin{minipage}[t]{0.293\textwidth}
\textsf{\scriptsize PERÍODO A SER OFERTADO:} \\
\textbf{0}
\end{minipage}
%
\begin{minipage}[t]{0.7\textwidth}
\textsf{\scriptsize NÚCLEO DE FORMAÇÃO:} \\
\textbf{COMPONENTES OPTATIVOS ÁREA TEMÁTICA MÍDIA E INTERAÇÃO}
\end{minipage}}

\noindent \fbox{
\begin{minipage}[t]{0.493\textwidth}
\textsf{\scriptsize TIPO:} \\
\textbf{OPTATIVO}
\end{minipage}
%
\begin{minipage}[t]{0.5\textwidth}
\textsf{\scriptsize CRÉDITOS:} \\
\textbf{4}
\end{minipage}}

\noindent \fbox{
\begin{minipage}[t]{0.3\textwidth}
\textsf{\scriptsize CARGA HORÁRIA TOTAL:} \\
\textbf{60}
\end{minipage}
%
\begin{minipage}[t]{0.19\textwidth}
\textsf{\scriptsize TEÓRICA:} \\
\textbf{60}
\end{minipage}
%
\begin{minipage}[t]{0.19\textwidth}
\textsf{\scriptsize PRÁTICA:} \\
\textbf{0}
\end{minipage}
%
\begin{minipage}[t]{0.30\textwidth}
\textsf{\scriptsize EAD-SEMIPRESENCIAL:} \\
\textbf{0}
\end{minipage}}

\noindent \fbox{
\begin{minipage}[t]{\textwidth}
\textsf{\scriptsize PRÉ-REQUISITOS:}
\begin{itemize}
\item
  CCMP3043 RECONHECIMENTO DE PADRÕES
\item
  MATM3032 CÁLCULO PARA COMPUTAÇÃO II
\end{itemize}
\end{minipage}}

\noindent \fbox{
\begin{minipage}[t]{\textwidth}
\textsf{\scriptsize CORREQUISITOS:} 
Não há.
\end{minipage}}

\noindent \fbox{
\begin{minipage}[t]{\textwidth}
\textsf{\scriptsize REQUISITO DE CARGA HORÁRIA:} 
Não há.
\end{minipage}}

\noindent \fbox{
\begin{minipage}[t]{\textwidth}
\textsf{\scriptsize EMENTA:} \\
Representação digital de sinais de áudio, imagens, e vídeo: amostragem,
quantização e aliasing. Transformada Discreta de Fourier e FFT (1D, 2D e
3D). Outras transformações: Transformada de Fourier (Contínua),
Transformada do Coseno Discreta, Transformada z, Transformada de
Walsh-Hadamard, Transformada de Haar. Convolução linear, circular e
secionada. Filtros lineares (FIR) e Filtros recursivos (IIR). Aplicações
de filtros: suavização, interpolação, realce, detecção de bordas e
segmentação. Espaço de transformação no tempo e no espaço, localização e
efeitos no espectro. Bancos de filtros e técnicas de análise-ressíntese.
Compressão: Predição Linear, compressão usando DCT, Compensação
deMovimento. Sinais aleatórios: Representação, Filtros de Wiener e de
Kalman.
\end{minipage}}

\noindent \fbox{
\begin{minipage}[t]{\textwidth}
\textsf{\scriptsize BIBLIOGRAFIA BÁSICA:}
\begin{enumerate}
\def\labelenumi{\arabic{enumi}.}
\item
  S. Allen Broughton and Kurt M. Bryan. Discrete Fourier Analysis and
  Wavelets: Applications to Signal and Image Processing.
  Wiley-Interscience, 2008.
\item
  John W. Woods. Multidimensional Signal, Image and Video Processing and
  Coding. Academic Press, 2006.
\item
  Rafael C. Gonzales and Richard E. Woods. Digital Image Processing, 3rd
  ed.~Prentice Hall, 2007.
\end{enumerate}
\end{minipage}}

\noindent \fbox{
\begin{minipage}[t]{\textwidth}
\textsf{\scriptsize BIBLIOGRAFIA COMPLEMENTAR:}
\begin{enumerate}
\def\labelenumi{\arabic{enumi}.}
\item
  Charles L. Byrne. Signal Processing: a Mathematical Approach. A. K.
  Peters Ltd., 2005.
\item
  Ronald N. Bracewell. Fourier Analysis and Imaging. Springer, 2004.
\item
  Richard W. Hamming. Digital Filters, 3rd ed.~Dover Publications, 1997.
\item
  Alan. V. Oppenheim and Ronald W. Schafer. Discrete-Time Signal
  Processing, 2nd ed.~Prentice Hall, 1999.
\end{enumerate}
\end{minipage}}
\end{scriptsize}

\newpage