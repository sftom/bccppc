\begin{figure*}
    \centering
    \includegraphics[width=2.57cm]{./images/brasao_da_republica.jpeg}
    
    \textsf{Ministério da Educação}\\
    \textsf{Universidade Federal do Agreste de Pernambuco}\\
    \textsf{Bacharelado em Ciência da Computação}
\end{figure*}

\vspace{0.5cm}

\begin{scriptsize}
\noindent \fbox{
\begin{minipage}[t]{\textwidth}
\textsf{\scriptsize COMPONENTE CURRICULAR:} \\
\textbf{EMPREENDIMENTOS EM TIC} \\
\textsf{\scriptsize CÓDIGO:} \textbf{ADMT3018}
\end{minipage}}

\noindent \fbox{
\begin{minipage}[t]{0.293\textwidth}
\textsf{\scriptsize PERÍODO A SER OFERTADO:} \\
\textbf{0}
\end{minipage}
%
\begin{minipage}[t]{0.7\textwidth}
\textsf{\scriptsize NÚCLEO DE FORMAÇÃO:} \\
\textbf{COMPONENTES OPTATIVOS ÁREA TEMÁTICA TECNOLOGIAS DA INFORMAÇÃO}
\end{minipage}}

\noindent \fbox{
\begin{minipage}[t]{0.493\textwidth}
\textsf{\scriptsize TIPO:} \textbf{OPTATIVO}
\end{minipage}
%
\begin{minipage}[t]{0.5\textwidth}
\textsf{\scriptsize CRÉDITOS:} \textbf{4}
\end{minipage}}

\noindent \fbox{
\begin{minipage}[t]{0.3\textwidth}
\textsf{\scriptsize CARGA HORÁRIA TOTAL:} 
\textbf{60}
\end{minipage}
%
\begin{minipage}[t]{0.19\textwidth}
\textsf{\scriptsize TEÓRICA:} 
\textbf{60}
\end{minipage}
%
\begin{minipage}[t]{0.19\textwidth}
\textsf{\scriptsize PRÁTICA:} 
\textbf{0}
\end{minipage}
%
\begin{minipage}[t]{0.30\textwidth}
\textsf{\scriptsize EAD-SEMIPRESENCIAL:} 
\textbf{0}
\end{minipage}}

\noindent \fbox{
\begin{minipage}[t]{\textwidth}
\textsf{\scriptsize PRÉ-REQUISITOS:}
Não há.
\end{minipage}}

\noindent \fbox{
\begin{minipage}[t]{\textwidth}
\textsf{\scriptsize CORREQUISITOS:} 
Não há.
\end{minipage}}

\noindent \fbox{
\begin{minipage}[t]{\textwidth}
\textsf{\scriptsize REQUISITO DE CARGA HORÁRIA:} 
Não há.
\end{minipage}}

\noindent \fbox{
\begin{minipage}[t]{\textwidth}
\textsf{\scriptsize EMENTA:} \\
Importância e Contribuição Econômica e Social do Empreendedorismo;
Empreendedores: Características e comportamentos empreendedores.
Motivações; Inovação tecnológica e empreendedorismo; O sucesso e o
fracasso de novos empreendimentos; Cases de sucesso em computação. Plano
de negócio: Importância, estruturação e apresentação. Pesquisa de
Mercado; Ferramentas Gerenciais para o empreendedor. Incubadoras: o que
são, objetivos; Caminhos a seguir e recursos disponíveis para o
empreendedor: Incubação, MCT, FINEP, Venture Capital, Start-ups.
\end{minipage}}

\noindent \fbox{
\begin{minipage}[t]{\textwidth}
\textsf{\scriptsize BIBLIOGRAFIA BÁSICA:}
\begin{enumerate}
\def\labelenumi{\arabic{enumi}.}
\item
  DRUCKER, P. F. Inovação e Espírito Empreendedor. São Paulo: Pioneira,
  1986.
\item
  DOLABELA, Fernando. Oficina do empreendedor. São Paulo: Cultura, 1999.
\item
  Leite, Emanuel (2002) -- O Fenómeno do Empreendedorismo, Recife,
  Edições Bagaço.
\end{enumerate}
\end{minipage}}

\noindent \fbox{
\begin{minipage}[t]{\textwidth}
\textsf{\scriptsize BIBLIOGRAFIA COMPLEMENTAR:}
\begin{enumerate}
\def\labelenumi{\arabic{enumi}.}
\item
  DEGEN, Ronald Jean. O Empreendedor: empreender como opção de carreira.
  São Paulo: Pearson Prentice Hall, 2009.
\item
  http://www.planodenegocios.com.br - portal brasileiro de plano de
  negócios, com artigos, cursos e eventos, links e bibligrafia sobre o
  tema;
\item
  http://www.empreendedor.com.br - revista para empreendedores;
\item
  http://www.anprotec.org.br - ANPROTEC;
\item
  http://emanueleite.blogspot.com/
\end{enumerate}
\end{minipage}}
\end{scriptsize}

\newpage