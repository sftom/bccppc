\begin{figure*}
    \centering
    \includegraphics[width=2.57cm]{./images/brasao_da_republica.jpeg}
    
    \textsf{Ministério da Educação}\\
    \textsf{Universidade Federal do Agreste de Pernambuco}\\
    \textsf{Bacharelado em Ciência da Computação}
\end{figure*}

\vspace{2cm}

\begin{scriptsize}
\noindent \fbox{
\begin{minipage}[t]{\textwidth}
\textsf{\scriptsize COMPONENTE CURRICULAR:} \\
\textbf{ESPECIFICAÇÃO FORMAL DE SOFTWARE} \\
\textsf{\scriptsize CÓDIGO:} \\
\textbf{BCC00002}
\end{minipage}}

\noindent \fbox{
\begin{minipage}[t]{0.293\textwidth}
\textsf{\scriptsize PERÍODO A SER OFERTADO:} \\
\textbf{0}
\end{minipage}
%
\begin{minipage}[t]{0.7\textwidth}
\textsf{\scriptsize NÚCLEO DE FORMAÇÃO:} \\
\textbf{COMPONENTES OPTATIVOS ÁREA TEMÁTICA ENGENHARIA DE SOFTWARE}
\end{minipage}}

\noindent \fbox{
\begin{minipage}[t]{0.493\textwidth}
\textsf{\scriptsize TIPO:} \\
\textbf{OPTATIVO}
\end{minipage}
%
\begin{minipage}[t]{0.5\textwidth}
\textsf{\scriptsize CRÉDITOS:} \\
\textbf{4}
\end{minipage}}

\noindent \fbox{
\begin{minipage}[t]{0.3\textwidth}
\textsf{\scriptsize CARGA HORÁRIA TOTAL:} \\
\textbf{60}
\end{minipage}
%
\begin{minipage}[t]{0.19\textwidth}
\textsf{\scriptsize TEÓRICA:} \\
\textbf{30}
\end{minipage}
%
\begin{minipage}[t]{0.19\textwidth}
\textsf{\scriptsize PRÁTICA:} \\
\textbf{30}
\end{minipage}
%
\begin{minipage}[t]{0.30\textwidth}
\textsf{\scriptsize EAD-SEMIPRESENCIAL:} \\
\textbf{0}
\end{minipage}}

\noindent \fbox{
\begin{minipage}[t]{\textwidth}
\textsf{\scriptsize PRÉ-REQUISITOS:}
CCMP3059 MATEMÁTICA DISCRETA
\end{minipage}}

\noindent \fbox{
\begin{minipage}[t]{\textwidth}
\textsf{\scriptsize CORREQUISITOS:} 
Não há.
\end{minipage}}

\noindent \fbox{
\begin{minipage}[t]{\textwidth}
\textsf{\scriptsize REQUISITO DE CARGA HORÁRIA:} 
Não há.
\end{minipage}}

\noindent \fbox{
\begin{minipage}[t]{\textwidth}
\textsf{\scriptsize EMENTA:} \\
Especificação de software baseada no paradigma imperativo: conjuntos,
dados, operações, refinamentos sucessivos e implementação (geração de
código). Especificação de software baseada no paradigma orientado a
objetos. Redes de Petri coloridas. Verificação automática de modelos.
\end{minipage}}

\noindent \fbox{
\begin{minipage}[t]{\textwidth}
\textsf{\scriptsize BIBLIOGRAFIA BÁSICA:}
\begin{enumerate}
\def\labelenumi{\arabic{enumi}.}
\item
  CORMEN, T. H., LEISERSON, C. E., RIVEST, R. L., STEIN, C. Algoritmos:
  Teoria e prática. Editora Campus, tradução da 2a edição Americana,
  2002.
\item
  GERSTING, Judith L. Fundamentos Matemáticos para a Ciência da
  Computação. LTC - Livros Técnicos e Científicos, 1982.
\item
  PAPADIMITRIOU, C. H., VAZIRANI, U. V., DASGUPTA, S. Algoritmos.
  McGraw-Hill, 2006.
\end{enumerate}
\end{minipage}}

\noindent \fbox{
\begin{minipage}[t]{\textwidth}
\textsf{\scriptsize BIBLIOGRAFIA COMPLEMENTAR:}
\begin{enumerate}
\def\labelenumi{\arabic{enumi}.}
\item
  SIPSER, Michael. Introdução. Teoria da Computação. Thomson Pioneira,
  2a edição, 2007.
\item
  MEDINA, M., FERTIG, C. Algoritmos e Programação: Teoria e Prática.
  Novatec, 2005.
\item
  HOPCROFT, J. E.; MOTWANI, R.; ULLMAN, J.D.: Introdução a Teoria de
  Autômatos, Linguagens e Computação. Rio de Janeiro: Campus, 2a edição,
  2002.
\end{enumerate}
\end{minipage}}
\end{scriptsize}

\newpage