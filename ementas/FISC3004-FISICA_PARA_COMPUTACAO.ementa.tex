\begin{figure*}
    \centering
    \includegraphics[width=2.57cm]{./images/brasao_da_republica.jpeg}
    
    \textsf{Ministério da Educação}\\
    \textsf{Universidade Federal do Agreste de Pernambuco}\\
    \textsf{Bacharelado em Ciência da Computação}
\end{figure*}

\vspace{0.5cm}

\begin{scriptsize}
\noindent \fbox{
\begin{minipage}[t]{\textwidth}
\textsf{\scriptsize COMPONENTE CURRICULAR:} \\
\textbf{FÍSICA PARA COMPUTAÇÃO} \\
\textsf{\scriptsize CÓDIGO:} \textbf{FISC3004}
\end{minipage}}

\noindent \fbox{
\begin{minipage}[t]{0.293\textwidth}
\textsf{\scriptsize PERÍODO A SER OFERTADO:} \\
\textbf{2}
\end{minipage}
%
\begin{minipage}[t]{0.7\textwidth}
\textsf{\scriptsize NÚCLEO DE FORMAÇÃO:} \\
\textbf{CICLO GERAL OU CICLO BÁSICO}
\end{minipage}}

\noindent \fbox{
\begin{minipage}[t]{0.493\textwidth}
\textsf{\scriptsize TIPO:} \textbf{OBRIGATÓRIO}
\end{minipage}
%
\begin{minipage}[t]{0.5\textwidth}
\textsf{\scriptsize CRÉDITOS:} \textbf{4}
\end{minipage}}

\noindent \fbox{
\begin{minipage}[t]{0.3\textwidth}
\textsf{\scriptsize CARGA HORÁRIA TOTAL:} 
\textbf{60}
\end{minipage}
%
\begin{minipage}[t]{0.19\textwidth}
\textsf{\scriptsize TEÓRICA:} 
\textbf{60}
\end{minipage}
%
\begin{minipage}[t]{0.19\textwidth}
\textsf{\scriptsize PRÁTICA:} 
\textbf{0}
\end{minipage}
%
\begin{minipage}[t]{0.30\textwidth}
\textsf{\scriptsize EAD-SEMIPRESENCIAL:} 
\textbf{0}
\end{minipage}}

\noindent \fbox{
\begin{minipage}[t]{\textwidth}
\textsf{\scriptsize PRÉ-REQUISITOS:}
MATM3031 CÁLCULO PARA COMPUTAÇÃO I
\end{minipage}}

\noindent \fbox{
\begin{minipage}[t]{\textwidth}
\textsf{\scriptsize CORREQUISITOS:} 
Não há.
\end{minipage}}

\noindent \fbox{
\begin{minipage}[t]{\textwidth}
\textsf{\scriptsize REQUISITO DE CARGA HORÁRIA:} 
Não há.
\end{minipage}}

\noindent \fbox{
\begin{minipage}[t]{\textwidth}
\textsf{\scriptsize EMENTA:} \\
Conceito de: (a) Carga elétrica, (b) Campo elétrico, (c) Potencial
elétrico, (d) corrente elétrica, (e) potência elétrica; Resistência
elétrica e lei de Ohm; associação de resistores: associação em série e
em paralelo, transformação estrela-triângulo; bateria elétrica;
circuitos resistivos e leis de Kirchhoff; Capacitor e circuitos RC;
Fontes do campo magnético, solenóide e imãs; Indutor, auto-indução,
indutância mútua, circuitos RL; Corrente alternada, circuitos RLC,
transformadores, motores e geradores elétricos; Espectro
eletromagnético, propagação de ondas eletromagnéticas, lasers; Metais,
isolantes e semicondutores; Diodo e circuitos com diodos; Transistor e
circuitos com transistor; Circuitos eletrônicos básicos.
\end{minipage}}

\noindent \fbox{
\begin{minipage}[t]{\textwidth}
\textsf{\scriptsize BIBLIOGRAFIA BÁSICA:}
\begin{enumerate}
\def\labelenumi{\arabic{enumi}.}
\item
  HALLIDAY, David, RESNICK, Robert e WALKER, Jearl, Fundamentos de
  Física Volume 3 Eletromagnetismo, Ed. LTC, Rio de Janeiro, 2007.
\item
  TIPLER, Paul A. e MOSCA, Gene, FÍSICA para Cientistas e Engenheiros
  Volume 2 Eletricidade e Magnetismo, Óptica, Ed. LTC, Rio de Janeiro,
  2009.
\item
  NUSSENZVEIG, H.Moysés, Curso de Física Básica 3 Eletromagnetismo, Ed.
  Edgard Blücher LTDA São Paulo, 1997.
\end{enumerate}
\end{minipage}}

\noindent \fbox{
\begin{minipage}[t]{\textwidth}
\textsf{\scriptsize BIBLIOGRAFIA COMPLEMENTAR:}
\begin{enumerate}
\def\labelenumi{\arabic{enumi}.}
\item
  FERRARO, Nicolau Gilberto; SOARES, Paulo Antônio de Toledo.
  Eletricidade, Física Moderna. 7 ed.~Reform. São Paulo: Atual, 2003.
\item
  ZEMANSKY, Sears e FREEDMAN, Young E. Física III Eletromagnetismo, Ed.
  Addisson Wesley 2009.
\item
  Física3: eletromagnetismo. 5. ed.~São Paulo: EDUSP, 2005.
\item
  MACHADO, Kleber Daum, Teoria do Eletromagnetismo Volume I, Ed. UEPG,
  Ponta Grossa, 2004.
\item
  ALONSO \& FINN, Física um Curso Universitário Volume II Campos e
  Ondas, Ed. Edgard Blücher LTDA São Paulo, 1972.
\item
  SERWAY, Raymond A. e JEWETT Jr, John W., Princípios de Física Volume 3
  Eletromagnetismo, Ed. Thomson São Paulo, 2006.
\end{enumerate}
\end{minipage}}
\end{scriptsize}

\newpage