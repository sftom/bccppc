\begin{figure*}
    \centering
    \includegraphics[width=2.57cm]{./images/brasao_da_republica.jpeg}
    
    \textsf{Ministério da Educação}\\
    \textsf{Universidade Federal do Agreste de Pernambuco}\\
    \textsf{Bacharelado em Ciência da Computação}
\end{figure*}

\vspace{0.5cm}

\begin{scriptsize}
\noindent \fbox{
\begin{minipage}[t]{\textwidth}
\textsf{\scriptsize COMPONENTE CURRICULAR:} \\
\textbf{ARQUITETURA DE COMPUTADORES} \\
\textsf{\scriptsize CÓDIGO:} \textbf{CCMP3010}
\end{minipage}}

\noindent \fbox{
\begin{minipage}[t]{0.293\textwidth}
\textsf{\scriptsize PERÍODO A SER OFERTADO:} \\
\textbf{4}
\end{minipage}
%
\begin{minipage}[t]{0.7\textwidth}
\textsf{\scriptsize NÚCLEO DE FORMAÇÃO:} \\
\textbf{CICLO GERAL OU CICLO BÁSICO}
\end{minipage}}

\noindent \fbox{
\begin{minipage}[t]{0.493\textwidth}
\textsf{\scriptsize TIPO:} \textbf{OBRIGATÓRIO}
\end{minipage}
%
\begin{minipage}[t]{0.5\textwidth}
\textsf{\scriptsize CRÉDITOS:} \textbf{4}
\end{minipage}}

\noindent \fbox{
\begin{minipage}[t]{0.3\textwidth}
\textsf{\scriptsize CARGA HORÁRIA TOTAL:} 
\textbf{60}
\end{minipage}
%
\begin{minipage}[t]{0.19\textwidth}
\textsf{\scriptsize TEÓRICA:} 
\textbf{45}
\end{minipage}
%
\begin{minipage}[t]{0.19\textwidth}
\textsf{\scriptsize PRÁTICA:} 
\textbf{15}
\end{minipage}
%
\begin{minipage}[t]{0.30\textwidth}
\textsf{\scriptsize EAD-SEMIPRESENCIAL:} 
\textbf{0}
\end{minipage}}

\noindent \fbox{
\begin{minipage}[t]{\textwidth}
\textsf{\scriptsize PRÉ-REQUISITOS:}
\begin{itemize}
\item
  CCMP3058 SISTEMAS DIGITAIS
\item
  FISC3004 FÍSICA PARA COMPUTAÇÃO
\item
  MATM3031 CÁLCULO PARA COMPUTAÇÃO I
\end{itemize}
\end{minipage}}

\noindent \fbox{
\begin{minipage}[t]{\textwidth}
\textsf{\scriptsize CORREQUISITOS:} 
Não há.
\end{minipage}}

\noindent \fbox{
\begin{minipage}[t]{\textwidth}
\textsf{\scriptsize REQUISITO DE CARGA HORÁRIA:} 
Não há.
\end{minipage}}

\noindent \fbox{
\begin{minipage}[t]{\textwidth}
\textsf{\scriptsize EMENTA:} \\
Organização de Computadores; Conjunto de Instruções, Mecanismos de
Interrupção e de Exceção; Barramento, Comunicações; Interfaces e
Periféricos, Hierarquia de Memória; Multiprocessadores;
Multicomputadores; Arquiteturas Paralelas.
\end{minipage}}

\noindent \fbox{
\begin{minipage}[t]{\textwidth}
\textsf{\scriptsize BIBLIOGRAFIA BÁSICA:}
\begin{enumerate}
\def\labelenumi{\arabic{enumi}.}
\item
  PATTERSON, D; HENNESSY, J. L. Organização e Projeto de Computadores.
  4ª Ed. Campus. 2014.
\item
  STALLINGS, W. Arquitetura e Organização de Computadores. 8ª Ed.
  Pearson. 2010.
\item
  MONTEIRO, M. A. Introdução à Organização de Computadores. 5ª Ed. LTC.
  2007.
\end{enumerate}
\end{minipage}}

\noindent \fbox{
\begin{minipage}[t]{\textwidth}
\textsf{\scriptsize BIBLIOGRAFIA COMPLEMENTAR:}
\begin{enumerate}
\def\labelenumi{\arabic{enumi}.}
\item
  HENNESSY, J. L; PATTERSON, D. Arquiteturas de Computadores -- Uma
  abordagem quantitativa. 5ª Ed. Campus. 2014.
\item
  TOCCI, Ronald J. Sistemas Digitais: Princípios e Aplicações. 10ª Ed.
  Pearson. São Paulo, 2007.
\item
  TANENBAUM, A. S. Organizacao Estruturada de Computadores. 5ª Ed.
  Pearson. 2009.
\item
  D'AMORE, R. VHDL: Descrição e Síntese de Circuitos Digitais. 1ª Ed.
  LTC. 2005.
\item
  TOCCI, R.J. Sistemas Digitais: Princípios e Aplicações. 10ª Ed.
  Pearson. São Paulo, 2007.
\end{enumerate}
\end{minipage}}
\end{scriptsize}

\newpage