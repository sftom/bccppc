\begin{figure*}
    \centering
    \includegraphics[width=2.57cm]{./images/brasao_da_republica.jpeg}
    
    \textsf{Ministério da Educação}\\
    \textsf{Universidade Federal do Agreste de Pernambuco}\\
    \textsf{Bacharelado em Ciência da Computação}
\end{figure*}

\vspace{2cm}

\begin{scriptsize}
\noindent \fbox{
\begin{minipage}[t]{\textwidth}
\textsf{\scriptsize COMPONENTE CURRICULAR:} \\
\textbf{MODELAGEM MATEMÁTICA} \\
\textsf{\scriptsize CÓDIGO:} \\
\textbf{CCMP3032}
\end{minipage}}

\noindent \fbox{
\begin{minipage}[t]{0.293\textwidth}
\textsf{\scriptsize PERÍODO A SER OFERTADO:} \\
\textbf{0}
\end{minipage}
%
\begin{minipage}[t]{0.7\textwidth}
\textsf{\scriptsize NÚCLEO DE FORMAÇÃO:} \\
\textbf{COMPONENTES OPTATIVOS ÁREA TEMÁTICA MATEMÁTICA E SIMULAÇÃO
COMPUTACIONAL}
\end{minipage}}

\noindent \fbox{
\begin{minipage}[t]{0.493\textwidth}
\textsf{\scriptsize TIPO:} \\
\textbf{OPTATIVO}
\end{minipage}
%
\begin{minipage}[t]{0.5\textwidth}
\textsf{\scriptsize CRÉDITOS:} \\
\textbf{4}
\end{minipage}}

\noindent \fbox{
\begin{minipage}[t]{0.3\textwidth}
\textsf{\scriptsize CARGA HORÁRIA TOTAL:} \\
\textbf{60}
\end{minipage}
%
\begin{minipage}[t]{0.19\textwidth}
\textsf{\scriptsize TEÓRICA:} \\
\textbf{60}
\end{minipage}
%
\begin{minipage}[t]{0.19\textwidth}
\textsf{\scriptsize PRÁTICA:} \\
\textbf{0}
\end{minipage}
%
\begin{minipage}[t]{0.30\textwidth}
\textsf{\scriptsize EAD-SEMIPRESENCIAL:} \\
\textbf{0}
\end{minipage}}

\noindent \fbox{
\begin{minipage}[t]{\textwidth}
\textsf{\scriptsize PRÉ-REQUISITOS:}
\begin{itemize}
\item
  MATM3031 CÁLCULO PARA COMPUTAÇÃO I
\item
  MATM3032 CÁLCULO PARA COMPUTAÇÃO II
\end{itemize}
\end{minipage}}

\noindent \fbox{
\begin{minipage}[t]{\textwidth}
\textsf{\scriptsize CORREQUISITOS:} 
Não há.
\end{minipage}}

\noindent \fbox{
\begin{minipage}[t]{\textwidth}
\textsf{\scriptsize REQUISITO DE CARGA HORÁRIA:} 
Não há.
\end{minipage}}

\noindent \fbox{
\begin{minipage}[t]{\textwidth}
\textsf{\scriptsize EMENTA:} \\
Princípios básicos (o que é um modelo, porque modelar, objetivos e
requisitos); Metodologia: etapas (identificação, formulação e solução),
modelos matemáticos (quantitativos e qualitativos), tipos de modelos
(determinísticos, fuzzy, estatístico, estocástico),modelos discretos e
contínuos,processos de modelagem; Noções de cálculo vetorial e
tensorial, significado físico dos operadores gradiente, divergente,
rotacional e laplaciano; Propriedades físicas; sistemas referências;
leis de conservação, equações constitutivas; Exemplos envolvendo todas
as etapas de modelagem (exceto a solução).
\end{minipage}}

\noindent \fbox{
\begin{minipage}[t]{\textwidth}
\textsf{\scriptsize BIBLIOGRAFIA BÁSICA:}
\begin{enumerate}
\def\labelenumi{\arabic{enumi}.}
\item
  C.L. Dym \& E.S. Ivey - Principles of Mathematical Modeling, Academic
  Press, 1980.
\item
  Karam F., J. e Almeida, R. C., Introdução à Modelagem Matemática,
  Notas impressas PosGraduação, LNCC, 2003.
\item
  T.L. Saaty \& J.M. Alexander, Thinking with Models - Mathematical
  Models in Physical, Biological and Social Sciences, Pergamon Press,
  1981.
\end{enumerate}
\end{minipage}}

\noindent \fbox{
\begin{minipage}[t]{\textwidth}
\textsf{\scriptsize BIBLIOGRAFIA COMPLEMENTAR:}
\begin{enumerate}
\def\labelenumi{\arabic{enumi}.}
\item
  R.B. Bird, W.E. Stewart \& E.N. Lightfoot, Transport Phenomena, John
  Wiley \& Sons, 1960.
\item
  Mathematical Modelling Techniques, Rutherford Aris, Dover, 1994.
\item
  Introduction to Continuum Mechanics, W. M. Lai, D. Rubin, E.
  Krempl,Pergamon Press, 1974.
\end{enumerate}
\end{minipage}}
\end{scriptsize}

\newpage