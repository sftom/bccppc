\begin{figure*}
    \centering
    \includegraphics[width=2.57cm]{./images/brasao_da_republica.jpeg}
    
    \textsf{Ministério da Educação}\\
    \textsf{Universidade Federal do Agreste de Pernambuco}\\
    \textsf{Bacharelado em Ciência da Computação}
\end{figure*}

\vspace{0.5cm}

\begin{scriptsize}
\noindent \fbox{
\begin{minipage}[t]{\textwidth}
\textsf{\scriptsize COMPONENTE CURRICULAR:} \\
\textbf{BANCO DE DADOS AVANÇADO} \\
\textsf{\scriptsize CÓDIGO:} \textbf{}
\end{minipage}}

\noindent \fbox{
\begin{minipage}[t]{0.293\textwidth}
\textsf{\scriptsize PERÍODO A SER OFERTADO:} \\
\textbf{0}
\end{minipage}
%
\begin{minipage}[t]{0.7\textwidth}
\textsf{\scriptsize NÚCLEO DE FORMAÇÃO:} \\
\textbf{COMPONENTES OPTATIVOS ÁREA TEMÁTICA BANCO DE DADOS}
\end{minipage}}

\noindent \fbox{
\begin{minipage}[t]{0.493\textwidth}
\textsf{\scriptsize TIPO:} \textbf{OPTATIVO}
\end{minipage}
%
\begin{minipage}[t]{0.5\textwidth}
\textsf{\scriptsize CRÉDITOS:} \textbf{4}
\end{minipage}}

\noindent \fbox{
\begin{minipage}[t]{0.3\textwidth}
\textsf{\scriptsize CARGA HORÁRIA TOTAL:} 
\textbf{60}
\end{minipage}
%
\begin{minipage}[t]{0.19\textwidth}
\textsf{\scriptsize TEÓRICA:} 
\textbf{30}
\end{minipage}
%
\begin{minipage}[t]{0.19\textwidth}
\textsf{\scriptsize PRÁTICA:} 
\textbf{30}
\end{minipage}
%
\begin{minipage}[t]{0.30\textwidth}
\textsf{\scriptsize EAD-SEMIPRESENCIAL:} 
\textbf{0}
\end{minipage}}

\noindent \fbox{
\begin{minipage}[t]{\textwidth}
\textsf{\scriptsize PRÉ-REQUISITOS:}
BANCO DE DADOS
\end{minipage}}

\noindent \fbox{
\begin{minipage}[t]{\textwidth}
\textsf{\scriptsize CORREQUISITOS:} 
Não há.
\end{minipage}}

\noindent \fbox{
\begin{minipage}[t]{\textwidth}
\textsf{\scriptsize REQUISITO DE CARGA HORÁRIA:} 
Não há.
\end{minipage}}

\noindent \fbox{
\begin{minipage}[t]{\textwidth}
\textsf{\scriptsize EMENTA:} \\
Conceitos avançados de Banco de Dados. SQL Avançado. Aspectos
Operacionais de SGBD (Controle de Concorrência, Restrições de
Integridade, Segurança e Recuperação após falhas). Modelo EER (ER
Estendido). Data Warehouse. Data Mining. BDs Móveis.BDs Multimídia. BDs
Geográficos. BDs Distribuídos. BD Orientado a Objetos. BD
Objeto-Relacional. BDs Ativos. BDs Temporais. BDs Biológicos. BDs P2P.
BDs Autônomos. Cloud Data Bases (Banco de Dados em Nuvens).
\end{minipage}}

\noindent \fbox{
\begin{minipage}[t]{\textwidth}
\textsf{\scriptsize BIBLIOGRAFIA BÁSICA:}
\begin{enumerate}
\def\labelenumi{\arabic{enumi}.}
\item
  SILBERSCHATZ, Abraham et al.~Sistemas de bancos de dados. 5. ed.~778
  p.~São Paulo : Makron Books, 2006.
\item
  ÖZSU and VALDURIEZ. Princípios de Sistemas de Bancos de dados
  Distribuídos, Editora Campus, 2001.
\item
  ELSMARI \& NAVATHE. Sistemas de Banco de Dados. 6. Ed. 744p. Pearson.
  2011.
\end{enumerate}
\end{minipage}}

\noindent \fbox{
\begin{minipage}[t]{\textwidth}
\textsf{\scriptsize BIBLIOGRAFIA COMPLEMENTAR:}
\begin{enumerate}
\def\labelenumi{\arabic{enumi}.}
\item
  REUSER, Carlos Alberto. Projeto de Banco de Dados. 6. Ed. 282p. Ed.
  Bookman 2009.
\item
  RAMAKRISHNAN, GEHRKE. Sistema de Gerenciamento de Banco de Dados. 3.
  Ed. 884 páginas, Mcgraw Hill 2008.
\item
  RIGAUX et. al.~Spatial Databases: with application to GIS, Morgan
  Kaufmann, 2002.
\end{enumerate}
\end{minipage}}
\end{scriptsize}

\newpage