\begin{figure*}
    \centering
    \includegraphics[width=2.57cm]{./images/brasao_da_republica.jpeg}
    
    \textsf{Ministério da Educação}\\
    \textsf{Universidade Federal do Agreste de Pernambuco}\\
    \textsf{Bacharelado em Ciência da Computação}
\end{figure*}

\vspace{0.5cm}

\begin{scriptsize}
\noindent \fbox{
\begin{minipage}[t]{\textwidth}
\textsf{\scriptsize COMPONENTE CURRICULAR:} \\
\textbf{PROBABILIDADE E ESTATÍSTICA} \\
\textsf{\scriptsize CÓDIGO:} \textbf{PRBE3006}
\end{minipage}}

\noindent \fbox{
\begin{minipage}[t]{0.293\textwidth}
\textsf{\scriptsize PERÍODO A SER OFERTADO:} \\
\textbf{3}
\end{minipage}
%
\begin{minipage}[t]{0.7\textwidth}
\textsf{\scriptsize NÚCLEO DE FORMAÇÃO:} \\
\textbf{CICLO GERAL OU CICLO BÁSICO}
\end{minipage}}

\noindent \fbox{
\begin{minipage}[t]{0.493\textwidth}
\textsf{\scriptsize TIPO:} \textbf{OBRIGATÓRIO}
\end{minipage}
%
\begin{minipage}[t]{0.5\textwidth}
\textsf{\scriptsize CRÉDITOS:} \textbf{4}
\end{minipage}}

\noindent \fbox{
\begin{minipage}[t]{0.3\textwidth}
\textsf{\scriptsize CARGA HORÁRIA TOTAL:} 
\textbf{60}
\end{minipage}
%
\begin{minipage}[t]{0.19\textwidth}
\textsf{\scriptsize TEÓRICA:} 
\textbf{60}
\end{minipage}
%
\begin{minipage}[t]{0.19\textwidth}
\textsf{\scriptsize PRÁTICA:} 
\textbf{0}
\end{minipage}
%
\begin{minipage}[t]{0.30\textwidth}
\textsf{\scriptsize EAD-SEMIPRESENCIAL:} 
\textbf{0}
\end{minipage}}

\noindent \fbox{
\begin{minipage}[t]{\textwidth}
\textsf{\scriptsize PRÉ-REQUISITOS:}
\begin{itemize}
\item
  MATM3031 CÁLCULO PARA COMPUTAÇÃO I
\item
  MATM3032 CÁLCULO PARA COMPUTAÇÃO II
\end{itemize}
\end{minipage}}

\noindent \fbox{
\begin{minipage}[t]{\textwidth}
\textsf{\scriptsize CORREQUISITOS:} 
Não há.
\end{minipage}}

\noindent \fbox{
\begin{minipage}[t]{\textwidth}
\textsf{\scriptsize REQUISITO DE CARGA HORÁRIA:} 
Não há.
\end{minipage}}

\noindent \fbox{
\begin{minipage}[t]{\textwidth}
\textsf{\scriptsize EMENTA:} \\
Análise combinatória. Planejamento de uma pesquisa. Análise exploratória
de dados. Probabilidade. Variáveis aleatórias discretas e contínuas.
Principais modelos teóricos. Estimação de parâmetros. Testes de
hipóteses.
\end{minipage}}

\noindent \fbox{
\begin{minipage}[t]{\textwidth}
\textsf{\scriptsize BIBLIOGRAFIA BÁSICA:}
\begin{enumerate}
\def\labelenumi{\arabic{enumi}.}
\item
  BUSSAB, W. O.; MORETTIN, P. A. -- Estatística Básica. 8.ed. Editora
  Saraiva, São Paulo, 2013.
\item
  MORETTIN, L. G. -- Estatística básica. 7. ed.~Makron Books, São Paulo,
  1999.
\item
  MONTGOMERY, D. C., RUNGER, G. C. -- Estatística aplicada e
  probabilidade para engenheiros. 5.ed. LTC, Rio de Janeiro, 2012
\end{enumerate}
\end{minipage}}

\noindent \fbox{
\begin{minipage}[t]{\textwidth}
\textsf{\scriptsize BIBLIOGRAFIA COMPLEMENTAR:}
\begin{enumerate}
\def\labelenumi{\arabic{enumi}.}
\item
  MEYER, P.L. - Probabilidade Aplicações à Estatística, 2.ed. LTC, Rio
  de Janeiro, 1983
\item
  TRIOLA, M.F. - Introdução à estatística. 10.ed. LTC, Rio de Janeiro,
  2008
\item
  BARBETTA, P.A., REIS, M.M., BORNIA, A.C. - Estatística para Cursos de
  Engenharia e Informática. 3.ed. Editora Atlas, São Paulo, 2010
\item
  WALPOLE, R.E., MYERS, R.H., MYERS, S.L.,YE, K.- Probabilidade e
  Estatística Para Engenharia e Ciências. 8.ed. Prentice Hall, São
  Paulo, 2009
\item
  LARSON, R.,FARBER, B.- Estatística Aplicada, 4.ed. Prentice Hall, São
  Paulo, 2010
\end{enumerate}
\end{minipage}}
\end{scriptsize}

\newpage