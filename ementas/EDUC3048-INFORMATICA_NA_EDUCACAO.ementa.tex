\begin{figure*}
    \centering
    \includegraphics[width=2.57cm]{./images/brasao_da_republica.jpeg}
    
    \textsf{Ministério da Educação}\\
    \textsf{Universidade Federal do Agreste de Pernambuco}\\
    \textsf{Bacharelado em Ciência da Computação}
\end{figure*}

\vspace{2cm}

\begin{scriptsize}
\noindent \fbox{
\begin{minipage}[t]{\textwidth}
\textsf{\scriptsize COMPONENTE CURRICULAR:} \\
\textbf{INFORMÁTICA NA EDUCAÇÃO} \\
\textsf{\scriptsize CÓDIGO:} \\
\textbf{EDUC3048}
\end{minipage}}

\noindent \fbox{
\begin{minipage}[t]{0.293\textwidth}
\textsf{\scriptsize PERÍODO A SER OFERTADO:} \\
\textbf{0}
\end{minipage}
%
\begin{minipage}[t]{0.7\textwidth}
\textsf{\scriptsize NÚCLEO DE FORMAÇÃO:} \\
\textbf{COMPONENTES OPTATIVOS ÁREA TEMÁTICA TECNOLOGIA EDUCACIONAL}
\end{minipage}}

\noindent \fbox{
\begin{minipage}[t]{0.493\textwidth}
\textsf{\scriptsize TIPO:} \\
\textbf{OPTATIVO}
\end{minipage}
%
\begin{minipage}[t]{0.5\textwidth}
\textsf{\scriptsize CRÉDITOS:} \\
\textbf{4}
\end{minipage}}

\noindent \fbox{
\begin{minipage}[t]{0.3\textwidth}
\textsf{\scriptsize CARGA HORÁRIA TOTAL:} \\
\textbf{60}
\end{minipage}
%
\begin{minipage}[t]{0.19\textwidth}
\textsf{\scriptsize TEÓRICA:} \\
\textbf{60}
\end{minipage}
%
\begin{minipage}[t]{0.19\textwidth}
\textsf{\scriptsize PRÁTICA:} \\
\textbf{0}
\end{minipage}
%
\begin{minipage}[t]{0.30\textwidth}
\textsf{\scriptsize EAD-SEMIPRESENCIAL:} \\
\textbf{0}
\end{minipage}}

\noindent \fbox{
\begin{minipage}[t]{\textwidth}
\textsf{\scriptsize PRÉ-REQUISITOS:}
CCMP3056 INTRODUÇÃO À COMPUTAÇÃO C
\end{minipage}}

\noindent \fbox{
\begin{minipage}[t]{\textwidth}
\textsf{\scriptsize CORREQUISITOS:} 
Não há.
\end{minipage}}

\noindent \fbox{
\begin{minipage}[t]{\textwidth}
\textsf{\scriptsize REQUISITO DE CARGA HORÁRIA:} 
Não há.
\end{minipage}}

\noindent \fbox{
\begin{minipage}[t]{\textwidth}
\textsf{\scriptsize EMENTA:} \\
Motivação e discussão crítica sobre o uso da informática na educação,
incluindo conhecimento sobre assuntos atuais; Histórico da informática
na educação; Ambientes educacionais baseados em computador; As
implicações pedagógicas e sociais do uso da informática na educação;
Informática na educação especial, na educação à distância e no
aprendizado cooperativo.
\end{minipage}}

\noindent \fbox{
\begin{minipage}[t]{\textwidth}
\textsf{\scriptsize BIBLIOGRAFIA BÁSICA:}
\begin{enumerate}
\def\labelenumi{\arabic{enumi}.}
\item
  COSCARELLI, Carla Viana (Org.). Novas tecnologias, novos textos, novas
  formas de pensar. Belo Horizonte: Autêntica, 2006.
\item
  FARIA, Elaine Turk. Educação presencial e virtual: espaços
  complementares essenciais na escola e na empresa. Porto Alegre:
  EDIPUCRS, 2006.
\item
  FRANCO, Sergio R.K. Informática na Educação: Estudos
  Interdisciplinares. Editora da UFRGS, 2004.
\end{enumerate}
\end{minipage}}

\noindent \fbox{
\begin{minipage}[t]{\textwidth}
\textsf{\scriptsize BIBLIOGRAFIA COMPLEMENTAR:}
\begin{enumerate}
\def\labelenumi{\arabic{enumi}.}
\item
  COX, Kenia. Informática na Educação Escolar, Autores Associados, 2003.
\item
  LEVY, Pierre. Cibercultura, Editora 34, 1999.
\item
  RBIE -- Revista Brasileira de Informática na Educação ISSN 1414-5685
\item
  RENOTE -- Revista Novas Tecnologias na Educação ISSN 1679-1916
\end{enumerate}
\end{minipage}}
\end{scriptsize}

\newpage