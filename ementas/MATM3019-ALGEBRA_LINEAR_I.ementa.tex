\begin{figure*}
    \centering
    \includegraphics[width=2.57cm]{./images/brasao_da_republica.jpeg}
    
    \textsf{Ministério da Educação}\\
    \textsf{Universidade Federal do Agreste de Pernambuco}\\
    \textsf{Bacharelado em Ciência da Computação}
\end{figure*}

\vspace{0.5cm}

\begin{scriptsize}
\noindent \fbox{
\begin{minipage}[t]{\textwidth}
\textsf{\scriptsize COMPONENTE CURRICULAR:} \\
\textbf{ÁLGEBRA LINEAR I} \\
\textsf{\scriptsize CÓDIGO:} \textbf{MATM3019}
\end{minipage}}

\noindent \fbox{
\begin{minipage}[t]{0.293\textwidth}
\textsf{\scriptsize PERÍODO A SER OFERTADO:} \\
\textbf{2}
\end{minipage}
%
\begin{minipage}[t]{0.7\textwidth}
\textsf{\scriptsize NÚCLEO DE FORMAÇÃO:} \\
\textbf{CICLO GERAL OU CICLO BÁSICO}
\end{minipage}}

\noindent \fbox{
\begin{minipage}[t]{0.493\textwidth}
\textsf{\scriptsize TIPO:} \textbf{OBRIGATÓRIO}
\end{minipage}
%
\begin{minipage}[t]{0.5\textwidth}
\textsf{\scriptsize CRÉDITOS:} \textbf{4}
\end{minipage}}

\noindent \fbox{
\begin{minipage}[t]{0.3\textwidth}
\textsf{\scriptsize CARGA HORÁRIA TOTAL:} 
\textbf{60}
\end{minipage}
%
\begin{minipage}[t]{0.19\textwidth}
\textsf{\scriptsize TEÓRICA:} 
\textbf{60}
\end{minipage}
%
\begin{minipage}[t]{0.19\textwidth}
\textsf{\scriptsize PRÁTICA:} 
\textbf{0}
\end{minipage}
%
\begin{minipage}[t]{0.30\textwidth}
\textsf{\scriptsize EAD-SEMIPRESENCIAL:} 
\textbf{0}
\end{minipage}}

\noindent \fbox{
\begin{minipage}[t]{\textwidth}
\textsf{\scriptsize PRÉ-REQUISITOS:}
MATM3021 GEOMETRIA ANALÍTICA A
\end{minipage}}

\noindent \fbox{
\begin{minipage}[t]{\textwidth}
\textsf{\scriptsize CORREQUISITOS:} 
Não há.
\end{minipage}}

\noindent \fbox{
\begin{minipage}[t]{\textwidth}
\textsf{\scriptsize REQUISITO DE CARGA HORÁRIA:} 
Não há.
\end{minipage}}

\noindent \fbox{
\begin{minipage}[t]{\textwidth}
\textsf{\scriptsize EMENTA:} \\
Vetores; Matrizes; Operações elementares e sistemas de equações; Espaços
vetorias (subespaços, dependência e independência linear, base e
dimensão, espaço linha, espaço coluna e posto, dimensão do conjunto
solução de um sistema linear); Determinantes; Transformações lineares
(núcleo e imagem, representação por matrizes, mudança de base, auto
valor, auto vetor e diagonalização).
\end{minipage}}

\noindent \fbox{
\begin{minipage}[t]{\textwidth}
\textsf{\scriptsize BIBLIOGRAFIA BÁSICA:}
\begin{enumerate}
\def\labelenumi{\arabic{enumi}.}
\item
  STEINBRUCH, Alfredo. WINTERLE, Paulo. Álgebra Linear. 2ed. São Paulo:
  Pearson Makron Books, 1987.
\item
  BOLDRINI, José Luiz et al.~Álgebra Linear. 3ed. São Paulo: Harper \&
  Row do Brasil, 1980.
\item
  HOWARD, Anton; RORRES, Chris. Álgebra Linear com Aplicações. 8ed.
  Porto Alegre: Bookman, 2001.
\end{enumerate}
\end{minipage}}

\noindent \fbox{
\begin{minipage}[t]{\textwidth}
\textsf{\scriptsize BIBLIOGRAFIA COMPLEMENTAR:}
\begin{enumerate}
\def\labelenumi{\arabic{enumi}.}
\item
  LANG, Serge. Álgebra Linear. Rio de Janeiro: Editora Ciência Moderna,
  2003.
\item
  COELHO, Flávio Ulhoa; LOURENÇO, Mary Lillian. Um Curso de Álgebra
  Linear. 2ed. São Paulo: Editora da Universidade de São Paulo, 2005.
\item
  POOLE, David. Álgebra Linear. São Paulo: Cengage Learning, 2004.
\item
  LAWSON, Terry. Álgebra Linear. São Paulo: Edgard Blücher, 1997.
\item
  KOLMAN, Bernard. Introdução à Álgebra Linear com Aplicações. Rio de
  Janeiro: LTC, 1999.
\end{enumerate}
\end{minipage}}
\end{scriptsize}

\newpage