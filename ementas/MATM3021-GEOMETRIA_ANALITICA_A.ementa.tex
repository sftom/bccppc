\begin{figure*}
    \centering
    \includegraphics[width=2.57cm]{./images/brasao_da_republica.jpeg}
    
    \textsf{Ministério da Educação}\\
    \textsf{Universidade Federal do Agreste de Pernambuco}\\
    \textsf{Bacharelado em Ciência da Computação}
\end{figure*}

\vspace{2cm}

\begin{scriptsize}
\noindent \fbox{
\begin{minipage}[t]{\textwidth}
\textsf{\scriptsize COMPONENTE CURRICULAR:} \\
\textbf{GEOMETRIA ANALÍTICA A} \\
\textsf{\scriptsize CÓDIGO:} \\
\textbf{MATM3021}
\end{minipage}}

\noindent \fbox{
\begin{minipage}[t]{0.293\textwidth}
\textsf{\scriptsize PERÍODO A SER OFERTADO:} \\
\textbf{1}
\end{minipage}
%
\begin{minipage}[t]{0.7\textwidth}
\textsf{\scriptsize NÚCLEO DE FORMAÇÃO:} \\
\textbf{CICLO GERAL OU CICLO BÁSICO}
\end{minipage}}

\noindent \fbox{
\begin{minipage}[t]{0.493\textwidth}
\textsf{\scriptsize TIPO:} \\
\textbf{OBRIGATÓRIO}
\end{minipage}
%
\begin{minipage}[t]{0.5\textwidth}
\textsf{\scriptsize CRÉDITOS:} \\
\textbf{4}
\end{minipage}}

\noindent \fbox{
\begin{minipage}[t]{0.3\textwidth}
\textsf{\scriptsize CARGA HORÁRIA TOTAL:} \\
\textbf{60}
\end{minipage}
%
\begin{minipage}[t]{0.19\textwidth}
\textsf{\scriptsize TEÓRICA:} \\
\textbf{60}
\end{minipage}
%
\begin{minipage}[t]{0.19\textwidth}
\textsf{\scriptsize PRÁTICA:} \\
\textbf{0}
\end{minipage}
%
\begin{minipage}[t]{0.30\textwidth}
\textsf{\scriptsize EAD-SEMIPRESENCIAL:} \\
\textbf{0}
\end{minipage}}

\noindent \fbox{
\begin{minipage}[t]{\textwidth}
\textsf{\scriptsize PRÉ-REQUISITOS:}
Não há.
\end{minipage}}

\noindent \fbox{
\begin{minipage}[t]{\textwidth}
\textsf{\scriptsize CORREQUISITOS:} 
Não há.
\end{minipage}}

\noindent \fbox{
\begin{minipage}[t]{\textwidth}
\textsf{\scriptsize REQUISITO DE CARGA HORÁRIA:} 
Não há.
\end{minipage}}

\noindent \fbox{
\begin{minipage}[t]{\textwidth}
\textsf{\scriptsize EMENTA:} \\
Vetores no \(R^2\); Produto interno no \(R^2\); Estudo da reta no
\(R^2\); Lugares geométricos no \(R^2\) (circunferência, elipse,
parábola e hipérbole); Vetores no \(R^3\); Produto interno no \(R^3\),
vetorial e misto; Aplicações: áreas e volumes; Equação da reta no
\(R^3\) e equação do plano; Equação da superfície esférica.
\end{minipage}}

\noindent \fbox{
\begin{minipage}[t]{\textwidth}
\textsf{\scriptsize BIBLIOGRAFIA BÁSICA:}
\begin{enumerate}
\def\labelenumi{\arabic{enumi}.}
\item
  STEINBRUSH, Alfredo; WINTERLE, Paulo. Geometria Analítica. 2ed. São
  Paulo: Pearson Makron Books, 1987.
\item
  REIS, Genésio Lima; SILVA, Valdir Vilmar. Geometria Analítica. 2ed.
  Rio de Janeiro: LTC, 2002.
\item
  CAMARGO, Ivan; BOULUS, Paulo. Geometria Analítica. 3ed. São Paulo:
  Prentice Hall, 2005.
\end{enumerate}
\end{minipage}}

\noindent \fbox{
\begin{minipage}[t]{\textwidth}
\textsf{\scriptsize BIBLIOGRAFIA COMPLEMENTAR:}
\begin{enumerate}
\def\labelenumi{\arabic{enumi}.}
\item
  LEHMANN, Charles H. Geometria Analítica. 9ed. São Paulo: Globo, 1998.
\item
  MACHADO, Antônio dos Santos. Álgebra Linear e Geometria Analítica.
  2ed. São Paulo: Atual, 1982.
\item
  LEITHOLD, Louis. O Cálculo com Geometria Analítica. Vol 1. 3ed. São
  Paulo: HARBRA, 1994.
\item
  SIMMONS, George F. Cálculo com Geometria Analítica. Vol 1. São Paulo:
  Pearson Makron Books, 1987.
\item
  MUNEM, Mustafa; FOULIS, David J. Cálculo. Vol 1. Rio de Janeiro: LTC,
  2008.
\end{enumerate}
\end{minipage}}
\end{scriptsize}

\newpage