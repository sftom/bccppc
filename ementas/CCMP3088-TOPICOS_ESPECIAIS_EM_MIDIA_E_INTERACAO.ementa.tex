\begin{figure*}
    \centering
    \includegraphics[width=2.57cm]{./images/brasao_da_republica.jpeg}
    
    \textsf{Ministério da Educação}\\
    \textsf{Universidade Federal do Agreste de Pernambuco}\\
    \textsf{Bacharelado em Ciência da Computação}
\end{figure*}

\vspace{2cm}

\begin{scriptsize}
\noindent \fbox{
\begin{minipage}[t]{\textwidth}
\textsf{\scriptsize COMPONENTE CURRICULAR:} \\
\textbf{TÓPICOS ESPECIAIS EM MÍDIA E INTERAÇÃO} \\
\textsf{\scriptsize CÓDIGO:} \\
\textbf{CCMP3088}
\end{minipage}}

\noindent \fbox{
\begin{minipage}[t]{0.293\textwidth}
\textsf{\scriptsize PERÍODO A SER OFERTADO:} \\
\textbf{0}
\end{minipage}
%
\begin{minipage}[t]{0.7\textwidth}
\textsf{\scriptsize NÚCLEO DE FORMAÇÃO:} \\
\textbf{COMPONENTES OPTATIVOS ÁREA TEMÁTICA MÍDIA E INTERAÇÃO}
\end{minipage}}

\noindent \fbox{
\begin{minipage}[t]{0.493\textwidth}
\textsf{\scriptsize TIPO:} \\
\textbf{OPTATIVO}
\end{minipage}
%
\begin{minipage}[t]{0.5\textwidth}
\textsf{\scriptsize CRÉDITOS:} \\
\textbf{4}
\end{minipage}}

\noindent \fbox{
\begin{minipage}[t]{0.3\textwidth}
\textsf{\scriptsize CARGA HORÁRIA TOTAL:} \\
\textbf{60}
\end{minipage}
%
\begin{minipage}[t]{0.19\textwidth}
\textsf{\scriptsize TEÓRICA:} \\
\textbf{60}
\end{minipage}
%
\begin{minipage}[t]{0.19\textwidth}
\textsf{\scriptsize PRÁTICA:} \\
\textbf{0}
\end{minipage}
%
\begin{minipage}[t]{0.30\textwidth}
\textsf{\scriptsize EAD-SEMIPRESENCIAL:} \\
\textbf{0}
\end{minipage}}

\noindent \fbox{
\begin{minipage}[t]{\textwidth}
\textsf{\scriptsize PRÉ-REQUISITOS:}
Não há.
\end{minipage}}

\noindent \fbox{
\begin{minipage}[t]{\textwidth}
\textsf{\scriptsize CORREQUISITOS:} 
Não há.
\end{minipage}}

\noindent \fbox{
\begin{minipage}[t]{\textwidth}
\textsf{\scriptsize REQUISITO DE CARGA HORÁRIA:} 
Não há.
\end{minipage}}

\noindent \fbox{
\begin{minipage}[t]{\textwidth}
\textsf{\scriptsize EMENTA:} \\
Estudo de técnicas avançadas em de interação e representação das
diversas mídias existentes, permitindo ao aluno conhecer o estado da
arte nesta área de pesquisa.
\end{minipage}}

\noindent \fbox{
\begin{minipage}[t]{\textwidth}
\textsf{\scriptsize BIBLIOGRAFIA BÁSICA:}
\begin{enumerate}
\def\labelenumi{\arabic{enumi}.}
\item
  CASTELS, M. A sociedade em rede: a era da informação, economia,
  sociedade e cultura. Vol. 1. São Paulo: Paz e Terra, 1999.
\item
  KUROSE, J.F \& ROSS, K.W. Redes de computadores e a internet: uma
  abordagem top-dowm. São Paulo: Addison Wesley, 2010.
\item
  PRIMO, A. Interação Mediada por Computador: comunicação, cibercultura,
  cognição. Porto Alegre: Sulina, 2008.
\end{enumerate}
\end{minipage}}

\noindent \fbox{
\begin{minipage}[t]{\textwidth}
\textsf{\scriptsize BIBLIOGRAFIA COMPLEMENTAR:}
\begin{enumerate}
\def\labelenumi{\arabic{enumi}.}
\item
  ROGERS, Y;SHARP, H \& PREECE, J. Design de interação: além da
  interação humano-computador. Porto Alegre: Bookman, 2013.
\item
  SANTAELLA, L. Culturas e artes do pós-humano: da cultura das mídias à
  cibercultura. São Paulo: Paulus, 2003.
\item
  SANTAELLA, L. A ecologia pluralista da comunciação: conectividade,
  mobilidade, ubiquidade. São Paulo: Paulus, 2010.
\item
  THOMPSON, J.B. A mídia e a modernidade: uma teoria social da mídia.
  Petrópolis, RJ: Vozes, 2009.
\item
  LEVY, P. A inteligência coletiva. Por uma antropologia do ciberespaço.
  Trad. Paulo Rouanet. São Paulo: Loyola, 1998.
\item
  LÉVY, P. As Tecnologias da Inteligência -- Tradução: Carlos Irineuda
  Costa -- Rio de Janeiro, 1998.
\item
  HENRY, J. Cultura da convergência : a colisão entre os velhos e novos
  meios de comunicação.São Paulo : Aleph, 2009.
\end{enumerate}
\end{minipage}}
\end{scriptsize}

\newpage