\begin{figure*}
    \centering
    \includegraphics[width=2.57cm]{./images/brasao_da_republica.jpeg}
    
    \textsf{Ministério da Educação}\\
    \textsf{Universidade Federal do Agreste de Pernambuco}\\
    \textsf{Bacharelado em Ciência da Computação}
\end{figure*}

\vspace{2cm}

\begin{scriptsize}
\noindent \fbox{
\begin{minipage}[t]{\textwidth}
\textsf{\scriptsize COMPONENTE CURRICULAR:} \\
\textbf{CÁLCULO PARA COMPUTAÇÃO I} \\
\textsf{\scriptsize CÓDIGO:} \\
\textbf{MATM3031}
\end{minipage}}

\noindent \fbox{
\begin{minipage}[t]{0.293\textwidth}
\textsf{\scriptsize PERÍODO A SER OFERTADO:} \\
\textbf{1}
\end{minipage}
%
\begin{minipage}[t]{0.7\textwidth}
\textsf{\scriptsize NÚCLEO DE FORMAÇÃO:} \\
\textbf{CICLO GERAL OU CICLO BÁSICO}
\end{minipage}}

\noindent \fbox{
\begin{minipage}[t]{0.493\textwidth}
\textsf{\scriptsize TIPO:} \\
\textbf{OBRIGATÓRIO}
\end{minipage}
%
\begin{minipage}[t]{0.5\textwidth}
\textsf{\scriptsize CRÉDITOS:} \\
\textbf{4}
\end{minipage}}

\noindent \fbox{
\begin{minipage}[t]{0.3\textwidth}
\textsf{\scriptsize CARGA HORÁRIA TOTAL:} \\
\textbf{60}
\end{minipage}
%
\begin{minipage}[t]{0.19\textwidth}
\textsf{\scriptsize TEÓRICA:} \\
\textbf{60}
\end{minipage}
%
\begin{minipage}[t]{0.19\textwidth}
\textsf{\scriptsize PRÁTICA:} \\
\textbf{0}
\end{minipage}
%
\begin{minipage}[t]{0.30\textwidth}
\textsf{\scriptsize EAD-SEMIPRESENCIAL:} \\
\textbf{0}
\end{minipage}}

\noindent \fbox{
\begin{minipage}[t]{\textwidth}
\textsf{\scriptsize PRÉ-REQUISITOS:}
Não há.
\end{minipage}}

\noindent \fbox{
\begin{minipage}[t]{\textwidth}
\textsf{\scriptsize CORREQUISITOS:} 
Não há.
\end{minipage}}

\noindent \fbox{
\begin{minipage}[t]{\textwidth}
\textsf{\scriptsize REQUISITO DE CARGA HORÁRIA:} 
Não há.
\end{minipage}}

\noindent \fbox{
\begin{minipage}[t]{\textwidth}
\textsf{\scriptsize EMENTA:} \\
Conjuntos numéricos. Funções elementares: linear, afim, quadrática,
modular. Funções diretas e inversas. Funções exponenciais e
logarítmicas. Introdução à trigonometria. Funções trigonométricas.
Limite e continuidade. Derivadas e aplicações.
\end{minipage}}

\noindent \fbox{
\begin{minipage}[t]{\textwidth}
\textsf{\scriptsize BIBLIOGRAFIA BÁSICA:}
\begin{enumerate}
\def\labelenumi{\arabic{enumi}.}
\item
  STEWART, James. Cálculo V.1. 2ed. São Paulo: Cengage Learning, 2010.
\item
  LEITHOLD, Louis. O Cálculo com Geometria Analítica Volume 1. 3ed. São
  Paulo: Harbra, 1994.
\item
  SIMMONS, George F. Cálculo com Geometria Analítica V. 1. São Paulo:
  Pearson Makron Books, 1987.
\end{enumerate}
\end{minipage}}

\noindent \fbox{
\begin{minipage}[t]{\textwidth}
\textsf{\scriptsize BIBLIOGRAFIA COMPLEMENTAR:}
\begin{enumerate}
\def\labelenumi{\arabic{enumi}.}
\item
  MUNEM, Foulis. Cálculo V.1. Rio de Janeiro: LTC, 1982.
\item
  GUIDORIZZI, Hamilton Luiz. Um Curso de Cálculo V.1. 5ed. Rio de
  Janeiro: LTC, 2008.
\item
  ÁVILA, Geraldo. Cálculo das Funções de uma Variável V.1. 7ed. Rio de
  Janeiro: LTC, 2003.
\item
  ANTON, Howard. Cálculo um Novo Horizonte V.1. 6ed. Porto alegre:
  Bookman, 2000.
\item
  FINNEY, Ross L. Cálculo de George B. Thomas Jr.~V.1. 10ed. São Paulo:
  Addison Wesley, 2002.
\end{enumerate}
\end{minipage}}
\end{scriptsize}

\newpage