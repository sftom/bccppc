\begin{figure*}
    \centering
    \includegraphics[width=2.57cm]{./images/brasao_da_republica.jpeg}
    
    \textsf{Ministério da Educação}\\
    \textsf{Universidade Federal do Agreste de Pernambuco}\\
    \textsf{Bacharelado em Ciência da Computação}
\end{figure*}

\vspace{0.5cm}

\begin{scriptsize}
\noindent \fbox{
\begin{minipage}[t]{\textwidth}
\textsf{\scriptsize COMPONENTE CURRICULAR:} \\
\textbf{GESTÃO DE PROCESSOS DE NEGÓCIO} \\
\textsf{\scriptsize CÓDIGO:} \textbf{UAG00024}
\end{minipage}}

\noindent \fbox{
\begin{minipage}[t]{0.293\textwidth}
\textsf{\scriptsize PERÍODO A SER OFERTADO:} \\
\textbf{0}
\end{minipage}
%
\begin{minipage}[t]{0.7\textwidth}
\textsf{\scriptsize NÚCLEO DE FORMAÇÃO:} \\
\textbf{COMPONENTES OPTATIVOS ÁREA TEMÁTICA TECNOLOGIAS DA INFORMAÇÃO}
\end{minipage}}

\noindent \fbox{
\begin{minipage}[t]{0.493\textwidth}
\textsf{\scriptsize TIPO:} \textbf{OPTATIVO}
\end{minipage}
%
\begin{minipage}[t]{0.5\textwidth}
\textsf{\scriptsize CRÉDITOS:} \textbf{4}
\end{minipage}}

\noindent \fbox{
\begin{minipage}[t]{0.3\textwidth}
\textsf{\scriptsize CARGA HORÁRIA TOTAL:} 
\textbf{60}
\end{minipage}
%
\begin{minipage}[t]{0.19\textwidth}
\textsf{\scriptsize TEÓRICA:} 
\textbf{60}
\end{minipage}
%
\begin{minipage}[t]{0.19\textwidth}
\textsf{\scriptsize PRÁTICA:} 
\textbf{0}
\end{minipage}
%
\begin{minipage}[t]{0.30\textwidth}
\textsf{\scriptsize EAD-SEMIPRESENCIAL:} 
\textbf{0}
\end{minipage}}

\noindent \fbox{
\begin{minipage}[t]{\textwidth}
\textsf{\scriptsize PRÉ-REQUISITOS:}
CCMP3067 SISTEMAS DE INFORMAÇÃO E TECNOLOGIAS
\end{minipage}}

\noindent \fbox{
\begin{minipage}[t]{\textwidth}
\textsf{\scriptsize CORREQUISITOS:} 
Não há.
\end{minipage}}

\noindent \fbox{
\begin{minipage}[t]{\textwidth}
\textsf{\scriptsize REQUISITO DE CARGA HORÁRIA:} 
Não há.
\end{minipage}}

\noindent \fbox{
\begin{minipage}[t]{\textwidth}
\textsf{\scriptsize EMENTA:} \\
Contextualizando o gerenciamento de processos. Conceitos de processos.
Engenharia de Processos de negócios: Desenho, Ferramentas, Metodologias,
Suporte de TI para Engenharia de Processos. Sistemas de Informação e os
processos organizacionais. a prática da modelagem de processos. BPM e
BPMS; Conceitos básicos: ciclo de vida de BPM; BPMS e serviços Web;
Modelagem de processos: BPMN.
\end{minipage}}

\noindent \fbox{
\begin{minipage}[t]{\textwidth}
\textsf{\scriptsize BIBLIOGRAFIA BÁSICA:}
\begin{enumerate}
\def\labelenumi{\arabic{enumi}.}
\item
  BALDAM R. et al.~Gerenciamento de Processos de Negócio. São Paulo:
  Érica, 2014.
\item
  DAVENPORT, H. T. Reengenharia de processos: como inovar na empresa
  através de tecnologia de informação. Rio de Janeiro, Campus, 1994.
\item
  PAIM, R.,et aL. Gestão de Processos: Pensar, Agir e Aprender, Porto
  Alegre: Bookman, 2009.
\end{enumerate}
\end{minipage}}

\noindent \fbox{
\begin{minipage}[t]{\textwidth}
\textsf{\scriptsize BIBLIOGRAFIA COMPLEMENTAR:}
\begin{enumerate}
\def\labelenumi{\arabic{enumi}.}
\item
  VALLE, R.; OLIVEIRA, B. S. Análise e Modelagem de Processos de Negócio
  -- Foco na notação BPMN. São Paulo: Atlas, 2009.
\item
  CRUZ, Tadeu. Manual Para Gerenciamento de Processos de Negócio -
  Metodologia Domp. Atlas. 2015.
\item
  CAMPOS, Andre L.N. Modelagem de Processos com BPMN. BRASPORT. 2014.
\item
  PRADELLA, Simone. Gestão de Processos. Atlas. 2012.
\end{enumerate}
\end{minipage}}
\end{scriptsize}

\newpage