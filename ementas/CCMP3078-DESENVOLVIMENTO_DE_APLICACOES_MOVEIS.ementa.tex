\begin{figure*}
    \centering
    \includegraphics[width=2.57cm]{./images/brasao_da_republica.jpeg}
    
    \textsf{Ministério da Educação}\\
    \textsf{Universidade Federal do Agreste de Pernambuco}\\
    \textsf{Bacharelado em Ciência da Computação}
\end{figure*}

\vspace{2cm}

\begin{scriptsize}
\noindent \fbox{
\begin{minipage}[t]{\textwidth}
\textsf{\scriptsize COMPONENTE CURRICULAR:} \\
\textbf{DESENVOLVIMENTO DE APLICAÇÕES MÓVEIS} \\
\textsf{\scriptsize CÓDIGO:} \\
\textbf{CCMP3078}
\end{minipage}}

\noindent \fbox{
\begin{minipage}[t]{0.293\textwidth}
\textsf{\scriptsize PERÍODO A SER OFERTADO:} \\
\textbf{0}
\end{minipage}
%
\begin{minipage}[t]{0.7\textwidth}
\textsf{\scriptsize NÚCLEO DE FORMAÇÃO:} \\
\textbf{COMPONENTES OPTATIVOS ÁREA TEMÁTICA ENGENHARIA DE SOFTWARE}
\end{minipage}}

\noindent \fbox{
\begin{minipage}[t]{0.493\textwidth}
\textsf{\scriptsize TIPO:} \\
\textbf{OPTATIVO}
\end{minipage}
%
\begin{minipage}[t]{0.5\textwidth}
\textsf{\scriptsize CRÉDITOS:} \\
\textbf{4}
\end{minipage}}

\noindent \fbox{
\begin{minipage}[t]{0.3\textwidth}
\textsf{\scriptsize CARGA HORÁRIA TOTAL:} \\
\textbf{60}
\end{minipage}
%
\begin{minipage}[t]{0.19\textwidth}
\textsf{\scriptsize TEÓRICA:} \\
\textbf{60}
\end{minipage}
%
\begin{minipage}[t]{0.19\textwidth}
\textsf{\scriptsize PRÁTICA:} \\
\textbf{0}
\end{minipage}
%
\begin{minipage}[t]{0.30\textwidth}
\textsf{\scriptsize EAD-SEMIPRESENCIAL:} \\
\textbf{0}
\end{minipage}}

\noindent \fbox{
\begin{minipage}[t]{\textwidth}
\textsf{\scriptsize PRÉ-REQUISITOS:}
Não há.
\end{minipage}}

\noindent \fbox{
\begin{minipage}[t]{\textwidth}
\textsf{\scriptsize CORREQUISITOS:} 
Não há.
\end{minipage}}

\noindent \fbox{
\begin{minipage}[t]{\textwidth}
\textsf{\scriptsize REQUISITO DE CARGA HORÁRIA:} 
Não há.
\end{minipage}}

\noindent \fbox{
\begin{minipage}[t]{\textwidth}
\textsf{\scriptsize EMENTA:} \\
Introdução ao desenvolvimento de software para dispositivos móveis.
Evolução dos dispositivos móveis. Tecnologias Existentes. Conceitos
Fundamentais. Principais Componentes de Tela, Layouts, Persistência,
Serviços em Background, Câmera e Arquivos, Mapas e GPS.
\end{minipage}}

\noindent \fbox{
\begin{minipage}[t]{\textwidth}
\textsf{\scriptsize BIBLIOGRAFIA BÁSICA:}
\begin{enumerate}
\def\labelenumi{\arabic{enumi}.}
\item
  Lecheta, Ricardo R. Google Android: aprenda a criar aplicações para
  dispositivos móveis com o Android SDK. Ed. Novatec. 2ª ed.~São Paulo,
  2010.
\item
  Marinacci, J. Construindo aplicativos móveis com Java. Novatec, 2012.
\item
  Lamarche, Jeff., Mark, Dave. Dominando o Desenvolvimento no Iphone.
  Explorando o Sdk do Iphone. Editora Alta Books, 2009.
\end{enumerate}
\end{minipage}}

\noindent \fbox{
\begin{minipage}[t]{\textwidth}
\textsf{\scriptsize BIBLIOGRAFIA COMPLEMENTAR:}
\begin{enumerate}
\def\labelenumi{\arabic{enumi}.}
\item
  Android Cookbook: Problems and Solutions for Android Developers.
  O'Reilly Media. Ian F. Darwin. 2017.
\item
  iOS Programming: The Big Nerd Ranch Guide. Big Nerd Ranch. Christian
  Keur e Aaron Hillegass. 2017.
\item
  Allen, Sarah., Graupera, Vidal., Lundrigan, Lee., Desenvolvimento
  Profissional Multiplataforma para smartphone: Iphone, Android, Windows
  Mobile e Blackberry; Alta Books; 2012.
\item
  Neil, T. Mobile Design Pattern Gallery: UI Patterns for Mobile
  Applications, O'Reilly, 2012.
\item
  Hoober, S. Berkman, E. Designing Mobile Interfaces, O'Reilly, 2011.
\end{enumerate}
\end{minipage}}
\end{scriptsize}

\newpage