\begin{figure*}
    \centering
    \includegraphics[width=2.57cm]{./images/brasao_da_republica.jpeg}
    
    \textsf{Ministério da Educação}\\
    \textsf{Universidade Federal do Agreste de Pernambuco}\\
    \textsf{Bacharelado em Ciência da Computação}
\end{figure*}

\vspace{2cm}

\begin{scriptsize}
\noindent \fbox{
\begin{minipage}[t]{\textwidth}
\textsf{\scriptsize COMPONENTE CURRICULAR:} \\
\textbf{LÍNGUA BRASILEIRA DE SINAIS LIBRAS L} \\
\textsf{\scriptsize CÓDIGO:} \\
\textbf{EDUC3090}
\end{minipage}}

\noindent \fbox{
\begin{minipage}[t]{0.293\textwidth}
\textsf{\scriptsize PERÍODO A SER OFERTADO:} \\
\textbf{0}
\end{minipage}
%
\begin{minipage}[t]{0.7\textwidth}
\textsf{\scriptsize NÚCLEO DE FORMAÇÃO:} \\
\textbf{COMPONENTES OPTATIVOS ÁREA TEMÁTICA TECNOLOGIA EDUCACIONAL}
\end{minipage}}

\noindent \fbox{
\begin{minipage}[t]{0.493\textwidth}
\textsf{\scriptsize TIPO:} \\
\textbf{OPTATIVO}
\end{minipage}
%
\begin{minipage}[t]{0.5\textwidth}
\textsf{\scriptsize CRÉDITOS:} \\
\textbf{3}
\end{minipage}}

\noindent \fbox{
\begin{minipage}[t]{0.3\textwidth}
\textsf{\scriptsize CARGA HORÁRIA TOTAL:} \\
\textbf{45}
\end{minipage}
%
\begin{minipage}[t]{0.19\textwidth}
\textsf{\scriptsize TEÓRICA:} \\
\textbf{45}
\end{minipage}
%
\begin{minipage}[t]{0.19\textwidth}
\textsf{\scriptsize PRÁTICA:} \\
\textbf{0}
\end{minipage}
%
\begin{minipage}[t]{0.30\textwidth}
\textsf{\scriptsize EAD-SEMIPRESENCIAL:} \\
\textbf{0}
\end{minipage}}

\noindent \fbox{
\begin{minipage}[t]{\textwidth}
\textsf{\scriptsize PRÉ-REQUISITOS:}
Não há.
\end{minipage}}

\noindent \fbox{
\begin{minipage}[t]{\textwidth}
\textsf{\scriptsize CORREQUISITOS:} 
Não há.
\end{minipage}}

\noindent \fbox{
\begin{minipage}[t]{\textwidth}
\textsf{\scriptsize REQUISITO DE CARGA HORÁRIA:} 
Não há.
\end{minipage}}

\noindent \fbox{
\begin{minipage}[t]{\textwidth}
\textsf{\scriptsize EMENTA:} \\
Fundamentos gramaticais da Língua Brasileira de Sinais - Libras. Relação
entre Libras e cultura das comunidades surdas. Ensino básico da Libras.
Legislação e políticas de inclusão
\end{minipage}}

\noindent \fbox{
\begin{minipage}[t]{\textwidth}
\textsf{\scriptsize BIBLIOGRAFIA BÁSICA:}
\begin{enumerate}
\def\labelenumi{\arabic{enumi}.}
\item
  GESSER, Audrei. LIBRAS: que língua é essa? São Paulo, Editora
  Parábola: 2009.
\item
  GESSER, Audrei. O ouvinte e a surdez: sobre ensinar e aprender a
  LIBRAS. São Paulo, Editora Parábola, 2012.
\item
  FELIPE, T.A. Libras em contexto: curso básico, livro do estudante
  cursista. Brasília: Programa Nacional de Apoio à Educação dos Surdos,
  MEC, SEESP, 2001. 164p.
\item
  LACERDA, Cristina Broglia Feitosa de; SANTOS, Lara Ferreira dos.Tenho
  um aluno surdo, e agora? São Carlos, Edufscar, 2014.
\end{enumerate}
\end{minipage}}

\noindent \fbox{
\begin{minipage}[t]{\textwidth}
\textsf{\scriptsize BIBLIOGRAFIA COMPLEMENTAR:}
\begin{enumerate}
\def\labelenumi{\arabic{enumi}.}
\item
  BOCK, Ana Mercês Bahia, AGUIAR, WANDA MARIA JUNQUEIRA DE. A dimensão
  subjetiva do processo educacional: uma leitura Sócio Histórica. São
  Paulo: Editora Cortez, 2016.
\item
  BERNARDINO, Elidéia Lúcia. Absurdo ou lógica? a produção lingüística
  do surdo/ Elidéia Lúcia Bernardino. Belo Horizonte, Editora:
  Profetizando Vidas, 2000.
\item
  BRASIL. Secretaria de Educação Especial. A educação dos surdos/
  organizado por Giuseppe Rinaldi et al.~Brasília: MEC/SEESP,1997.
\end{enumerate}
\end{minipage}}
\end{scriptsize}

\newpage