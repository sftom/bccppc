\begin{figure*}
    \centering
    \includegraphics[width=2.57cm]{./images/brasao_da_republica.jpeg}
    
    \textsf{Ministério da Educação}\\
    \textsf{Universidade Federal do Agreste de Pernambuco}\\
    \textsf{Bacharelado em Ciência da Computação}
\end{figure*}

\vspace{2cm}

\begin{scriptsize}
\noindent \fbox{
\begin{minipage}[t]{\textwidth}
\textsf{\scriptsize COMPONENTE CURRICULAR:} \\
\textbf{INTEGRAÇÃO DE DADOS E DATA WAREHOUSE} \\
\textsf{\scriptsize CÓDIGO:} \\
\textbf{CCMP3091}
\end{minipage}}

\noindent \fbox{
\begin{minipage}[t]{0.293\textwidth}
\textsf{\scriptsize PERÍODO A SER OFERTADO:} \\
\textbf{0}
\end{minipage}
%
\begin{minipage}[t]{0.7\textwidth}
\textsf{\scriptsize NÚCLEO DE FORMAÇÃO:} \\
\textbf{COMPONENTES OPTATIVOS ÁREA TEMÁTICA BANCO DE DADOS}
\end{minipage}}

\noindent \fbox{
\begin{minipage}[t]{0.493\textwidth}
\textsf{\scriptsize TIPO:} \\
\textbf{OPTATIVO}
\end{minipage}
%
\begin{minipage}[t]{0.5\textwidth}
\textsf{\scriptsize CRÉDITOS:} \\
\textbf{4}
\end{minipage}}

\noindent \fbox{
\begin{minipage}[t]{0.3\textwidth}
\textsf{\scriptsize CARGA HORÁRIA TOTAL:} \\
\textbf{60}
\end{minipage}
%
\begin{minipage}[t]{0.19\textwidth}
\textsf{\scriptsize TEÓRICA:} \\
\textbf{30}
\end{minipage}
%
\begin{minipage}[t]{0.19\textwidth}
\textsf{\scriptsize PRÁTICA:} \\
\textbf{30}
\end{minipage}
%
\begin{minipage}[t]{0.30\textwidth}
\textsf{\scriptsize EAD-SEMIPRESENCIAL:} \\
\textbf{0}
\end{minipage}}

\noindent \fbox{
\begin{minipage}[t]{\textwidth}
\textsf{\scriptsize PRÉ-REQUISITOS:}
CCMP3066 BANCO DE DADOS I
\end{minipage}}

\noindent \fbox{
\begin{minipage}[t]{\textwidth}
\textsf{\scriptsize CORREQUISITOS:} 
Não há.
\end{minipage}}

\noindent \fbox{
\begin{minipage}[t]{\textwidth}
\textsf{\scriptsize REQUISITO DE CARGA HORÁRIA:} 
Não há.
\end{minipage}}

\noindent \fbox{
\begin{minipage}[t]{\textwidth}
\textsf{\scriptsize EMENTA:} \\
Definir o processo de construção de um DataWarehouse (DW). Explicar as
diferentes fases de construção de um DW. Listar os principais fatores
que definem um projeto com sucesso. Analisar e transformar exigências
empresariais em um modelo de negócios (conceitual). Utilizar diagramas
de relacionamentos de entidades para transformar o modelo de negócios em
um modelo dimensional (lógico). Transformar o modelo dimensional em um
projeto de dados físico. Apresentar as principais estruturas que
cooperam no desempenho e criação de uma base DW.
\end{minipage}}

\noindent \fbox{
\begin{minipage}[t]{\textwidth}
\textsf{\scriptsize BIBLIOGRAFIA BÁSICA:}
\begin{enumerate}
\def\labelenumi{\arabic{enumi}.}
\item
  KIMBALL, R.; ROSS, M. The Data Warehouse Toolkit: The Definitive Guide
  to Dimensional Modeling. Wiley, 2013.
\item
  KIMBALL, R. The Data Warehouse Toolkit: guia completo para modelagem
  dimensional. Campus, 2002.
\item
  MACHADO, F. N. Tecnologia e projeto de Data Warehouse: uma visão
  multidimensional. Erica, 2008.
\end{enumerate}
\end{minipage}}

\noindent \fbox{
\begin{minipage}[t]{\textwidth}
\textsf{\scriptsize BIBLIOGRAFIA COMPLEMENTAR:}
\begin{enumerate}
\def\labelenumi{\arabic{enumi}.}
\item
  JUKIC, N.; VRBSKY, S.; NESTOROV, S. Database Systems: Introduction to
  Databases and Data Warehouses. Prospect Press, 2016.
\item
  KIMBALL, R.; CASERTA, J. The Data Warehouse ETL Toolkit: Practical
  Techniques for Extracting, Cleaning, Conforming, and Delivering Data.
  Wiley, 2004.
\item
  BOUMAN R.; DONGEN, J. Pentaho Solutions: Business Intelligence and
  Data Warehousing with Pentaho and MySQL. Wiley, 2009.
\end{enumerate}
\end{minipage}}
\end{scriptsize}

\newpage