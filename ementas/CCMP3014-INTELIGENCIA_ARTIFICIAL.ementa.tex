\begin{figure*}
    \centering
    \includegraphics[width=2.57cm]{./images/brasao_da_republica.jpeg}
    
    \textsf{Ministério da Educação}\\
    \textsf{Universidade Federal do Agreste de Pernambuco}\\
    \textsf{Bacharelado em Ciência da Computação}
\end{figure*}

\vspace{2cm}

\begin{scriptsize}
\noindent \fbox{
\begin{minipage}[t]{\textwidth}
\textsf{\scriptsize COMPONENTE CURRICULAR:} \\
\textbf{INTELIGÊNCIA ARTIFICIAL} \\
\textsf{\scriptsize CÓDIGO:} \\
\textbf{CCMP3014}
\end{minipage}}

\noindent \fbox{
\begin{minipage}[t]{0.293\textwidth}
\textsf{\scriptsize PERÍODO A SER OFERTADO:} \\
\textbf{5}
\end{minipage}
%
\begin{minipage}[t]{0.7\textwidth}
\textsf{\scriptsize NÚCLEO DE FORMAÇÃO:} \\
\textbf{CICLO PROFISSIONAL OU TRONCO COMUM}
\end{minipage}}

\noindent \fbox{
\begin{minipage}[t]{0.493\textwidth}
\textsf{\scriptsize TIPO:} \\
\textbf{OBRIGATÓRIO}
\end{minipage}
%
\begin{minipage}[t]{0.5\textwidth}
\textsf{\scriptsize CRÉDITOS:} \\
\textbf{4}
\end{minipage}}

\noindent \fbox{
\begin{minipage}[t]{0.3\textwidth}
\textsf{\scriptsize CARGA HORÁRIA TOTAL:} \\
\textbf{60}
\end{minipage}
%
\begin{minipage}[t]{0.19\textwidth}
\textsf{\scriptsize TEÓRICA:} \\
\textbf{60}
\end{minipage}
%
\begin{minipage}[t]{0.19\textwidth}
\textsf{\scriptsize PRÁTICA:} \\
\textbf{0}
\end{minipage}
%
\begin{minipage}[t]{0.30\textwidth}
\textsf{\scriptsize EAD-SEMIPRESENCIAL:} \\
\textbf{0}
\end{minipage}}

\noindent \fbox{
\begin{minipage}[t]{\textwidth}
\textsf{\scriptsize PRÉ-REQUISITOS:}
\begin{itemize}
\item
  CCMP3006 ALGORITMOS E ESTRUTURA DE DADOS I
\item
  CCMP3016 ALGORITMOS E ESTRUTURA DE DADOS II
\item
  CCMP3057 INTRODUÇÃO À PROGRAMAÇÃO
\item
  CCMP3064 PROJETO E ANÁLISE DE ALGORITMOS
\item
  CCMP3065 PARADIGMAS DE LINGUAGENS DE PROGRAMAÇÃO
\item
  MATM3008 LÓGICA MATEMÁTICA
\end{itemize}
\end{minipage}}

\noindent \fbox{
\begin{minipage}[t]{\textwidth}
\textsf{\scriptsize CORREQUISITOS:} 
Não há.
\end{minipage}}

\noindent \fbox{
\begin{minipage}[t]{\textwidth}
\textsf{\scriptsize REQUISITO DE CARGA HORÁRIA:} 
Não há.
\end{minipage}}

\noindent \fbox{
\begin{minipage}[t]{\textwidth}
\textsf{\scriptsize EMENTA:} \\
Introdução. Sistemas especialistas. Agentes Inteligentes. Resolução de
problemas por meio de busca. Problema de satisfação de restrição.
Linguagens Simbólicas. Esquemas para representação do conhecimento:
lógicos, em rede, estruturados, procedurais. Formalismos para a
representação de conhecimento incerto. Redes Bayesianas. Conjuntos e
lógica Fuzzy. Introdução à Computação Evolucionária. Algoritmos
Genéticos. Ajuste de parâmetros em algoritmos genéticos. Projeto.
\end{minipage}}

\noindent \fbox{
\begin{minipage}[t]{\textwidth}
\textsf{\scriptsize BIBLIOGRAFIA BÁSICA:}
\begin{enumerate}
\def\labelenumi{\arabic{enumi}.}
\item
  RUSSELL, Stuart e NORVIG, Peter. Inteligência Artificial. 3ª Ed.
  Campus, Rio de Janeiro, 2013.
\item
  LUGER, George F. Inteligência Artificial. 6ª Ed. Pearson, São Paulo,
  2014.
\item
  COPPIN, Ben. Intelgência Artificial. 1ª Ed. LTC, 2010.
\end{enumerate}
\end{minipage}}

\noindent \fbox{
\begin{minipage}[t]{\textwidth}
\textsf{\scriptsize BIBLIOGRAFIA COMPLEMENTAR:}
\begin{enumerate}
\def\labelenumi{\arabic{enumi}.}
\item
  ROSA, JOÃO LUIS GARCIA. Fundamentos da Inteligência Artificial. 1ª
  ed.~LTC, 2011.
\item
  BITTENCOURT, Guilherme. Inteligência artificial: ferramentas e
  teorias. 3. ed.~rev. Florianópolis, SC: Ed. Da UFSC, 2006.
\item
  RICH, Elaine; KNIGHT, Kevin. Inteligência artificial. 2ª ed.~Makron
  Books, São Paulo, 1994.
\end{enumerate}
\end{minipage}}
\end{scriptsize}

\newpage