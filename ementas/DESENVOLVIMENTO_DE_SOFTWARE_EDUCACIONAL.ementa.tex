\begin{figure*}
    \centering
    \includegraphics[width=2.57cm]{./images/brasao_da_republica.jpeg}
    
    \textsf{Ministério da Educação}\\
    \textsf{Universidade Federal do Agreste de Pernambuco}\\
    \textsf{Bacharelado em Ciência da Computação}
\end{figure*}

\vspace{0.5cm}

\begin{scriptsize}
\noindent \fbox{
\begin{minipage}[t]{\textwidth}
\textsf{\scriptsize COMPONENTE CURRICULAR:} \\
\textbf{DESENVOLVIMENTO DE SOFTWARE EDUCACIONAL} \\
\textsf{\scriptsize CÓDIGO:} \textbf{}
\end{minipage}}

\noindent \fbox{
\begin{minipage}[t]{0.293\textwidth}
\textsf{\scriptsize PERÍODO A SER OFERTADO:} \\
\textbf{0}
\end{minipage}
%
\begin{minipage}[t]{0.7\textwidth}
\textsf{\scriptsize NÚCLEO DE FORMAÇÃO:} \\
\textbf{COMPONENTES OPTATIVOS ÁREA TEMÁTICA TECNOLOGIA EDUCACIONAL}
\end{minipage}}

\noindent \fbox{
\begin{minipage}[t]{0.493\textwidth}
\textsf{\scriptsize TIPO:} \textbf{OPTATIVO}
\end{minipage}
%
\begin{minipage}[t]{0.5\textwidth}
\textsf{\scriptsize CRÉDITOS:} \textbf{4}
\end{minipage}}

\noindent \fbox{
\begin{minipage}[t]{0.3\textwidth}
\textsf{\scriptsize CARGA HORÁRIA TOTAL:} 
\textbf{60}
\end{minipage}
%
\begin{minipage}[t]{0.19\textwidth}
\textsf{\scriptsize TEÓRICA:} 
\textbf{30}
\end{minipage}
%
\begin{minipage}[t]{0.19\textwidth}
\textsf{\scriptsize PRÁTICA:} 
\textbf{30}
\end{minipage}
%
\begin{minipage}[t]{0.30\textwidth}
\textsf{\scriptsize EAD-SEMIPRESENCIAL:} 
\textbf{0}
\end{minipage}}

\noindent \fbox{
\begin{minipage}[t]{\textwidth}
\textsf{\scriptsize PRÉ-REQUISITOS:}
CCMP3018 ENGENHARIA DE SOFTWARE
\end{minipage}}

\noindent \fbox{
\begin{minipage}[t]{\textwidth}
\textsf{\scriptsize CORREQUISITOS:} 
Não há.
\end{minipage}}

\noindent \fbox{
\begin{minipage}[t]{\textwidth}
\textsf{\scriptsize REQUISITO DE CARGA HORÁRIA:} 
Não há.
\end{minipage}}

\noindent \fbox{
\begin{minipage}[t]{\textwidth}
\textsf{\scriptsize EMENTA:} \\
O computador no processo ensino-aprendizagem; Concepções de conhecimento
e Teorias Cognitivas. Qualidade de software (produto) e qualidade no
desenvolvimento (processo). Modelos de avaliação de softwares
educacionais. Etapas para o desenvolvimento de um software educacional
(ciclo de vida).
\end{minipage}}

\noindent \fbox{
\begin{minipage}[t]{\textwidth}
\textsf{\scriptsize BIBLIOGRAFIA BÁSICA:}
\begin{enumerate}
\def\labelenumi{\arabic{enumi}.}
\item
  BARBOSA, S. D. J.; Silva, B. S. Interação Humano-Computador. Rio de
  Janeiro: Elservier, 2010.
\item
  PREECE, Rogers e Sharp; Design de Interação Além da Interação
  Homem-computador. São Paulo: Bookman, 2005.
\item
  MAYER. R, Cambridge Handbook of Multimedia Learning. New York:
  Cambridge University Press, 2005.
\end{enumerate}
\end{minipage}}

\noindent \fbox{
\begin{minipage}[t]{\textwidth}
\textsf{\scriptsize BIBLIOGRAFIA COMPLEMENTAR:}
\begin{enumerate}
\def\labelenumi{\arabic{enumi}.}
\item
  COX, Kenia. Informática na Educação Escolar, Autores Associados, 2003.
\item
  LEVY, Pierre. Cibercultura, Editora 34, 1999.
\item
  RBIE -- Revista Brasileira de Informática na Educação ISSN 1414-5685
\item
  RENOTE -- Revista Novas Tecnologias na Educação ISSN 1679-1916.
\end{enumerate}
\end{minipage}}
\end{scriptsize}

\newpage