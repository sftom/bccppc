\begin{figure*}
    \centering
    \includegraphics[width=2.57cm]{./images/brasao_da_republica.jpeg}
    
    \textsf{Ministério da Educação}\\
    \textsf{Universidade Federal do Agreste de Pernambuco}\\
    \textsf{Bacharelado em Ciência da Computação}
\end{figure*}

\vspace{0.5cm}

\begin{scriptsize}
\noindent \fbox{
\begin{minipage}[t]{\textwidth}
\textsf{\scriptsize COMPONENTE CURRICULAR:} \\
\textbf{BIOMETRIA} \\
\textsf{\scriptsize CÓDIGO:} \textbf{CCMP3087}
\end{minipage}}

\noindent \fbox{
\begin{minipage}[t]{0.293\textwidth}
\textsf{\scriptsize PERÍODO A SER OFERTADO:} \\
\textbf{0}
\end{minipage}
%
\begin{minipage}[t]{0.7\textwidth}
\textsf{\scriptsize NÚCLEO DE FORMAÇÃO:} \\
\textbf{COMPONENTES OPTATIVOS ÁREA TEMÁTICA INTELIGÊNCIA COMPUTACIONAL}
\end{minipage}}

\noindent \fbox{
\begin{minipage}[t]{0.493\textwidth}
\textsf{\scriptsize TIPO:} \textbf{OPTATIVO}
\end{minipage}
%
\begin{minipage}[t]{0.5\textwidth}
\textsf{\scriptsize CRÉDITOS:} \textbf{4}
\end{minipage}}

\noindent \fbox{
\begin{minipage}[t]{0.3\textwidth}
\textsf{\scriptsize CARGA HORÁRIA TOTAL:} 
\textbf{60}
\end{minipage}
%
\begin{minipage}[t]{0.19\textwidth}
\textsf{\scriptsize TEÓRICA:} 
\textbf{60}
\end{minipage}
%
\begin{minipage}[t]{0.19\textwidth}
\textsf{\scriptsize PRÁTICA:} 
\textbf{0}
\end{minipage}
%
\begin{minipage}[t]{0.30\textwidth}
\textsf{\scriptsize EAD-SEMIPRESENCIAL:} 
\textbf{0}
\end{minipage}}

\noindent \fbox{
\begin{minipage}[t]{\textwidth}
\textsf{\scriptsize PRÉ-REQUISITOS:}
Não há.
\end{minipage}}

\noindent \fbox{
\begin{minipage}[t]{\textwidth}
\textsf{\scriptsize CORREQUISITOS:} 
Não há.
\end{minipage}}

\noindent \fbox{
\begin{minipage}[t]{\textwidth}
\textsf{\scriptsize REQUISITO DE CARGA HORÁRIA:} 
Não há.
\end{minipage}}

\noindent \fbox{
\begin{minipage}[t]{\textwidth}
\textsf{\scriptsize EMENTA:} \\
Introdução a Biometria. Etapas de um sistema computacional para
reconhecimento/verificação dos seguintes elementos: digitais, face, voz,
íris, retina, veias, mão, pé, assinaturas e manuscritos. Sistemas de
segurança biométricos. Seminário. Projeto.
\end{minipage}}

\noindent \fbox{
\begin{minipage}[t]{\textwidth}
\textsf{\scriptsize BIBLIOGRAFIA BÁSICA:}
\begin{enumerate}
\def\labelenumi{\arabic{enumi}.}
\item
  Jerzy Pejas, Andrzej Piegat. Enhanced Methods in Computer Security,
  Biometric and Artificial Intelligence Systems. Springer, 2004.
\item
  John R. Vacca. Biometric Technologies and Verification Systems.
  Butterworth-Heinemann, 2007.
\item
  Ted Dunstone, Neil Yager. Biometric System and Data Analysis: Design,
  Evaluation, and Data Mining. Springer, 2008.
\end{enumerate}
\end{minipage}}

\noindent \fbox{
\begin{minipage}[t]{\textwidth}
\textsf{\scriptsize BIBLIOGRAFIA COMPLEMENTAR:}
\begin{enumerate}
\def\labelenumi{\arabic{enumi}.}
\item
  Carvalho, M. A.; Andreozzi, V. L.; Codeço, C. T.; Barbosa, M. T. S.;
  Shimakura, S. E. Análise de Sobrevida: Teoria e Aplicações em Saúde.
  Ed. Fiocruz, 2005.
\item
  Sokal, R. R. and Rohlf, F. J. Biometry: the principles and practice of
  statistics in biological research. W. H. Freeman and Company, 1998.
\item
  Quinn, G. P., Keough, M. J. Experimental Design and Data Analysis for
  Biologists. Cambridge University Press, 2002.
\item
  Vieira, S. Bioestatística: Tópicos Avançados. Ed. Campus, 2003
\item
  Artigos recentes da área.
\end{enumerate}
\end{minipage}}
\end{scriptsize}

\newpage