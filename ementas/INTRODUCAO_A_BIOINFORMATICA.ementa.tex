\begin{figure*}
    \centering
    \includegraphics[width=2.57cm]{./images/brasao_da_republica.jpeg}
    
    \textsf{Ministério da Educação}\\
    \textsf{Universidade Federal do Agreste de Pernambuco}\\
    \textsf{Bacharelado em Ciência da Computação}
\end{figure*}

\vspace{0.5cm}

\begin{scriptsize}
\noindent \fbox{
\begin{minipage}[t]{\textwidth}
\textsf{\scriptsize COMPONENTE CURRICULAR:} \\
\textbf{INTRODUÇÃO À BIOINFORMÁTICA} \\
\textsf{\scriptsize CÓDIGO:} \textbf{}
\end{minipage}}

\noindent \fbox{
\begin{minipage}[t]{0.293\textwidth}
\textsf{\scriptsize PERÍODO A SER OFERTADO:} \\
\textbf{0}
\end{minipage}
%
\begin{minipage}[t]{0.7\textwidth}
\textsf{\scriptsize NÚCLEO DE FORMAÇÃO:} \\
\textbf{COMPONENTES OPTATIVOS ÁREA TEMÁTICA INTELIGÊNCIA COMPUTACIONAL}
\end{minipage}}

\noindent \fbox{
\begin{minipage}[t]{0.493\textwidth}
\textsf{\scriptsize TIPO:} \textbf{OPTATIVO}
\end{minipage}
%
\begin{minipage}[t]{0.5\textwidth}
\textsf{\scriptsize CRÉDITOS:} \textbf{4}
\end{minipage}}

\noindent \fbox{
\begin{minipage}[t]{0.3\textwidth}
\textsf{\scriptsize CARGA HORÁRIA TOTAL:} 
\textbf{60}
\end{minipage}
%
\begin{minipage}[t]{0.19\textwidth}
\textsf{\scriptsize TEÓRICA:} 
\textbf{60}
\end{minipage}
%
\begin{minipage}[t]{0.19\textwidth}
\textsf{\scriptsize PRÁTICA:} 
\textbf{0}
\end{minipage}
%
\begin{minipage}[t]{0.30\textwidth}
\textsf{\scriptsize EAD-SEMIPRESENCIAL:} 
\textbf{0}
\end{minipage}}

\noindent \fbox{
\begin{minipage}[t]{\textwidth}
\textsf{\scriptsize PRÉ-REQUISITOS:}
\begin{itemize}
\item
  INTELIGÊNCIA ARTIFICIAL
\item
  RECONHECIMENTO DE PADRÕES
\end{itemize}
\end{minipage}}

\noindent \fbox{
\begin{minipage}[t]{\textwidth}
\textsf{\scriptsize CORREQUISITOS:} 
Não há.
\end{minipage}}

\noindent \fbox{
\begin{minipage}[t]{\textwidth}
\textsf{\scriptsize REQUISITO DE CARGA HORÁRIA:} 
Não há.
\end{minipage}}

\noindent \fbox{
\begin{minipage}[t]{\textwidth}
\textsf{\scriptsize EMENTA:} \\
Introdução aos conceitos de Bio-informática e Biologia Molecular.
Aplicações envolvendo métodos computacionais, matemáticos e
estatísticos. Introdução aos diferentes tipos de dados biológicos,
Bancos de Dados Biológicos e ferramentas público. Alinhamento de
Sequências e Sequenciamento de DNA. Classificação e Anotação de
Sequências Biológicas. Estruturas de Dados Biológicos e Busca em
Cadeias. Transcrição, regulação e expressão gênica. Análise de dados de
microarray. Famílias de Proteínas e Predição de Estruturas. Modelagem de
Sistemas Biológicos. Árvores Evolucionárias e Filogenia.
\end{minipage}}

\noindent \fbox{
\begin{minipage}[t]{\textwidth}
\textsf{\scriptsize BIBLIOGRAFIA BÁSICA:}
\begin{enumerate}
\def\labelenumi{\arabic{enumi}.}
\item
  ALURU, S. Handbook of Computational Molecular Biology. Chapman \&
  Hall/CRC, 2006.
\item
  BAXEVANIS, A., OUELLETTE, F. Bioinformatics: A Practical Guide to the
  Analysis of Genes and Proteins. Wiley-Interscience, 1998.
\item
  SETUBAL, J. C., MEIDANIS, J. Introduction to Computational Molecular
  Biology. PSW Publ. Co., 1997.
\item
  ALBERTS, B. Essential Cell Biology: An Introduction to the Molecular
  Biology of the Cell. Garland Pub., 1997.
\end{enumerate}
\end{minipage}}

\noindent \fbox{
\begin{minipage}[t]{\textwidth}
\textsf{\scriptsize BIBLIOGRAFIA COMPLEMENTAR:}
\begin{enumerate}
\def\labelenumi{\arabic{enumi}.}
\item
  Artigos recentes da área.
\item
\end{enumerate}
\end{minipage}}
\end{scriptsize}

\newpage