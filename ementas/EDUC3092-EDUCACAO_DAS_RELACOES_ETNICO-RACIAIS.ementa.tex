\begin{figure*}
    \centering
    \includegraphics[width=2.57cm]{./images/brasao_da_republica.jpeg}
    
    \textsf{Ministério da Educação}\\
    \textsf{Universidade Federal do Agreste de Pernambuco}\\
    \textsf{Bacharelado em Ciência da Computação}
\end{figure*}

\vspace{0.5cm}

\begin{scriptsize}
\noindent \fbox{
\begin{minipage}[t]{\textwidth}
\textsf{\scriptsize COMPONENTE CURRICULAR:} \\
\textbf{EDUCAÇÃO DAS RELAÇÕES ÉTNICO-RACIAIS} \\
\textsf{\scriptsize CÓDIGO:} \textbf{EDUC3092}
\end{minipage}}

\noindent \fbox{
\begin{minipage}[t]{0.293\textwidth}
\textsf{\scriptsize PERÍODO A SER OFERTADO:} \\
\textbf{0}
\end{minipage}
%
\begin{minipage}[t]{0.7\textwidth}
\textsf{\scriptsize NÚCLEO DE FORMAÇÃO:} \\
\textbf{COMPONENTES OPTATIVOS ÁREA TEMÁTICA TECNOLOGIA EDUCACIONAL}
\end{minipage}}

\noindent \fbox{
\begin{minipage}[t]{0.493\textwidth}
\textsf{\scriptsize TIPO:} \textbf{OPTATIVO}
\end{minipage}
%
\begin{minipage}[t]{0.5\textwidth}
\textsf{\scriptsize CRÉDITOS:} \textbf{4}
\end{minipage}}

\noindent \fbox{
\begin{minipage}[t]{0.3\textwidth}
\textsf{\scriptsize CARGA HORÁRIA TOTAL:} 
\textbf{60}
\end{minipage}
%
\begin{minipage}[t]{0.19\textwidth}
\textsf{\scriptsize TEÓRICA:} 
\textbf{60}
\end{minipage}
%
\begin{minipage}[t]{0.19\textwidth}
\textsf{\scriptsize PRÁTICA:} 
\textbf{0}
\end{minipage}
%
\begin{minipage}[t]{0.30\textwidth}
\textsf{\scriptsize EAD-SEMIPRESENCIAL:} 
\textbf{0}
\end{minipage}}

\noindent \fbox{
\begin{minipage}[t]{\textwidth}
\textsf{\scriptsize PRÉ-REQUISITOS:}
Não há.
\end{minipage}}

\noindent \fbox{
\begin{minipage}[t]{\textwidth}
\textsf{\scriptsize CORREQUISITOS:} 
Não há.
\end{minipage}}

\noindent \fbox{
\begin{minipage}[t]{\textwidth}
\textsf{\scriptsize REQUISITO DE CARGA HORÁRIA:} 
Não há.
\end{minipage}}

\noindent \fbox{
\begin{minipage}[t]{\textwidth}
\textsf{\scriptsize EMENTA:} \\
Formação das identidades brasileiras: elementos históricos. Relações
sociais e étnico-raciais. África e Brasil, semelhanças e diferenças em
suas formações. Interações Brasil-África na contemporaneidade.
Preconceito, estereótipo, etnia, interculturalidade. A educação indígena
no Brasil, historicidade e perspectivas teórico-metodológicas. Ensino e
aprendizagem na perspectiva da pluralidade cultural. Pluralidade étnica
do nordeste e de Pernambuco: especificidades e situação
sócio-educacional. Multiculturalismo e transculturalismo crítico.
\end{minipage}}

\noindent \fbox{
\begin{minipage}[t]{\textwidth}
\textsf{\scriptsize BIBLIOGRAFIA BÁSICA:}
\begin{enumerate}
\def\labelenumi{\arabic{enumi}.}
\item
  MUNANGA, Kabenguele. Superando o racismo na escola. Brasília, MEC,
  2005.
\item
  RIBEIRO, Darcy. O povo brasileiro: formação e sentido do Brasil. São
  Paulo: Companhia das Letras, 2009.
\item
  SILVA, Aracy Lopes da; GRUPIONI, Luís Donizete Benzi (org.). A
  temática indígena na escola: novos subsídios para professores de 1o e
  2o graus. Brasília, MEC/MARI/UNESCO, 1995.
\end{enumerate}
\end{minipage}}

\noindent \fbox{
\begin{minipage}[t]{\textwidth}
\textsf{\scriptsize BIBLIOGRAFIA COMPLEMENTAR:}
\begin{enumerate}
\def\labelenumi{\arabic{enumi}.}
\item
  BRASIL. Diretrizes Curriculares Nacionais para a Educação das Relações
  Etnicorraciais e para o Ensino de História e Cultura Afro-Brasileira e
  Africana. Brasília: MEC/CNE 10/03/2004.
\item
  CASHMORE, Ellis. Dicionários de Relações Étnicas e Raciais. São
  Paulo-SP: Summus, 2000.
\item
  FOUCAULT, Michel. Microfísica do poder. Rio de Janeiro: Edições Graal,
  1979.
\item
  MEC/SECAD. Orientações e Ações para a Educação das Relações
  Étnico-raciais. Brasília-DF: MEC/SECAD, 2006.
\item
  SCHWARCZ, Lilia M. O espetáculo das raças: cientistas, instituições e
  questão racial no Brasil. 1870-1930. São Paulo: Cia das Letras, 2011.
\item
  HALL, Stuart. A identidade cultural na pós-modernidade. Tradução de
  Tomaz Tadeu da Silva, Graracira Lopes Louro. Rio de Janeiro: DP\&A,
  2006.
\end{enumerate}
\end{minipage}}
\end{scriptsize}

\newpage