\begin{figure*}
    \centering
    \includegraphics[width=2.57cm]{./images/brasao_da_republica.jpeg}
    
    \textsf{Ministério da Educação}\\
    \textsf{Universidade Federal do Agreste de Pernambuco}\\
    \textsf{Bacharelado em Ciência da Computação}
\end{figure*}

\vspace{0.5cm}

\begin{scriptsize}
\noindent \fbox{
\begin{minipage}[t]{\textwidth}
\textsf{\scriptsize COMPONENTE CURRICULAR:} \\
\textbf{REDES COMPLEXAS} \\
\textsf{\scriptsize CÓDIGO:} \textbf{}
\end{minipage}}

\noindent \fbox{
\begin{minipage}[t]{0.293\textwidth}
\textsf{\scriptsize PERÍODO A SER OFERTADO:} \\
\textbf{0}
\end{minipage}
%
\begin{minipage}[t]{0.7\textwidth}
\textsf{\scriptsize NÚCLEO DE FORMAÇÃO:} \\
\textbf{COMPONENTES OPTATIVOS ÁREA TEMÁTICA INTELIGÊNCIA COMPUTACIONAL}
\end{minipage}}

\noindent \fbox{
\begin{minipage}[t]{0.493\textwidth}
\textsf{\scriptsize TIPO:} \textbf{OPTATIVO}
\end{minipage}
%
\begin{minipage}[t]{0.5\textwidth}
\textsf{\scriptsize CRÉDITOS:} \textbf{4}
\end{minipage}}

\noindent \fbox{
\begin{minipage}[t]{0.3\textwidth}
\textsf{\scriptsize CARGA HORÁRIA TOTAL:} 
\textbf{60}
\end{minipage}
%
\begin{minipage}[t]{0.19\textwidth}
\textsf{\scriptsize TEÓRICA:} 
\textbf{30}
\end{minipage}
%
\begin{minipage}[t]{0.19\textwidth}
\textsf{\scriptsize PRÁTICA:} 
\textbf{30}
\end{minipage}
%
\begin{minipage}[t]{0.30\textwidth}
\textsf{\scriptsize EAD-SEMIPRESENCIAL:} 
\textbf{0}
\end{minipage}}

\noindent \fbox{
\begin{minipage}[t]{\textwidth}
\textsf{\scriptsize PRÉ-REQUISITOS:}
CCMP3064 PROJETO E ANÁLISE DE ALGORITMOS
\end{minipage}}

\noindent \fbox{
\begin{minipage}[t]{\textwidth}
\textsf{\scriptsize CORREQUISITOS:} 
Não há
\end{minipage}}

\noindent \fbox{
\begin{minipage}[t]{\textwidth}
\textsf{\scriptsize REQUISITO DE CARGA HORÁRIA:} 
Não há
\end{minipage}}

\noindent \fbox{
\begin{minipage}[t]{\textwidth}
\textsf{\scriptsize EMENTA:} \\
Introdução às Redes Complexas. Redes sociais, biológicas e tecnológicas.
Revisão e métricas sobre grafos. Modelos de redes complexas: aleatória,
pequeno mundo e livre de escala. Algoritmos sobre redes complexas.
Epidemias em redes complexas. Trabalhos atuais. Projeto.
\end{minipage}}

\noindent \fbox{
\begin{minipage}[t]{\textwidth}
\textsf{\scriptsize BIBLIOGRAFIA BÁSICA:}
\begin{enumerate}
\def\labelenumi{\arabic{enumi}.}
\item
  NEWMAN, Mark. \textbf{Networks: An Introduction}. Oxford University
  Press, 2010.
\item
  BARABÁSI, Albert-László. \textbf{Network Science}. Cambridge
  University Press, 2016.
\item
  NEWMAN, M. E. J.; BARABÁSI, A.-L.; WATTS, D. J.
  \textbf{The Structure and Dynamics of Networks}. Princeton University
  Press, 2006.
\end{enumerate}
\end{minipage}}

\noindent \fbox{
\begin{minipage}[t]{\textwidth}
\textsf{\scriptsize BIBLIOGRAFIA COMPLEMENTAR:}
\begin{enumerate}
\def\labelenumi{\arabic{enumi}.}
\item
  BARRATA, Alain; BARTHÉLEMY, Marc; VESPIGNANI, Alessandro.
  \textbf{Dynamical Processes on Complex Networks}. Cambridge University
  Press, 2008.
\item
  CARRINGTON, Peter J.; SCOTT, John; WASSERMAN, Stanley.
  \textbf{Models and Methods in Social Network Analysis}. Cambridge
  University Press, 2005.
\item
  DOROGOVTSEV, S.N.; MENDES, J.F.F.
  \textbf{Evolution of Networks: From biological networks to the Internet and WWW}.
  Oxford University Press, 2003.
\item
  BARABÁSI, Albert-László.
  \textbf{Linked: How Everything Is Connected to Everything Else and What It Means}.
  Plume, 2003.
\item
  WATTS, Duncan J. \textbf{Six Degrees: The Science of a Connected Age}.
  W. W. Norton \& Company, 2003.
\end{enumerate}
\end{minipage}}
\end{scriptsize}

\newpage