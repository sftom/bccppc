\begin{figure*}
    \centering
    \includegraphics[width=2.57cm]{./images/brasao_da_republica.jpeg}
    
    \textsf{Ministério da Educação}\\
    \textsf{Universidade Federal do Agreste de Pernambuco}\\
    \textsf{Bacharelado em Ciência da Computação}
\end{figure*}

\vspace{2cm}

\begin{scriptsize}
\noindent \fbox{
\begin{minipage}[t]{\textwidth}
\textsf{\scriptsize COMPONENTE CURRICULAR:} \\
\textbf{BANCO DE DADOS I} \\
\textsf{\scriptsize CÓDIGO:} \\
\textbf{CCMP3066}
\end{minipage}}

\noindent \fbox{
\begin{minipage}[t]{0.293\textwidth}
\textsf{\scriptsize PERÍODO A SER OFERTADO:} \\
\textbf{4}
\end{minipage}
%
\begin{minipage}[t]{0.7\textwidth}
\textsf{\scriptsize NÚCLEO DE FORMAÇÃO:} \\
\textbf{CICLO PROFISSIONAL OU TRONCO COMUM}
\end{minipage}}

\noindent \fbox{
\begin{minipage}[t]{0.493\textwidth}
\textsf{\scriptsize TIPO:} \\
\textbf{OBRIGATÓRIO}
\end{minipage}
%
\begin{minipage}[t]{0.5\textwidth}
\textsf{\scriptsize CRÉDITOS:} \\
\textbf{4}
\end{minipage}}

\noindent \fbox{
\begin{minipage}[t]{0.3\textwidth}
\textsf{\scriptsize CARGA HORÁRIA TOTAL:} \\
\textbf{60}
\end{minipage}
%
\begin{minipage}[t]{0.19\textwidth}
\textsf{\scriptsize TEÓRICA:} \\
\textbf{30}
\end{minipage}
%
\begin{minipage}[t]{0.19\textwidth}
\textsf{\scriptsize PRÁTICA:} \\
\textbf{30}
\end{minipage}
%
\begin{minipage}[t]{0.30\textwidth}
\textsf{\scriptsize EAD-SEMIPRESENCIAL:} \\
\textbf{0}
\end{minipage}}

\noindent \fbox{
\begin{minipage}[t]{\textwidth}
\textsf{\scriptsize PRÉ-REQUISITOS:}
Não há.
\end{minipage}}

\noindent \fbox{
\begin{minipage}[t]{\textwidth}
\textsf{\scriptsize CORREQUISITOS:} 
Não há.
\end{minipage}}

\noindent \fbox{
\begin{minipage}[t]{\textwidth}
\textsf{\scriptsize REQUISITO DE CARGA HORÁRIA:} 
Não há.
\end{minipage}}

\noindent \fbox{
\begin{minipage}[t]{\textwidth}
\textsf{\scriptsize EMENTA:} \\
Conceituação. Arquitetura de SGDB. Modelagem de dados: modelo E-R e suas
variações, abstrações por agregação e generalização. Modelos de
representação (relacional, hierárquico e redes). Normalização e
manutenção da integridade. Arquitetura de Sistemas de Bancos de Dados,
SQL.
\end{minipage}}

\noindent \fbox{
\begin{minipage}[t]{\textwidth}
\textsf{\scriptsize BIBLIOGRAFIA BÁSICA:}
\begin{enumerate}
\def\labelenumi{\arabic{enumi}.}
\item
  ELMASRI, R. E.; NAVATHE, S. Sistemas de Banco de Dados. Pearson
  Education - Br, 2011.
\item
  SILBERSCHATZ, A.; KORTH, H.; SUDARSHAN, S. Sistemas de Bancos de
  Dados. Elsevier - Campus, 2012.
\item
  DATE, C. J. Introdução a Sistemas de Bancos de Dados. Campus, 2004.
\end{enumerate}
\end{minipage}}

\noindent \fbox{
\begin{minipage}[t]{\textwidth}
\textsf{\scriptsize BIBLIOGRAFIA COMPLEMENTAR:}
\begin{enumerate}
\def\labelenumi{\arabic{enumi}.}
\item
  GOUVEIA, FELIZ. Fundamentos de Banco de Dados. Editora FCA, 2014.
\item
  JASON, PRICE. Oracle 11g SQL. Bookman, 2008.
\item
  ROBERT, P.; CORONEL, C. Sistemas de Banco de Dados - Projeto,
  Implementação e Administração. 8. ed.~Cengage Learning, 2011.
\item
  CARDOSO, V.; CARDOSO, V. Linguagem em SQL -- Fundamentos e Práticas.
  Saraiva, 2013.
\item
  DATA, D. J. Projeto de Banco de Dados e Teoria Relacional -- Formam
  Normais e Tudo o Mais. Editora Novatec, 2015.
\end{enumerate}
\end{minipage}}
\end{scriptsize}

\newpage