\begin{figure*}
    \centering
    \includegraphics[width=2.57cm]{./images/brasao_da_republica.jpeg}
    
    \textsf{Ministério da Educação}\\
    \textsf{Universidade Federal do Agreste de Pernambuco}\\
    \textsf{Bacharelado em Ciência da Computação}
\end{figure*}

\vspace{2cm}

\begin{scriptsize}
\noindent \fbox{
\begin{minipage}[t]{\textwidth}
\textsf{\scriptsize COMPONENTE CURRICULAR:} \\
\textbf{EMPREENDEDORISMO I} \\
\textsf{\scriptsize CÓDIGO:} \\
\textbf{BCC00003}
\end{minipage}}

\noindent \fbox{
\begin{minipage}[t]{0.293\textwidth}
\textsf{\scriptsize PERÍODO A SER OFERTADO:} \\
\textbf{6}
\end{minipage}
%
\begin{minipage}[t]{0.7\textwidth}
\textsf{\scriptsize NÚCLEO DE FORMAÇÃO:} \\
\textbf{CICLO PROFISSIONAL OU TRONCO COMUM}
\end{minipage}}

\noindent \fbox{
\begin{minipage}[t]{0.493\textwidth}
\textsf{\scriptsize TIPO:} \\
\textbf{OBRIGATÓRIO}
\end{minipage}
%
\begin{minipage}[t]{0.5\textwidth}
\textsf{\scriptsize CRÉDITOS:} \\
\textbf{4}
\end{minipage}}

\noindent \fbox{
\begin{minipage}[t]{0.3\textwidth}
\textsf{\scriptsize CARGA HORÁRIA TOTAL:} \\
\textbf{60}
\end{minipage}
%
\begin{minipage}[t]{0.19\textwidth}
\textsf{\scriptsize TEÓRICA:} \\
\textbf{30}
\end{minipage}
%
\begin{minipage}[t]{0.19\textwidth}
\textsf{\scriptsize PRÁTICA:} \\
\textbf{30}
\end{minipage}
%
\begin{minipage}[t]{0.30\textwidth}
\textsf{\scriptsize EAD-SEMIPRESENCIAL:} \\
\textbf{0}
\end{minipage}}

\noindent \fbox{
\begin{minipage}[t]{\textwidth}
\textsf{\scriptsize PRÉ-REQUISITOS:}
Não há.
\end{minipage}}

\noindent \fbox{
\begin{minipage}[t]{\textwidth}
\textsf{\scriptsize CORREQUISITOS:} 
Não há.
\end{minipage}}

\noindent \fbox{
\begin{minipage}[t]{\textwidth}
\textsf{\scriptsize REQUISITO DE CARGA HORÁRIA:} 
Não há.
\end{minipage}}

\noindent \fbox{
\begin{minipage}[t]{\textwidth}
\textsf{\scriptsize EMENTA:} \\
Apresentar os principais conceitos sobre empreendedorismo destacando sua
importância e contribuições. Apresentar o cenário do empreendedorismo e
inovação no Brasil, suas características e suas perspectivas. Fazer uma
introdução da inovação destacando seus principais tipos.Abordar a
definição do problema, introduzir os conceitos de funil de Ideias,
SCAMPER e outras. Apresentar o Design Thinking, a Matriz CSD, o Lean
Canvas. Definir o MVP e o Pitch. Definir o que é uma startup. Apresentar
o modelo de Plano de Negócios. Abordar o Investimento anjo, o ambientes
de inovação, os tipos de fomento à inovação, o ecossistema de inovação,
a tríplice-hélice e outros.
\end{minipage}}

\noindent \fbox{
\begin{minipage}[t]{\textwidth}
\textsf{\scriptsize BIBLIOGRAFIA BÁSICA:}
\begin{enumerate}
\def\labelenumi{\arabic{enumi}.}
\item
  DORNELAS, José C. A. Empreendedorismo: transformando ideias em
  negócios. 7 ed.~Editora Empreende, 2018.
\item
  DOLABELA, Fernando. Oficina do empreendedor. São Paulo: Sextante,
  2008.
\item
  LEITE, Emanuel -- O Fenômeno do Empreendedorismo, Ed saraiva. 2012.
\item
  OSTERWALDER, Alexander; PIGNEUR, Yves. Canvas. In: Business model
  generation: inovação em modelos de negócios. Rio de Janeiro: Alta
  Books, p.~16-25, 2011.
\item
  BROWN, Tim. Design Thinking: Uma metodologia poderosa para decretar o
  fim das velhas ideias. Rio de Janeiro: Elsevier, Campus, 2010.
\end{enumerate}
\end{minipage}}

\noindent \fbox{
\begin{minipage}[t]{\textwidth}
\textsf{\scriptsize BIBLIOGRAFIA COMPLEMENTAR:}
\begin{enumerate}
\def\labelenumi{\arabic{enumi}.}
\item
  DRUCKER, P. F. Inovação e Espírito Empreendedor: Prática e princípios.
  São Paulo: Cengage Learning, 2016.
\item
  GERBER, Michael E. O Mito do Empreendedor. Ed Fundamento. 2014.
\item
  FOSS, L.; GIBSON, D. V. (Ed.). The Entrepreneurial University: Context
  and Institutional Change. Londres: Routledge, 2015.
\item
  LAM, A. From `ivory tower traditionalists to entrepreneurial
  scientists'? Academic scientists in fuzzy university-industry
  boundaries. Social Studies of Science, 2010, v. 40, n.~3, p.~307-340,
  2010.
\item
  MAZZUCATO, M. O Estado empreendedor. São Paulo: Portfolio-Penguin,
  2014.
\item
  MUELLER, R. M.; THORING, K. Design Thinking Vs Lean Startup: A
  Comparison of Two Userdriven Innovation Strategies. Proceedings of
  2012 International Design Management Research Conference.
  Anais\ldots2012.
\item
  LEITE, Emanuel, O FENÔMENO DO EMPREENDEDORISMO CRIANDO. RIQUEZAS,
  Editora Bagaço, Recife, 2000;
\item
  STICKDORN, Marc. Isto é design thinking de serviços -- Fundamentos,
  ferramentas e casos. Porto Alegre: Bookman, 2014.
\item
  Artigos diversos da área
\end{enumerate}
\end{minipage}}
\end{scriptsize}

\newpage