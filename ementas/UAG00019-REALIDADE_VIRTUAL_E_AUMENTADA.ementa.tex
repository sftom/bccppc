\begin{figure*}
    \centering
    \includegraphics[width=2.57cm]{./images/brasao_da_republica.jpeg}
    
    \textsf{Ministério da Educação}\\
    \textsf{Universidade Federal do Agreste de Pernambuco}\\
    \textsf{Bacharelado em Ciência da Computação}
\end{figure*}

\vspace{2cm}

\begin{scriptsize}
\noindent \fbox{
\begin{minipage}[t]{\textwidth}
\textsf{\scriptsize COMPONENTE CURRICULAR:} \\
\textbf{REALIDADE VIRTUAL E AUMENTADA} \\
\textsf{\scriptsize CÓDIGO:} \\
\textbf{UAG00019}
\end{minipage}}

\noindent \fbox{
\begin{minipage}[t]{0.293\textwidth}
\textsf{\scriptsize PERÍODO A SER OFERTADO:} \\
\textbf{0}
\end{minipage}
%
\begin{minipage}[t]{0.7\textwidth}
\textsf{\scriptsize NÚCLEO DE FORMAÇÃO:} \\
\textbf{COMPONENTES OPTATIVOS ÁREA TEMÁTICA MÍDIA E INTERAÇÃO}
\end{minipage}}

\noindent \fbox{
\begin{minipage}[t]{0.493\textwidth}
\textsf{\scriptsize TIPO:} \\
\textbf{OPTATIVO}
\end{minipage}
%
\begin{minipage}[t]{0.5\textwidth}
\textsf{\scriptsize CRÉDITOS:} \\
\textbf{4}
\end{minipage}}

\noindent \fbox{
\begin{minipage}[t]{0.3\textwidth}
\textsf{\scriptsize CARGA HORÁRIA TOTAL:} \\
\textbf{60}
\end{minipage}
%
\begin{minipage}[t]{0.19\textwidth}
\textsf{\scriptsize TEÓRICA:} \\
\textbf{60}
\end{minipage}
%
\begin{minipage}[t]{0.19\textwidth}
\textsf{\scriptsize PRÁTICA:} \\
\textbf{0}
\end{minipage}
%
\begin{minipage}[t]{0.30\textwidth}
\textsf{\scriptsize EAD-SEMIPRESENCIAL:} \\
\textbf{0}
\end{minipage}}

\noindent \fbox{
\begin{minipage}[t]{\textwidth}
\textsf{\scriptsize PRÉ-REQUISITOS:}
\begin{itemize}
\item
  CCMP3019 COMPUTAÇÃO GRÁFICA
\item
  CCMP3057 INTRODUÇÃO À PROGRAMAÇÃO
\item
  MATM3019 ÁLGEBRA LINEAR I
\item
  MATM3021 GEOMETRIA ANALÍTICA A
\end{itemize}
\end{minipage}}

\noindent \fbox{
\begin{minipage}[t]{\textwidth}
\textsf{\scriptsize CORREQUISITOS:} 
Não há.
\end{minipage}}

\noindent \fbox{
\begin{minipage}[t]{\textwidth}
\textsf{\scriptsize REQUISITO DE CARGA HORÁRIA:} 
Não há.
\end{minipage}}

\noindent \fbox{
\begin{minipage}[t]{\textwidth}
\textsf{\scriptsize EMENTA:} \\
Introdução - Histórico. Aplicações: mercados tradicionais e emergentes.
Tecnologias Básicas. Definições e Caracterizações. Fatores Humanos,
Percepção Humana e Interação. Princípios Básicos de Computação Gráfica
aplicados a RV e RA. Princípios de Modelagem Geométrica Aplicados a RV e
RA. Modelagem de Ambientes Virtuais. Ferramentas de Desenvolvimento de
Ambientes Virtuais. Tópicos Especiais em Realidade Virtual.
\end{minipage}}

\noindent \fbox{
\begin{minipage}[t]{\textwidth}
\textsf{\scriptsize BIBLIOGRAFIA BÁSICA:}
\begin{enumerate}
\def\labelenumi{\arabic{enumi}.}
\item
  Desenvolvimento De Jogos 3d E Aplicações Em Realidade Virtual,
  AZEVEDO, EDUARDO; STELKO, MICHELLE; MEYER, HOMERO. CAMPUS. ISBN:
  8535215697. 2005.
\item
  Craig AB. Understanding augmented reality: concepts and applications.
  Morgan Kaufmann, Waltham. 2013.
\item
  Kipper G, Rampolla J. Augmented Reality: An Emerging Technologies
  Guide to AR. 1st Edition, Syngress, 2012.
\end{enumerate}
\end{minipage}}

\noindent \fbox{
\begin{minipage}[t]{\textwidth}
\textsf{\scriptsize BIBLIOGRAFIA COMPLEMENTAR:}
\begin{enumerate}
\def\labelenumi{\arabic{enumi}.}
\item
  MCCARTH,Martin, et al.~Reality Architecture: Building 3DWorlds with
  Java and VRML. Hertfordshire: Prentice-Hall, 1998.
\item
  Grigore C. Burdea et al.~Virtual Reality Technology, 2nd. edition,
  Wiley-Interscience, 2003.
\item
  Oliver Bimber et al.~Spatial Augmented Reality: Merging Real and
  Virtual Worlds, A K Peters, 2005.
\item
  SCHMALSTIEG, D., AND HOLLERER, T. 2013. Augmented Reality: Theory and
  Practice. Addison-Wesley.
\item
  T. Parisi, Learning Virtual Reality - Developing Immersive Experiences
  and Applications for Desktop Web and Mobile, O'Reilly Media, 2015.
\end{enumerate}
\end{minipage}}
\end{scriptsize}

\newpage