\begin{figure*}
    \centering
    \includegraphics[width=2.57cm]{./images/brasao_da_republica.jpeg}
    
    \textsf{Ministério da Educação}\\
    \textsf{Universidade Federal do Agreste de Pernambuco}\\
    \textsf{Bacharelado em Ciência da Computação}
\end{figure*}

\vspace{2cm}

\begin{scriptsize}
\noindent \fbox{
\begin{minipage}[t]{\textwidth}
\textsf{\scriptsize COMPONENTE CURRICULAR:} \\
\textbf{AVALIAÇÃO DE DESEMPENHO DE SISTEMAS} \\
\textsf{\scriptsize CÓDIGO:} \\
\textbf{UAG00304}
\end{minipage}}

\noindent \fbox{
\begin{minipage}[t]{0.293\textwidth}
\textsf{\scriptsize PERÍODO A SER OFERTADO:} \\
\textbf{0}
\end{minipage}
%
\begin{minipage}[t]{0.7\textwidth}
\textsf{\scriptsize NÚCLEO DE FORMAÇÃO:} \\
\textbf{COMPONENTES OPTATIVOS ÁREA TEMÁTICA REDES E SISTEMAS
DISTRIBUÍDOS}
\end{minipage}}

\noindent \fbox{
\begin{minipage}[t]{0.493\textwidth}
\textsf{\scriptsize TIPO:} \\
\textbf{OPTATIVO}
\end{minipage}
%
\begin{minipage}[t]{0.5\textwidth}
\textsf{\scriptsize CRÉDITOS:} \\
\textbf{4}
\end{minipage}}

\noindent \fbox{
\begin{minipage}[t]{0.3\textwidth}
\textsf{\scriptsize CARGA HORÁRIA TOTAL:} \\
\textbf{60}
\end{minipage}
%
\begin{minipage}[t]{0.19\textwidth}
\textsf{\scriptsize TEÓRICA:} \\
\textbf{30}
\end{minipage}
%
\begin{minipage}[t]{0.19\textwidth}
\textsf{\scriptsize PRÁTICA:} \\
\textbf{30}
\end{minipage}
%
\begin{minipage}[t]{0.30\textwidth}
\textsf{\scriptsize EAD-SEMIPRESENCIAL:} \\
\textbf{0}
\end{minipage}}

\noindent \fbox{
\begin{minipage}[t]{\textwidth}
\textsf{\scriptsize PRÉ-REQUISITOS:}
\begin{itemize}
\item
  CCMP3006 ALGORITMOS E ESTRUTURA DE DADOS I
\item
  CCMP3016 ALGORITMOS E ESTRUTURA DE DADOS II
\item
  CCMP3057 INTRODUÇÃO À PROGRAMAÇÃO
\end{itemize}
\end{minipage}}

\noindent \fbox{
\begin{minipage}[t]{\textwidth}
\textsf{\scriptsize CORREQUISITOS:} 
Não há.
\end{minipage}}

\noindent \fbox{
\begin{minipage}[t]{\textwidth}
\textsf{\scriptsize REQUISITO DE CARGA HORÁRIA:} 
Não há.
\end{minipage}}

\noindent \fbox{
\begin{minipage}[t]{\textwidth}
\textsf{\scriptsize EMENTA:} \\
Conceitos sobre avaliação de desempenho. Conceitos básicos e erros em
medição. Técnicas de medição e ferramentas: medição direta, medição
indireta, tracing, benchmarking, medição baseada em amostragem. Tópicos
em estatística descritiva. Tópicos em inferência estatística.
Planejamento de capacidade. Estudo de casos.
\end{minipage}}

\noindent \fbox{
\begin{minipage}[t]{\textwidth}
\textsf{\scriptsize BIBLIOGRAFIA BÁSICA:}
\begin{enumerate}
\def\labelenumi{\arabic{enumi}.}
\item
  Raj Jain. ``Art of Computer Systems Performance Analysis: Techniques
  For Experimental Design Measurements Simulation And Modeling''. Wiley
  Computer Publishing, John Wiley \& Sons, Inc, 1991.
\item
  D. J. Lilja. ``Measuring computer performance: a practitioner's
  guide''. Cambridge Univ Pr, 2005.
\item
  J. Thienne. Avaliação de desempenho de sistemas computacionais. Rio de
  Janeiro: LTC, 2011.
\end{enumerate}
\end{minipage}}

\noindent \fbox{
\begin{minipage}[t]{\textwidth}
\textsf{\scriptsize BIBLIOGRAFIA COMPLEMENTAR:}
\begin{enumerate}
\def\labelenumi{\arabic{enumi}.}
\item
  Dror G. Feitelson. ``Workload Modeling for Computer Systems
  Performance Evaluation''. 2014. New York, NY, USA: Cambridge
  University Press.
\item
  Gunter Bolch, Stefan Greiner, Hermann de Meer, Kishor S. Trivedi.
  ``Queueing Networks and Markov Chains: Modeling and Performance
  Evaluation with Computer Science Applications''. Second Edition.
  WILEYINTERSCIENCE, 2007.
\item
  Kishor S. Trivedi. ``Probability and Statistics with Reliability,
  Queueing, and Computer Science Applications''. 2nd edition, Wiley,
  2002.
\item
  MAGALHÃES, Marcos Nascimento; LIMA, Antônio Carlos Pedroso de. Noções
  de probabilidade e estatística. 6. ed.~rev. São Paulo: Edusp, 2005.
  xv, 392p. ISBN 8531406773 (broch.).
\end{enumerate}
\end{minipage}}
\end{scriptsize}

\newpage