\begin{figure*}
    \centering
    \includegraphics[width=2.57cm]{./images/brasao_da_republica.jpeg}
    
    \textsf{Ministério da Educação}\\
    \textsf{Universidade Federal do Agreste de Pernambuco}\\
    \textsf{Bacharelado em Ciência da Computação}
\end{figure*}

\vspace{0.5cm}

\begin{scriptsize}
\noindent \fbox{
\begin{minipage}[t]{\textwidth}
\textsf{\scriptsize COMPONENTE CURRICULAR:} \\
\textbf{SISTEMAS DIGITAIS} \\
\textsf{\scriptsize CÓDIGO:} \textbf{CCMP3058}
\end{minipage}}

\noindent \fbox{
\begin{minipage}[t]{0.293\textwidth}
\textsf{\scriptsize PERÍODO A SER OFERTADO:} \\
\textbf{3}
\end{minipage}
%
\begin{minipage}[t]{0.7\textwidth}
\textsf{\scriptsize NÚCLEO DE FORMAÇÃO:} \\
\textbf{CICLO GERAL OU CICLO BÁSICO}
\end{minipage}}

\noindent \fbox{
\begin{minipage}[t]{0.493\textwidth}
\textsf{\scriptsize TIPO:} \textbf{OBRIGATÓRIO}
\end{minipage}
%
\begin{minipage}[t]{0.5\textwidth}
\textsf{\scriptsize CRÉDITOS:} \textbf{4}
\end{minipage}}

\noindent \fbox{
\begin{minipage}[t]{0.3\textwidth}
\textsf{\scriptsize CARGA HORÁRIA TOTAL:} 
\textbf{60}
\end{minipage}
%
\begin{minipage}[t]{0.19\textwidth}
\textsf{\scriptsize TEÓRICA:} 
\textbf{45}
\end{minipage}
%
\begin{minipage}[t]{0.19\textwidth}
\textsf{\scriptsize PRÁTICA:} 
\textbf{15}
\end{minipage}
%
\begin{minipage}[t]{0.30\textwidth}
\textsf{\scriptsize EAD-SEMIPRESENCIAL:} 
\textbf{0}
\end{minipage}}

\noindent \fbox{
\begin{minipage}[t]{\textwidth}
\textsf{\scriptsize PRÉ-REQUISITOS:}
Não há.
\end{minipage}}

\noindent \fbox{
\begin{minipage}[t]{\textwidth}
\textsf{\scriptsize CORREQUISITOS:} 
Não há.
\end{minipage}}

\noindent \fbox{
\begin{minipage}[t]{\textwidth}
\textsf{\scriptsize REQUISITO DE CARGA HORÁRIA:} 
Não há.
\end{minipage}}

\noindent \fbox{
\begin{minipage}[t]{\textwidth}
\textsf{\scriptsize EMENTA:} \\
Sistemas de Numeração e Códigos; Aritmética Binária; Porta Lógicas;
Análise e Projeto de Circuitos Combinacionais; Minimização por Mapa de
Karnaugh; Somadores; Decodificadores; Codificadores; Multiplexadores;
Demultiplexadores; Análise e Síntese de Circuitos Sequenciais; Latches e
Flip-Flops; Minimização de Estado; Registradores; Registradores de
Deslocamento; Dispositivos Lógicos Programáveis; Memória).
\end{minipage}}

\noindent \fbox{
\begin{minipage}[t]{\textwidth}
\textsf{\scriptsize BIBLIOGRAFIA BÁSICA:}
\begin{enumerate}
\def\labelenumi{\arabic{enumi}.}
\item
  TOCCI, R.J. Sistemas Digitais: Princípios e Aplicações. 10ª Ed.
  Pearson. São Paulo, 2007.
\item
  VAHID, F.: Sistemas Digitais: Projeto, Otimização e HDLs. Artmed,
  2008.
\item
  D'AMORE, R. VHDL: Descrição e Síntese de Circuitos Digitais. 1ª Ed.
  LTC. 2005.
\end{enumerate}
\end{minipage}}

\noindent \fbox{
\begin{minipage}[t]{\textwidth}
\textsf{\scriptsize BIBLIOGRAFIA COMPLEMENTAR:}
\begin{enumerate}
\def\labelenumi{\arabic{enumi}.}
\item
  Fletcher, W. I. An Engineering Approach to Digital Design. Englewood
  Cliffs, NJ: Prentice-Hall, 1980.
\item
  Mano, M. Morris. Computer Engineering: Hardware Design. Englewood
  Cliffs, NJ: Prentice-Hall, 1988.
\item
  Floyd, T. L. Digital Fundamentals, 10 ed.~Pearson Education, 2011.
\item
  PATTERSON, D; HENNESSY, J. L. Organização e Projeto de Computadores.
  4ª Ed. Campus. 2014.
\item
  DOETA, I. Elementos de eletrônica digital. 5ª Ed. Érica. São Paulo.
  2003
\end{enumerate}
\end{minipage}}
\end{scriptsize}

\newpage