\begin{figure*}
    \centering
    \includegraphics[width=2.57cm]{./images/brasao_da_republica.jpeg}
    
    \textsf{Ministério da Educação}\\
    \textsf{Universidade Federal do Agreste de Pernambuco}\\
    \textsf{Bacharelado em Ciência da Computação}
\end{figure*}

\vspace{2cm}

\begin{scriptsize}
\noindent \fbox{
\begin{minipage}[t]{\textwidth}
\textsf{\scriptsize COMPONENTE CURRICULAR:} \\
\textbf{CÁLCULO NUMÉRICO E COMPUTACIONAL} \\
\textsf{\scriptsize CÓDIGO:} \\
\textbf{MATM3017}
\end{minipage}}

\noindent \fbox{
\begin{minipage}[t]{0.293\textwidth}
\textsf{\scriptsize PERÍODO A SER OFERTADO:} \\
\textbf{0}
\end{minipage}
%
\begin{minipage}[t]{0.7\textwidth}
\textsf{\scriptsize NÚCLEO DE FORMAÇÃO:} \\
\textbf{COMPONENTES OPTATIVOS ÁREA TEMÁTICA MATEMÁTICA E SIMULAÇÃO
COMPUTACIONAL}
\end{minipage}}

\noindent \fbox{
\begin{minipage}[t]{0.493\textwidth}
\textsf{\scriptsize TIPO:} \\
\textbf{OPTATIVO}
\end{minipage}
%
\begin{minipage}[t]{0.5\textwidth}
\textsf{\scriptsize CRÉDITOS:} \\
\textbf{4}
\end{minipage}}

\noindent \fbox{
\begin{minipage}[t]{0.3\textwidth}
\textsf{\scriptsize CARGA HORÁRIA TOTAL:} \\
\textbf{60}
\end{minipage}
%
\begin{minipage}[t]{0.19\textwidth}
\textsf{\scriptsize TEÓRICA:} \\
\textbf{60}
\end{minipage}
%
\begin{minipage}[t]{0.19\textwidth}
\textsf{\scriptsize PRÁTICA:} \\
\textbf{0}
\end{minipage}
%
\begin{minipage}[t]{0.30\textwidth}
\textsf{\scriptsize EAD-SEMIPRESENCIAL:} \\
\textbf{0}
\end{minipage}}

\noindent \fbox{
\begin{minipage}[t]{\textwidth}
\textsf{\scriptsize PRÉ-REQUISITOS:}
\begin{itemize}
\item
  MATM3019 ÁLGEBRA LINEAR I
\item
  MATM3021 GEOMETRIA ANALÍTICA A
\item
  MATM3031 CÁLCULO PARA COMPUTAÇÃO I
\item
  MATM3032 CÁLCULO PARA COMPUTAÇÃO II
\end{itemize}
\end{minipage}}

\noindent \fbox{
\begin{minipage}[t]{\textwidth}
\textsf{\scriptsize CORREQUISITOS:} 
Não há.
\end{minipage}}

\noindent \fbox{
\begin{minipage}[t]{\textwidth}
\textsf{\scriptsize REQUISITO DE CARGA HORÁRIA:} 
Não há.
\end{minipage}}

\noindent \fbox{
\begin{minipage}[t]{\textwidth}
\textsf{\scriptsize EMENTA:} \\
Máquinas digitais: precisão, exatidão e erros. Aritmética de ponto
flutuante. Sistemas de enumeração. Sistemas lineares. Resolução
computacional de sistemas de equações lineares. Resolução de equações
transcendentes. Aproximação de funções: interpolação spline, ajustamento
de curvas, aproximação racional e por polinômios de Chebyschev.
Integração numérica: Newton-Cotes e quadratura Gaussiana.
\end{minipage}}

\noindent \fbox{
\begin{minipage}[t]{\textwidth}
\textsf{\scriptsize BIBLIOGRAFIA BÁSICA:}
\begin{enumerate}
\def\labelenumi{\arabic{enumi}.}
\item
  BARROSO, L. C. et al.~Cálculo Numérico (Com Aplicações), 2a edição.
  São Paulo: Editora Harbra, 1987.
\item
  FRANCO, Neide Bertoldi. Cálculo Numérico. São Paulo: Pearson Prentice
  Hall, 2006.
\item
  RUGGIERO, Marcia A. Gomes; LOPES, Vera Lucia da Rocha. Cálculo
  numérico: aspectos teóricos e computacionais. 2. ed.~São Paulo, SP:
  McGraw-Hill, 1996.
\end{enumerate}
\end{minipage}}

\noindent \fbox{
\begin{minipage}[t]{\textwidth}
\textsf{\scriptsize BIBLIOGRAFIA COMPLEMENTAR:}
\begin{enumerate}
\def\labelenumi{\arabic{enumi}.}
\item
  SPERANDIO, D., MENDES, J.T., MONKEN E SILVA, L.H. Cálculo Numérico:
  Características Matemáticas e Computacionais dos Métodos Numéricos.
  São Paulo: Prentice Hall, 2006.
\item
  ARENALES, Selma Helena de Vasconcelos; DAREZZO FILHO, Artur. Cálculo
  numérico: aprendizagem com apoio de software. São Paulo: Thomson
  Learning, 2008.
\item
  BURDEN, Richard L., DOUGLAS, J. Análise Numérica. São Paulo: Cengage
  Learning, 2008.
\item
  LEITHOLD, Louis. O Cálculo com Geometria Analítica Volume 1. 3ed. São
  Paulo: Harbra, 1994.
\item
  ANTON, Howard. Cálculo um Novo Horizonte V.1. 6ed. Porto alegre:
  Bookman, 2000.
\end{enumerate}
\end{minipage}}
\end{scriptsize}

\newpage