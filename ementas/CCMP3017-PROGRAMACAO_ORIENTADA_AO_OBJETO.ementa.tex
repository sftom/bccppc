\begin{figure*}
    \centering
    \includegraphics[width=2.57cm]{./images/brasao_da_republica.jpeg}
    
    \textsf{Ministério da Educação}\\
    \textsf{Universidade Federal do Agreste de Pernambuco}\\
    \textsf{Bacharelado em Ciência da Computação}
\end{figure*}

\vspace{2cm}

\begin{scriptsize}
\noindent \fbox{
\begin{minipage}[t]{\textwidth}
\textsf{\scriptsize COMPONENTE CURRICULAR:} \\
\textbf{PROGRAMAÇÃO ORIENTADA AO OBJETO} \\
\textsf{\scriptsize CÓDIGO:} \\
\textbf{CCMP3017}
\end{minipage}}

\noindent \fbox{
\begin{minipage}[t]{0.293\textwidth}
\textsf{\scriptsize PERÍODO A SER OFERTADO:} \\
\textbf{2}
\end{minipage}
%
\begin{minipage}[t]{0.7\textwidth}
\textsf{\scriptsize NÚCLEO DE FORMAÇÃO:} \\
\textbf{CICLO GERAL OU CICLO BÁSICO}
\end{minipage}}

\noindent \fbox{
\begin{minipage}[t]{0.493\textwidth}
\textsf{\scriptsize TIPO:} \\
\textbf{OBRIGATÓRIO}
\end{minipage}
%
\begin{minipage}[t]{0.5\textwidth}
\textsf{\scriptsize CRÉDITOS:} \\
\textbf{4}
\end{minipage}}

\noindent \fbox{
\begin{minipage}[t]{0.3\textwidth}
\textsf{\scriptsize CARGA HORÁRIA TOTAL:} \\
\textbf{60}
\end{minipage}
%
\begin{minipage}[t]{0.19\textwidth}
\textsf{\scriptsize TEÓRICA:} \\
\textbf{60}
\end{minipage}
%
\begin{minipage}[t]{0.19\textwidth}
\textsf{\scriptsize PRÁTICA:} \\
\textbf{0}
\end{minipage}
%
\begin{minipage}[t]{0.30\textwidth}
\textsf{\scriptsize EAD-SEMIPRESENCIAL:} \\
\textbf{0}
\end{minipage}}

\noindent \fbox{
\begin{minipage}[t]{\textwidth}
\textsf{\scriptsize PRÉ-REQUISITOS:}
CCMP3057 INTRODUÇÃO À PROGRAMAÇÃO
\end{minipage}}

\noindent \fbox{
\begin{minipage}[t]{\textwidth}
\textsf{\scriptsize CORREQUISITOS:} 
Não há.
\end{minipage}}

\noindent \fbox{
\begin{minipage}[t]{\textwidth}
\textsf{\scriptsize REQUISITO DE CARGA HORÁRIA:} 
Não há.
\end{minipage}}

\noindent \fbox{
\begin{minipage}[t]{\textwidth}
\textsf{\scriptsize EMENTA:} \\
Contextualização de Programação Orientada a Objetos. Conceitos de
orientação a Objetos:Objetos, Operações, Mensagens, Métodos e Estados.
Ambientes de Desenvolvimento de Software Orientado a Objetos. Classes,
Métodos e Objetos. Tipo de Entrada de Dados. Tipo de Saída de Dados.
Estruturas de Controle e Repetição. Modificadores de Classe e de
Métodos. Construtores e Finalizadores. Herança Simples e Múltipla: Super
e sub classes. Polimorfismo, Abstrações e Generalizações. Diagramas UML.
Interface Gráfica com Usuário.
\end{minipage}}

\noindent \fbox{
\begin{minipage}[t]{\textwidth}
\textsf{\scriptsize BIBLIOGRAFIA BÁSICA:}
\begin{enumerate}
\def\labelenumi{\arabic{enumi}.}
\item
  Kathy Sierra , Bert Bates. Use a Cabeça Java. Alta Books. 2ª Edição,
  2005.
\item
  Harvey M. Deitel, Paul J. Deitel. Java: Como Programar. Prentice Hall.
  10ª Edição, 2016.
\item
  Bruce Eckel. Thinking in Java. 4ª Edição, 2006.
\end{enumerate}
\end{minipage}}

\noindent \fbox{
\begin{minipage}[t]{\textwidth}
\textsf{\scriptsize BIBLIOGRAFIA COMPLEMENTAR:}
\begin{enumerate}
\def\labelenumi{\arabic{enumi}.}
\item
  Rafael Santos. Introdução à Programação Orientada à Objetos Usando
  Java. Ed. Campus. 2ª Edição, 2013.
\item
  Barnes, D. J., Kölling, M. Programação Orientada a Objetos com Java.
  Pearson/Prentice-Hall. 4ª Edição, 2004.
\item
  Horstmann, C. S., Cornell, G. Core Java 2 -- Fundamentos (Volume I).
  Alta Books. 8ª Edição, 2008.
\item
  Lynn Andrea Stein. Interactive Programming in Java. 2003.
\item
  Bertrand Meyer. Object-Oriented Software Construction. Prentice Hall.
  2ª Edição, 2000.
\end{enumerate}
\end{minipage}}
\end{scriptsize}

\newpage