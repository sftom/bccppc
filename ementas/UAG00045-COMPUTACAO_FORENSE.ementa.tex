\begin{figure*}
    \centering
    \includegraphics[width=2.57cm]{./images/brasao_da_republica.jpeg}
    
    \textsf{Ministério da Educação}\\
    \textsf{Universidade Federal do Agreste de Pernambuco}\\
    \textsf{Bacharelado em Ciência da Computação}
\end{figure*}

\vspace{2cm}

\begin{scriptsize}
\noindent \fbox{
\begin{minipage}[t]{\textwidth}
\textsf{\scriptsize COMPONENTE CURRICULAR:} \\
\textbf{COMPUTAÇÃO FORENSE} \\
\textsf{\scriptsize CÓDIGO:} \\
\textbf{UAG00045}
\end{minipage}}

\noindent \fbox{
\begin{minipage}[t]{0.293\textwidth}
\textsf{\scriptsize PERÍODO A SER OFERTADO:} \\
\textbf{0}
\end{minipage}
%
\begin{minipage}[t]{0.7\textwidth}
\textsf{\scriptsize NÚCLEO DE FORMAÇÃO:} \\
\textbf{COMPONENTES OPTATIVOS ÁREA TEMÁTICA INTELIGÊNCIA COMPUTACIONAL}
\end{minipage}}

\noindent \fbox{
\begin{minipage}[t]{0.493\textwidth}
\textsf{\scriptsize TIPO:} \\
\textbf{OPTATIVO}
\end{minipage}
%
\begin{minipage}[t]{0.5\textwidth}
\textsf{\scriptsize CRÉDITOS:} \\
\textbf{4}
\end{minipage}}

\noindent \fbox{
\begin{minipage}[t]{0.3\textwidth}
\textsf{\scriptsize CARGA HORÁRIA TOTAL:} \\
\textbf{60}
\end{minipage}
%
\begin{minipage}[t]{0.19\textwidth}
\textsf{\scriptsize TEÓRICA:} \\
\textbf{60}
\end{minipage}
%
\begin{minipage}[t]{0.19\textwidth}
\textsf{\scriptsize PRÁTICA:} \\
\textbf{0}
\end{minipage}
%
\begin{minipage}[t]{0.30\textwidth}
\textsf{\scriptsize EAD-SEMIPRESENCIAL:} \\
\textbf{0}
\end{minipage}}

\noindent \fbox{
\begin{minipage}[t]{\textwidth}
\textsf{\scriptsize PRÉ-REQUISITOS:}
\begin{itemize}
\item
  CCMP3006 ALGORITMOS E ESTRUTURA DE DADOS I
\item
  CCMP3014 INTELIGÊNCIA ARTIFICIAL
\item
  CCMP3016 ALGORITMOS E ESTRUTURA DE DADOS II
\item
  CCMP3043 RECONHECIMENTO DE PADRÕES
\item
  CCMP3057 INTRODUÇÃO À PROGRAMAÇÃO
\item
  CCMP3065 PARADIGMAS DE LINGUAGENS DE PROGRAMAÇÃO
\item
  MATM3008 LÓGICA MATEMÁTICA
\end{itemize}
\end{minipage}}

\noindent \fbox{
\begin{minipage}[t]{\textwidth}
\textsf{\scriptsize CORREQUISITOS:} 
Não há.
\end{minipage}}

\noindent \fbox{
\begin{minipage}[t]{\textwidth}
\textsf{\scriptsize REQUISITO DE CARGA HORÁRIA:} 
Não há.
\end{minipage}}

\noindent \fbox{
\begin{minipage}[t]{\textwidth}
\textsf{\scriptsize EMENTA:} \\
Princípios básicos de Ciência Forense e áreas de atuação. Apresentar os
conceitos básicos da perícia criminal e cível. Novas tecnologias
disponíveis nas áreas de Computação Forense e ferramentas tecnológicas
para processamento e análise de evidências. Seminário e Projeto.
\end{minipage}}

\noindent \fbox{
\begin{minipage}[t]{\textwidth}
\textsf{\scriptsize BIBLIOGRAFIA BÁSICA:}
\begin{enumerate}
\def\labelenumi{\arabic{enumi}.}
\item
  DUDA, R.O., HART, P.E., STORK, D.G. Pattern Classification. Second
  Edition. Wiley, 2001.
\item
  MARQUES, J. S. Reconhecimento de Padrões - Métodos Estatísticos e
  Neurais. IST Press, 2005.
\item
  BISHOP, C. M. Pattern Recognition and Machine Learning. Springer,
  2006.
\end{enumerate}
\end{minipage}}

\noindent \fbox{
\begin{minipage}[t]{\textwidth}
\textsf{\scriptsize BIBLIOGRAFIA COMPLEMENTAR:}
\begin{enumerate}
\def\labelenumi{\arabic{enumi}.}
\item
  MELO, Sandro. Computação Forense com Software Livre. Rio de Janeiro:
  Alta Books, 2009. 1ª edição.
\item
  MERCURI, Rebecca T. Challenges in Forensic Computing. Communications
  of the ACM. ACM, 2005.
\item
  US-CERT. Computer Forensics. Disponível em: http://www.us-cert.gov/.
  2008.
\item
  Artigos recentes da área.
\end{enumerate}
\end{minipage}}
\end{scriptsize}

\newpage