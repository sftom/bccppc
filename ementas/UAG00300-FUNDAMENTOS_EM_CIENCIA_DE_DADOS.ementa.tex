\begin{figure*}
    \centering
    \includegraphics[width=2.57cm]{./images/brasao_da_republica.jpeg}
    
    \textsf{Ministério da Educação}\\
    \textsf{Universidade Federal do Agreste de Pernambuco}\\
    \textsf{Bacharelado em Ciência da Computação}
\end{figure*}

\vspace{0.5cm}

\begin{scriptsize}
\noindent \fbox{
\begin{minipage}[t]{\textwidth}
\textsf{\scriptsize COMPONENTE CURRICULAR:} \\
\textbf{FUNDAMENTOS EM CIÊNCIA DE DADOS} \\
\textsf{\scriptsize CÓDIGO:} \textbf{UAG00300}
\end{minipage}}

\noindent \fbox{
\begin{minipage}[t]{0.293\textwidth}
\textsf{\scriptsize PERÍODO A SER OFERTADO:} \\
\textbf{0}
\end{minipage}
%
\begin{minipage}[t]{0.7\textwidth}
\textsf{\scriptsize NÚCLEO DE FORMAÇÃO:} \\
\textbf{COMPONENTES OPTATIVOS ÁREA TEMÁTICA TECNOLOGIAS DA INFORMAÇÃO}
\end{minipage}}

\noindent \fbox{
\begin{minipage}[t]{0.493\textwidth}
\textsf{\scriptsize TIPO:} \textbf{OPTATIVO}
\end{minipage}
%
\begin{minipage}[t]{0.5\textwidth}
\textsf{\scriptsize CRÉDITOS:} \textbf{4}
\end{minipage}}

\noindent \fbox{
\begin{minipage}[t]{0.3\textwidth}
\textsf{\scriptsize CARGA HORÁRIA TOTAL:} 
\textbf{60}
\end{minipage}
%
\begin{minipage}[t]{0.19\textwidth}
\textsf{\scriptsize TEÓRICA:} 
\textbf{60}
\end{minipage}
%
\begin{minipage}[t]{0.19\textwidth}
\textsf{\scriptsize PRÁTICA:} 
\textbf{0}
\end{minipage}
%
\begin{minipage}[t]{0.30\textwidth}
\textsf{\scriptsize EAD-SEMIPRESENCIAL:} 
\textbf{0}
\end{minipage}}

\noindent \fbox{
\begin{minipage}[t]{\textwidth}
\textsf{\scriptsize PRÉ-REQUISITOS:}
CCMP3066 BANCO DE DADOS I
\end{minipage}}

\noindent \fbox{
\begin{minipage}[t]{\textwidth}
\textsf{\scriptsize CORREQUISITOS:} 
Não há.
\end{minipage}}

\noindent \fbox{
\begin{minipage}[t]{\textwidth}
\textsf{\scriptsize REQUISITO DE CARGA HORÁRIA:} 
Não há.
\end{minipage}}

\noindent \fbox{
\begin{minipage}[t]{\textwidth}
\textsf{\scriptsize EMENTA:} \\
Apresentar os principais conceitos sobre ciência de dados, big data e
inteligência de negócios permitindo o conhecimento dos principais
conceitos relacionados a ciência de dados. Refletir e destacar a
importância da ciência de dados, big data e inteligência de negócios.
Apresentar as principais tecnologias para a ciência de dados.
\end{minipage}}

\noindent \fbox{
\begin{minipage}[t]{\textwidth}
\textsf{\scriptsize BIBLIOGRAFIA BÁSICA:}
\begin{enumerate}
\def\labelenumi{\arabic{enumi}.}
\item
  FAWCETT, T.\& PROVOST, F. Data Science para negócios: O que você
  precisa saber sobre mineração de dados e pensamento analítico de
  dados. Alta Books. Rio de Janeiro. 2016.
\item
  GRUS, Joel. Data Science do zero: Primeiras Regras com o Python. Alta
  Books. Rio de janeiro.2016.
\item
  OLIVEIRA, Paulo Felipe de; GUERRA, Saulo; MCDONNELL, Robert. Ciência
  de dados com R: Introdução. Editora IBPAD. Brasília:, 2018.
\end{enumerate}
\end{minipage}}

\noindent \fbox{
\begin{minipage}[t]{\textwidth}
\textsf{\scriptsize BIBLIOGRAFIA COMPLEMENTAR:}
\begin{enumerate}
\def\labelenumi{\arabic{enumi}.}
\item
  MCKINNEY, Wes. Python Para Análise de Dados: Tratamento de Dados com
  Pandas, NumPy e IPython. Novatec. Rio de Janeiro. 2018.
\item
  IGUAL, Laura; SEGUÍ, Santi. Introduction to Data Science: A Python
  Approach to Concepts, Techniques and Applications. Springer. 2017.
\item
  Artigos, periódicos e materiais da área
\end{enumerate}
\end{minipage}}
\end{scriptsize}

\newpage