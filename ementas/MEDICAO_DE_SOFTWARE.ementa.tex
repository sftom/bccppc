\begin{figure*}
    \centering
    \includegraphics[width=2.57cm]{./images/brasao_da_republica.jpeg}
    
    \textsf{Ministério da Educação}\\
    \textsf{Universidade Federal do Agreste de Pernambuco}\\
    \textsf{Bacharelado em Ciência da Computação}
\end{figure*}

\vspace{0.5cm}

\begin{scriptsize}
\noindent \fbox{
\begin{minipage}[t]{\textwidth}
\textsf{\scriptsize COMPONENTE CURRICULAR:} \\
\textbf{MEDIÇÃO DE SOFTWARE} \\
\textsf{\scriptsize CÓDIGO:} \textbf{}
\end{minipage}}

\noindent \fbox{
\begin{minipage}[t]{0.293\textwidth}
\textsf{\scriptsize PERÍODO A SER OFERTADO:} \\
\textbf{0}
\end{minipage}
%
\begin{minipage}[t]{0.7\textwidth}
\textsf{\scriptsize NÚCLEO DE FORMAÇÃO:} \\
\textbf{COMPONENTES OPTATIVOS ÁREA TEMÁTICA ENGENHARIA DE SOFTWARE}
\end{minipage}}

\noindent \fbox{
\begin{minipage}[t]{0.493\textwidth}
\textsf{\scriptsize TIPO:} \textbf{OPTATIVO}
\end{minipage}
%
\begin{minipage}[t]{0.5\textwidth}
\textsf{\scriptsize CRÉDITOS:} \textbf{2}
\end{minipage}}

\noindent \fbox{
\begin{minipage}[t]{0.3\textwidth}
\textsf{\scriptsize CARGA HORÁRIA TOTAL:} 
\textbf{30}
\end{minipage}
%
\begin{minipage}[t]{0.19\textwidth}
\textsf{\scriptsize TEÓRICA:} 
\textbf{15}
\end{minipage}
%
\begin{minipage}[t]{0.19\textwidth}
\textsf{\scriptsize PRÁTICA:} 
\textbf{15}
\end{minipage}
%
\begin{minipage}[t]{0.30\textwidth}
\textsf{\scriptsize EAD-SEMIPRESENCIAL:} 
\textbf{0}
\end{minipage}}

\noindent \fbox{
\begin{minipage}[t]{\textwidth}
\textsf{\scriptsize PRÉ-REQUISITOS:}
ENGENHARIA DE SOFTWARE
\end{minipage}}

\noindent \fbox{
\begin{minipage}[t]{\textwidth}
\textsf{\scriptsize CORREQUISITOS:} 
Não há.
\end{minipage}}

\noindent \fbox{
\begin{minipage}[t]{\textwidth}
\textsf{\scriptsize REQUISITO DE CARGA HORÁRIA:} 
Não há.
\end{minipage}}

\noindent \fbox{
\begin{minipage}[t]{\textwidth}
\textsf{\scriptsize EMENTA:} \\
Apresentar aos alunos os princípios subjacentes ao desenvolvimento de
software que precisa para satisfazer objetivos específicos, onde estas
metas devem ser expressas em termos mensuráveis Ela abrange os
princípios de medição de software, e as maneiras pelas quais elas são
usadas no planejamento de projetos de software, e no acompanhamento dos
projetos bem como sua realização. Discute a aplicação de técnicas de
medição de software para estes métodos de ensaio.
\end{minipage}}

\noindent \fbox{
\begin{minipage}[t]{\textwidth}
\textsf{\scriptsize BIBLIOGRAFIA BÁSICA:}
\begin{enumerate}
\def\labelenumi{\arabic{enumi}.}
\item
  PRESMANN. Engenharia de Software 7. ed.~776 p.~McGraw-Hill, 2011.
\item
  JONES. Applied Software Measurement 3. Ed. 700 p.~McGraw-Hill, 2008.
\item
  BROOKS. O Mítico homem mês 1.Ed. 328p. Campus. 2009.
\end{enumerate}
\end{minipage}}

\noindent \fbox{
\begin{minipage}[t]{\textwidth}
\textsf{\scriptsize BIBLIOGRAFIA COMPLEMENTAR:}
\begin{enumerate}
\def\labelenumi{\arabic{enumi}.}
\item
  VAZQUES ET. AL. Análise de Pontos de Função: Medição, Estimativas e
  Gerenciamento de Projetos de Software 7. Edição. Ed. Érica. 2007.
\item
  SOMMERVILLE. Engenharia de Software. 8. ed.~568 p.~Addison Wesley,
  2007.
\item
  MCGARRY ET. AL. Practical Software Measurement: Objective Information
  for Decision Makers. Addison-Wesley. 2001.
\end{enumerate}
\end{minipage}}
\end{scriptsize}

\newpage