\begin{figure*}
    \centering
    \includegraphics[width=2.57cm]{./images/brasao_da_republica.jpeg}
    
    \textsf{Ministério da Educação}\\
    \textsf{Universidade Federal do Agreste de Pernambuco}\\
    \textsf{Bacharelado em Ciência da Computação}
\end{figure*}

\vspace{0.5cm}

\begin{scriptsize}
\noindent \fbox{
\begin{minipage}[t]{\textwidth}
\textsf{\scriptsize COMPONENTE CURRICULAR:} \\
\textbf{TÓPICOS ESPECIAIS EM GESTÃO DE PROJETOS} \\
\textsf{\scriptsize CÓDIGO:} \textbf{CCMP3083}
\end{minipage}}

\noindent \fbox{
\begin{minipage}[t]{0.293\textwidth}
\textsf{\scriptsize PERÍODO A SER OFERTADO:} \\
\textbf{0}
\end{minipage}
%
\begin{minipage}[t]{0.7\textwidth}
\textsf{\scriptsize NÚCLEO DE FORMAÇÃO:} \\
\textbf{COMPONENTES OPTATIVOS ÁREA TEMÁTICA ENGENHARIA DE SOFTWARE}
\end{minipage}}

\noindent \fbox{
\begin{minipage}[t]{0.493\textwidth}
\textsf{\scriptsize TIPO:} \textbf{OPTATIVO}
\end{minipage}
%
\begin{minipage}[t]{0.5\textwidth}
\textsf{\scriptsize CRÉDITOS:} \textbf{4}
\end{minipage}}

\noindent \fbox{
\begin{minipage}[t]{0.3\textwidth}
\textsf{\scriptsize CARGA HORÁRIA TOTAL:} 
\textbf{60}
\end{minipage}
%
\begin{minipage}[t]{0.19\textwidth}
\textsf{\scriptsize TEÓRICA:} 
\textbf{60}
\end{minipage}
%
\begin{minipage}[t]{0.19\textwidth}
\textsf{\scriptsize PRÁTICA:} 
\textbf{0}
\end{minipage}
%
\begin{minipage}[t]{0.30\textwidth}
\textsf{\scriptsize EAD-SEMIPRESENCIAL:} 
\textbf{0}
\end{minipage}}

\noindent \fbox{
\begin{minipage}[t]{\textwidth}
\textsf{\scriptsize PRÉ-REQUISITOS:}
\begin{itemize}
\item
  CCMP3017 PROGRAMAÇÃO ORIENTADA AO OBJETO
\item
  CCMP3018 ENGENHARIA DE SOFTWARE
\item
  CCMP3057 INTRODUÇÃO À PROGRAMAÇÃO
\end{itemize}
\end{minipage}}

\noindent \fbox{
\begin{minipage}[t]{\textwidth}
\textsf{\scriptsize CORREQUISITOS:} 
Não há.
\end{minipage}}

\noindent \fbox{
\begin{minipage}[t]{\textwidth}
\textsf{\scriptsize REQUISITO DE CARGA HORÁRIA:} 
Não há.
\end{minipage}}

\noindent \fbox{
\begin{minipage}[t]{\textwidth}
\textsf{\scriptsize EMENTA:} \\
Breve histórico da evolução do gerenciamento de projetos; Modelos de
gerenciamento de projetos: Revisão PMBoK e dos conceitos de metodologias
ágeis; Categorização de projetos. Modelos de Maturidade. Inovação,
complexidade e incerteza em projetos. Sustentabilidade em projetos.
Projetos globais. Ferramentas de gerenciamento de projetos. Apresentação
das áreas de pesquisa em projetos.
\end{minipage}}

\noindent \fbox{
\begin{minipage}[t]{\textwidth}
\textsf{\scriptsize BIBLIOGRAFIA BÁSICA:}
\begin{enumerate}
\def\labelenumi{\arabic{enumi}.}
\item
  ALMEIDA, N., ALMEIDA, F. Metodologia de Gerenciamento de Portfólio.
  Rio de Janeiro: Editora Brasport, 2013
\item
  HELDMAN, Kim. Gerência de projetos: fundamentos. Rio de Janeiro:
  Elsevier, 2005
\item
  PROJECT MANAGEMENT INSTITUTE. Um guia do conhecimento em gerenciamento
  de projetos. 4. ed.~São Paulo: Project Management Institute, 2008.
  xxvi, 459 p.~ISBN 9781933890708
\end{enumerate}
\end{minipage}}

\noindent \fbox{
\begin{minipage}[t]{\textwidth}
\textsf{\scriptsize BIBLIOGRAFIA COMPLEMENTAR:}
\begin{enumerate}
\def\labelenumi{\arabic{enumi}.}
\item
  BERNARDES, Silva; MOREIRA, Mauricio. Microsoft Project 2013 - Gestão e
  desenvolvimento de Projetos. Editora Érica, 2013.
\item
  DINSMORE, P. Campbell e CAVALIERI, Adriane. Como se tornar um
  profissional em gerenciamento de projetos. 2a ed.~Rio de Janeiro: Ed.
  Qualitymark, 2005.
\item
  MAXIMIANO, A. C. Amaru. Administração de projetos: transformando
  idéias em resultados. 2a edição. São Paulo: Atlas, 2002.
\item
  PROJECT MANAGEMENT INSTITUTE. Organizational project management
  maturity model (OPM3). Third edition. Newtown Square: PMI, 2013.
\end{enumerate}
\end{minipage}}
\end{scriptsize}

\newpage