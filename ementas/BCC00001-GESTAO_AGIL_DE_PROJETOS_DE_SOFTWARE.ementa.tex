\begin{figure*}
    \centering
    \includegraphics[width=2.57cm]{./images/brasao_da_republica.jpeg}
    
    \textsf{Ministério da Educação}\\
    \textsf{Universidade Federal do Agreste de Pernambuco}\\
    \textsf{Bacharelado em Ciência da Computação}
\end{figure*}

\vspace{0.5cm}

\begin{scriptsize}
\noindent \fbox{
\begin{minipage}[t]{\textwidth}
\textsf{\scriptsize COMPONENTE CURRICULAR:} \\
\textbf{GESTÃO ÁGIL DE PROJETOS DE SOFTWARE} \\
\textsf{\scriptsize CÓDIGO:} \textbf{BCC00001}
\end{minipage}}

\noindent \fbox{
\begin{minipage}[t]{0.293\textwidth}
\textsf{\scriptsize PERÍODO A SER OFERTADO:} \\
\textbf{0}
\end{minipage}
%
\begin{minipage}[t]{0.7\textwidth}
\textsf{\scriptsize NÚCLEO DE FORMAÇÃO:} \\
\textbf{COMPONENTES OPTATIVOS ÁREA TEMÁTICA ENGENHARIA DE SOFTWARE}
\end{minipage}}

\noindent \fbox{
\begin{minipage}[t]{0.493\textwidth}
\textsf{\scriptsize TIPO:} \textbf{OPTATIVO}
\end{minipage}
%
\begin{minipage}[t]{0.5\textwidth}
\textsf{\scriptsize CRÉDITOS:} \textbf{4}
\end{minipage}}

\noindent \fbox{
\begin{minipage}[t]{0.3\textwidth}
\textsf{\scriptsize CARGA HORÁRIA TOTAL:} 
\textbf{60}
\end{minipage}
%
\begin{minipage}[t]{0.19\textwidth}
\textsf{\scriptsize TEÓRICA:} 
\textbf{30}
\end{minipage}
%
\begin{minipage}[t]{0.19\textwidth}
\textsf{\scriptsize PRÁTICA:} 
\textbf{30}
\end{minipage}
%
\begin{minipage}[t]{0.30\textwidth}
\textsf{\scriptsize EAD-SEMIPRESENCIAL:} 
\textbf{0}
\end{minipage}}

\noindent \fbox{
\begin{minipage}[t]{\textwidth}
\textsf{\scriptsize PRÉ-REQUISITOS:}
CCMP3018 ENGENHARIA DE SOFTWARE
\end{minipage}}

\noindent \fbox{
\begin{minipage}[t]{\textwidth}
\textsf{\scriptsize CORREQUISITOS:} 
Não há.
\end{minipage}}

\noindent \fbox{
\begin{minipage}[t]{\textwidth}
\textsf{\scriptsize REQUISITO DE CARGA HORÁRIA:} 
Não há.
\end{minipage}}

\noindent \fbox{
\begin{minipage}[t]{\textwidth}
\textsf{\scriptsize EMENTA:} \\
Fundamentos de Gerenciamento de Projetos (GP); Projetos de Software;
Processo de GP; Processo de Desenvolvimento de Software; Proposta e
Plano de Projeto; Atividade e Fases executadas na Gestão de um projeto;
Ferramentas de planejamento e gerenciamento de projetos; Gestão de
implantação; Métodos Ágeis; Ferramentas colaborativas; Gestão Ágil de
software; Scrum na Prática.
\end{minipage}}

\noindent \fbox{
\begin{minipage}[t]{\textwidth}
\textsf{\scriptsize BIBLIOGRAFIA BÁSICA:}
\begin{enumerate}
\def\labelenumi{\arabic{enumi}.}
\item
  PRESSMAN, Roger. Engenharia de Software - Uma Abordagem Profissional.
  Amgh Editora. 8a Ed. 2016;
\item
  SOMMERVILLE, Ian. Engenharia de Software, Pearson, 9a. edição. ISBN:
  978-0137035151. 2011;
\item
  SCHWABER, Ken; SUTHERLAND, Jeff. Guia do Scrum. Um guia definitivo
  para o Scrum: As regras do jogo. Tradução de CRUZ, Fábio et
  al.~Scrum.org e Scruminc, 2014.
\end{enumerate}
\end{minipage}}

\noindent \fbox{
\begin{minipage}[t]{\textwidth}
\textsf{\scriptsize BIBLIOGRAFIA COMPLEMENTAR:}
\begin{enumerate}
\def\labelenumi{\arabic{enumi}.}
\item
  PMI. Um Guia do Conhecimento em Gerenciamento de Projetos - Guia PMBOK
  5a Edição. EUA: Project Management Institute, 2013.
\item
  Camargo, Robson. Ribas, Thomaz. Gestão ágil de projetos. Saraiva.
  2019.
\item
  Carvalho, Marly Monteiro. Roque Rabechini Jr.~Fundamentos em Gestão de
  Projetos: construindo competências para gerenciar projetos. Atlas. 5
  Edição. 2019.
\item
  Introduction to Agile Methods. Sondra Ashmore and Kristin Runyan.
  Addison-Wesley. 2014.
\item
  Cranked: a lean and agile software development method. Steve Fenton.
  Fore 2014.
\end{enumerate}
\end{minipage}}
\end{scriptsize}

\newpage