\begin{figure*}
    \centering
    \includegraphics[width=2.57cm]{./images/brasao_da_republica.jpeg}
    
    \textsf{Ministério da Educação}\\
    \textsf{Universidade Federal do Agreste de Pernambuco}\\
    \textsf{Bacharelado em Ciência da Computação}
\end{figure*}

\vspace{0.5cm}

\begin{scriptsize}
\noindent \fbox{
\begin{minipage}[t]{\textwidth}
\textsf{\scriptsize COMPONENTE CURRICULAR:} \\
\textbf{MÉTODOS DE PESQUISA EM COMPUTAÇÃO} \\
\textsf{\scriptsize CÓDIGO:} \textbf{CCMP3084}
\end{minipage}}

\noindent \fbox{
\begin{minipage}[t]{0.293\textwidth}
\textsf{\scriptsize PERÍODO A SER OFERTADO:} \\
\textbf{0}
\end{minipage}
%
\begin{minipage}[t]{0.7\textwidth}
\textsf{\scriptsize NÚCLEO DE FORMAÇÃO:} \\
\textbf{COMPONENTES OPTATIVOS ÁREA TEMÁTICA METODOLOGIA E TÉCNICAS DA
COMPUTAÇÃO}
\end{minipage}}

\noindent \fbox{
\begin{minipage}[t]{0.493\textwidth}
\textsf{\scriptsize TIPO:} \textbf{OPTATIVO}
\end{minipage}
%
\begin{minipage}[t]{0.5\textwidth}
\textsf{\scriptsize CRÉDITOS:} \textbf{4}
\end{minipage}}

\noindent \fbox{
\begin{minipage}[t]{0.3\textwidth}
\textsf{\scriptsize CARGA HORÁRIA TOTAL:} 
\textbf{60}
\end{minipage}
%
\begin{minipage}[t]{0.19\textwidth}
\textsf{\scriptsize TEÓRICA:} 
\textbf{60}
\end{minipage}
%
\begin{minipage}[t]{0.19\textwidth}
\textsf{\scriptsize PRÁTICA:} 
\textbf{0}
\end{minipage}
%
\begin{minipage}[t]{0.30\textwidth}
\textsf{\scriptsize EAD-SEMIPRESENCIAL:} 
\textbf{0}
\end{minipage}}

\noindent \fbox{
\begin{minipage}[t]{\textwidth}
\textsf{\scriptsize PRÉ-REQUISITOS:}
CIEN3005 METODOLOGIA CIENTÍFICA
\end{minipage}}

\noindent \fbox{
\begin{minipage}[t]{\textwidth}
\textsf{\scriptsize CORREQUISITOS:} 
Não há.
\end{minipage}}

\noindent \fbox{
\begin{minipage}[t]{\textwidth}
\textsf{\scriptsize REQUISITO DE CARGA HORÁRIA:} 
Não há.
\end{minipage}}

\noindent \fbox{
\begin{minipage}[t]{\textwidth}
\textsf{\scriptsize EMENTA:} \\
A metodologia científica é o estudo de como se conduz/produz pesquisa
científica. Metodologia científica é necessária, entre outras razões,
para tornar os resultados da pesquisa mais confiáveis e possíveis de
serem reproduzidos, de forma independente, por outros pesquisadores.
Este curso irá apresentar estratégias e métodos para pesquisa em
computação desde a formulação do problema até a validação de uma
possível solução. Em particular, o curso irá focar em métodos
experimentais e explorar o papel da experimentação na pesquisa em
computação.
\end{minipage}}

\noindent \fbox{
\begin{minipage}[t]{\textwidth}
\textsf{\scriptsize BIBLIOGRAFIA BÁSICA:}
\begin{enumerate}
\def\labelenumi{\arabic{enumi}.}
\item
  Magne Jørgensen's site:
  http://simula.no/people/magnej/bibliography?b\_size:int=9999999\&b\_start:int=0\&-C=
\item
  ESERNET - the Experimental Software Engineering Network
\item
  Steve Easterbrook's course CSC2130S: Empirical Research Methods in
  Software Engineering, at University of Toronto;
\end{enumerate}
\end{minipage}}

\noindent \fbox{
\begin{minipage}[t]{\textwidth}
\textsf{\scriptsize BIBLIOGRAFIA COMPLEMENTAR:}
\begin{enumerate}
\def\labelenumi{\arabic{enumi}.}
\item
  Susan Sim's course ICS 280: Research Methodology for Software at UC
  Irvine;
\item
  Dewayne Perry's course EE382C Empirical Studies in Software
  Engineering at U Texas;
\item
  Mary Shaw's course 17-939A What Makes Good Research in Software
  Engineering at CMU;
\item
  Jim Herbsleb's course 17-810 Empirical Methods in Software Engineering
  Research at CMU;
\item
  Wilhelm Hasselbring's course 2.01.261 Research Methods in Software
  Engineering at U Oldenburg;
\item
  Philip Johnson's Readings in Empirical Evaluation for Budding Software
  Engineering Researchers at U Hawaii;
\item
  Letizia Jaccheri's Empirical software engineering course at NTNU
  Trondheim;
\item
  Andreas Zeller's course Empirical Software Engineering at Saarland U.
\item
  Jonathan Sillito. Qualitative Research Methods in Software Engineering
  (CPSC 601.23), Winter 2007, 2008 and 2009.
\item
  Ioannidis: Why Most Published Research Findings Are False.
\item
  Lung, J., Aranda, J., Easterbrook, S. M. and Wilson, G. V., On the
  Difficulty of Replicating Human Subjects Studies in Software
  Engineering. 30th ACM/IEEE International Conference on Software
  Engineering (ICSE'2008), Leipzig, Germany, May 10-18, 2008.
\end{enumerate}
\end{minipage}}
\end{scriptsize}

\newpage