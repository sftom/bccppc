\begin{figure*}
    \centering
    \includegraphics[width=2.57cm]{./images/brasao_da_republica.jpeg}
    
    \textsf{Ministério da Educação}\\
    \textsf{Universidade Federal do Agreste de Pernambuco}\\
    \textsf{Bacharelado em Ciência da Computação}
\end{figure*}

\vspace{2cm}

\begin{scriptsize}
\noindent \fbox{
\begin{minipage}[t]{\textwidth}
\textsf{\scriptsize COMPONENTE CURRICULAR:} \\
\textbf{SISTEMAS OPERACIONAIS} \\
\textsf{\scriptsize CÓDIGO:} \\
\textbf{CCMP3009}
\end{minipage}}

\noindent \fbox{
\begin{minipage}[t]{0.293\textwidth}
\textsf{\scriptsize PERÍODO A SER OFERTADO:} \\
\textbf{5}
\end{minipage}
%
\begin{minipage}[t]{0.7\textwidth}
\textsf{\scriptsize NÚCLEO DE FORMAÇÃO:} \\
\textbf{CICLO GERAL OU CICLO BÁSICO}
\end{minipage}}

\noindent \fbox{
\begin{minipage}[t]{0.493\textwidth}
\textsf{\scriptsize TIPO:} \\
\textbf{OBRIGATÓRIO}
\end{minipage}
%
\begin{minipage}[t]{0.5\textwidth}
\textsf{\scriptsize CRÉDITOS:} \\
\textbf{4}
\end{minipage}}

\noindent \fbox{
\begin{minipage}[t]{0.3\textwidth}
\textsf{\scriptsize CARGA HORÁRIA TOTAL:} \\
\textbf{60}
\end{minipage}
%
\begin{minipage}[t]{0.19\textwidth}
\textsf{\scriptsize TEÓRICA:} \\
\textbf{60}
\end{minipage}
%
\begin{minipage}[t]{0.19\textwidth}
\textsf{\scriptsize PRÁTICA:} \\
\textbf{0}
\end{minipage}
%
\begin{minipage}[t]{0.30\textwidth}
\textsf{\scriptsize EAD-SEMIPRESENCIAL:} \\
\textbf{0}
\end{minipage}}

\noindent \fbox{
\begin{minipage}[t]{\textwidth}
\textsf{\scriptsize PRÉ-REQUISITOS:}
\begin{itemize}
\item
  CCMP3010 ARQUITETURA DE COMPUTADORES
\item
  CCMP3058 SISTEMAS DIGITAIS
\item
  FISC3004 FÍSICA PARA COMPUTAÇÃO
\item
  MATM3031 CÁLCULO PARA COMPUTAÇÃO I
\end{itemize}
\end{minipage}}

\noindent \fbox{
\begin{minipage}[t]{\textwidth}
\textsf{\scriptsize CORREQUISITOS:} 
Não há.
\end{minipage}}

\noindent \fbox{
\begin{minipage}[t]{\textwidth}
\textsf{\scriptsize REQUISITO DE CARGA HORÁRIA:} 
Não há.
\end{minipage}}

\noindent \fbox{
\begin{minipage}[t]{\textwidth}
\textsf{\scriptsize EMENTA:} \\
Estrutura de um Sistema Operacional. Processos Concorrentes.
Escalonamento. Gerenciamento de Memória. Memória Virtual. Gerenciamento
de Disco. Sistemas de Arquivos. Proteção e Segurança. Sistemas
Distribuídos. Estudos de Caso.
\end{minipage}}

\noindent \fbox{
\begin{minipage}[t]{\textwidth}
\textsf{\scriptsize BIBLIOGRAFIA BÁSICA:}
\begin{enumerate}
\def\labelenumi{\arabic{enumi}.}
\item
  TANENBAUM, A. S. \& WOODHULL, A. S. Sistemas Operacionais: Projeto e
  Implementação. 3ª Edição, Porto Alegre, Ed. Bookman, 2008.
\item
  TANENBAUM, A \& Bos, H. Sistemas Operacionais Modernos. 4ª Edição,
  Pearson, 2016.
\item
  STALLINGS, W. Operating Systems: Internals and Design Principles.8a
  Ed. 2014.
\end{enumerate}
\end{minipage}}

\noindent \fbox{
\begin{minipage}[t]{\textwidth}
\textsf{\scriptsize BIBLIOGRAFIA COMPLEMENTAR:}
\begin{enumerate}
\def\labelenumi{\arabic{enumi}.}
\item
  Mckusisk, M. K.; Neville-Neil, G. V.; Watson, R. N. M. The Design and
  Implementation of the FreeBSD Operating System. 2ª
  Edição,Addison-Wesley Professional, 2014.
\item
  Penumuchu, C. V. Simple Real-time Operating System: A Kernel Inside
  View for a Beginner. 1ª Edição,Trafford Publishing,2007.
\item
  DEITEL; CHOFFNES. Sistemas Operacionais. Pearson Education. 3a Ed.
  2005.
\item
  Silberschatz, A. Fundamentos de Sistemas Operacionais. 9ª Edição,LTC,
  2015.
\end{enumerate}
\end{minipage}}
\end{scriptsize}

\newpage