\begin{figure*}
    \centering
    \includegraphics[width=2.57cm]{./images/brasao_da_republica.jpeg}
    
    \textsf{Ministério da Educação}\\
    \textsf{Universidade Federal do Agreste de Pernambuco}\\
    \textsf{Bacharelado em Ciência da Computação}
\end{figure*}

\vspace{2cm}

\begin{scriptsize}
\noindent \fbox{
\begin{minipage}[t]{\textwidth}
\textsf{\scriptsize COMPONENTE CURRICULAR:} \\
\textbf{COMPUTADORES E SOCIEDADE} \\
\textsf{\scriptsize CÓDIGO:} \\
\textbf{CCMP3071}
\end{minipage}}

\noindent \fbox{
\begin{minipage}[t]{0.293\textwidth}
\textsf{\scriptsize PERÍODO A SER OFERTADO:} \\
\textbf{7}
\end{minipage}
%
\begin{minipage}[t]{0.7\textwidth}
\textsf{\scriptsize NÚCLEO DE FORMAÇÃO:} \\
\textbf{CICLO PROFISSIONAL OU TRONCO COMUM}
\end{minipage}}

\noindent \fbox{
\begin{minipage}[t]{0.493\textwidth}
\textsf{\scriptsize TIPO:} \\
\textbf{OBRIGATÓRIO}
\end{minipage}
%
\begin{minipage}[t]{0.5\textwidth}
\textsf{\scriptsize CRÉDITOS:} \\
\textbf{2}
\end{minipage}}

\noindent \fbox{
\begin{minipage}[t]{0.3\textwidth}
\textsf{\scriptsize CARGA HORÁRIA TOTAL:} \\
\textbf{30}
\end{minipage}
%
\begin{minipage}[t]{0.19\textwidth}
\textsf{\scriptsize TEÓRICA:} \\
\textbf{30}
\end{minipage}
%
\begin{minipage}[t]{0.19\textwidth}
\textsf{\scriptsize PRÁTICA:} \\
\textbf{0}
\end{minipage}
%
\begin{minipage}[t]{0.30\textwidth}
\textsf{\scriptsize EAD-SEMIPRESENCIAL:} \\
\textbf{0}
\end{minipage}}

\noindent \fbox{
\begin{minipage}[t]{\textwidth}
\textsf{\scriptsize PRÉ-REQUISITOS:}
Não há.
\end{minipage}}

\noindent \fbox{
\begin{minipage}[t]{\textwidth}
\textsf{\scriptsize CORREQUISITOS:} 
Não há.
\end{minipage}}

\noindent \fbox{
\begin{minipage}[t]{\textwidth}
\textsf{\scriptsize REQUISITO DE CARGA HORÁRIA:} 
Não há.
\end{minipage}}

\noindent \fbox{
\begin{minipage}[t]{\textwidth}
\textsf{\scriptsize EMENTA:} \\
CONSEQÜÊNCIAS DA INFORMATIZAÇÃO DA SOCIEDADE: A informatização e o
aspecto educacional; Efeitos políticos e econômicos; Impactos sociais;
Informatização e privacidade; POLÍTICA NACIONAL DE INFORMÁTICA:
Indústria nacional de informática; O papel do analista e sistemas na
sociedade; AUTOMAÇÃO DE ATIVIDADES: Comerciais; Industriais;
Escritórios; APLICAÇÕES DA INFORMÁTICA: Científica; Administrativa;
Jurídica; Humanística; Educação; ERGONOMIA E DOENÇAS PROFISSIONAIS:
Tipos; Características.
\end{minipage}}

\noindent \fbox{
\begin{minipage}[t]{\textwidth}
\textsf{\scriptsize BIBLIOGRAFIA BÁSICA:}
\begin{enumerate}
\def\labelenumi{\arabic{enumi}.}
\item
  BRASIL. Sociedade da Informação no Brasil - Livro Verde. Brasília:
  Ministério da Ciência e Tecnologia. Imprensa Nacional, 2000.
\item
  CASTELS, M. A sociedade em rede: a era da informação, economia,
  sociedade e cultura. Vol. 1. São Paulo: Paz e Terra, 1999.
\item
  HENRY, J. Cultura da convergência : a colisão entre os velhos e novos
  meios de comunicação. São Paulo: Aleph, 2009.
\end{enumerate}
\end{minipage}}

\noindent \fbox{
\begin{minipage}[t]{\textwidth}
\textsf{\scriptsize BIBLIOGRAFIA COMPLEMENTAR:}
\begin{enumerate}
\def\labelenumi{\arabic{enumi}.}
\item
  LEVY, Pierre. Cibercultura, Editora 34, 1999.
\item
  MASIEIRO, Paulo C. Ética em Computação. São Paulo : Editora da
  Universidade de São Paulo. 2000.
\item
  SANTAELLA, L. Culturas e artes do pós-humano: da cultura das mídias à
  cibercultura. São Paulo: Paulus, 2003.
\item
  SANTAELLA, L. A ecologia pluralista da comunicação: conectividade,
  mobilidade, ubiquidade. São Paulo: Paulus, 2010.
\item
  THOMPSON, J.B. A mídia e a modernidade: uma teoria social da mídia.
  Petrópolis, RJ: Vozes, 2009.
\end{enumerate}
\end{minipage}}
\end{scriptsize}

\newpage