\begin{figure*}
    \centering
    \includegraphics[width=2.57cm]{./images/brasao_da_republica.jpeg}
    
    \textsf{Ministério da Educação}\\
    \textsf{Universidade Federal do Agreste de Pernambuco}\\
    \textsf{Bacharelado em Ciência da Computação}
\end{figure*}

\vspace{0.5cm}

\begin{scriptsize}
\noindent \fbox{
\begin{minipage}[t]{\textwidth}
\textsf{\scriptsize COMPONENTE CURRICULAR:} \\
\textbf{PROGRAMAÇÃO ORIENTADA A ASPECTOS} \\
\textsf{\scriptsize CÓDIGO:} \textbf{}
\end{minipage}}

\noindent \fbox{
\begin{minipage}[t]{0.293\textwidth}
\textsf{\scriptsize PERÍODO A SER OFERTADO:} \\
\textbf{0}
\end{minipage}
%
\begin{minipage}[t]{0.7\textwidth}
\textsf{\scriptsize NÚCLEO DE FORMAÇÃO:} \\
\textbf{COMPONENTES OPTATIVOS ÁREA TEMÁTICA ENGENHARIA DE SOFTWARE}
\end{minipage}}

\noindent \fbox{
\begin{minipage}[t]{0.493\textwidth}
\textsf{\scriptsize TIPO:} \textbf{OPTATIVO}
\end{minipage}
%
\begin{minipage}[t]{0.5\textwidth}
\textsf{\scriptsize CRÉDITOS:} \textbf{4}
\end{minipage}}

\noindent \fbox{
\begin{minipage}[t]{0.3\textwidth}
\textsf{\scriptsize CARGA HORÁRIA TOTAL:} 
\textbf{60}
\end{minipage}
%
\begin{minipage}[t]{0.19\textwidth}
\textsf{\scriptsize TEÓRICA:} 
\textbf{30}
\end{minipage}
%
\begin{minipage}[t]{0.19\textwidth}
\textsf{\scriptsize PRÁTICA:} 
\textbf{30}
\end{minipage}
%
\begin{minipage}[t]{0.30\textwidth}
\textsf{\scriptsize EAD-SEMIPRESENCIAL:} 
\textbf{0}
\end{minipage}}

\noindent \fbox{
\begin{minipage}[t]{\textwidth}
\textsf{\scriptsize PRÉ-REQUISITOS:}
Não há.
\end{minipage}}

\noindent \fbox{
\begin{minipage}[t]{\textwidth}
\textsf{\scriptsize CORREQUISITOS:} 
Não há.
\end{minipage}}

\noindent \fbox{
\begin{minipage}[t]{\textwidth}
\textsf{\scriptsize REQUISITO DE CARGA HORÁRIA:} 
Não há.
\end{minipage}}

\noindent \fbox{
\begin{minipage}[t]{\textwidth}
\textsf{\scriptsize EMENTA:} \\
Introdução ao paradigma orientado a aspectos: Problemas no paradigma OO;
Modularidade; Extensibilidade; Reusabilidade; AspectJ; Aspectos;
Pointcuts; Join points; Advice; Static and dynamic crosscutting;
Discutir detalhes da semântica estática e dinâmica de AspectJ;
Apresentação de exemplos de aplicações do paradigma orientado a
aspectos; Padrões de projetos orientados a aspectos; Padrões GOF
(implementações em Java e AspectJ ); PDC (Persistent Data Collection);
Refactorings orientados a aspectos; AspectJ dioms para construção de
software orientado a aspectos; Catálogo de refactorings e code smells
para AspectJ.
\end{minipage}}

\noindent \fbox{
\begin{minipage}[t]{\textwidth}
\textsf{\scriptsize BIBLIOGRAFIA BÁSICA:}
\begin{enumerate}
\def\labelenumi{\arabic{enumi}.}
\item
  Ladda, Ramnivas. AspectJ in Action: Practical Aspect-Oriented
  programming. Greenwich, CT, USA, 1 ed., 2003.
\item
  Kiselev, I. Aspect-Oriented Programming with AspectJ.
  ISBN:\textasciitilde0672324105. Sams Publisher. 1 ed., 2002.
\item
  Miles, R.; AspectJ Cookbook. ISBN: 0596006543. Ed. O'REILLY \& ASSOC.
  2009.
\end{enumerate}
\end{minipage}}

\noindent \fbox{
\begin{minipage}[t]{\textwidth}
\textsf{\scriptsize BIBLIOGRAFIA COMPLEMENTAR:}
\begin{enumerate}
\def\labelenumi{\arabic{enumi}.}
\item
  Sommerville, Ian. Engenharia de Software. ISBN.: 8588639076. Addison -
  Wesley.
\item
  Roger S. Pressman. Engenharia de Software. ISBN.: 8586804576.
  McGraw-Hill.
\end{enumerate}
\end{minipage}}
\end{scriptsize}

\newpage