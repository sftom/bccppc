\begin{figure*}
    \centering
    \includegraphics[width=2.57cm]{./images/brasao_da_republica.jpeg}
    
    \textsf{Ministério da Educação}\\
    \textsf{Universidade Federal do Agreste de Pernambuco}\\
    \textsf{Bacharelado em Ciência da Computação}
\end{figure*}

\vspace{2cm}

\begin{scriptsize}
\noindent \fbox{
\begin{minipage}[t]{\textwidth}
\textsf{\scriptsize COMPONENTE CURRICULAR:} \\
\textbf{ALGORITMOS E ESTRUTURA DE DADOS II} \\
\textsf{\scriptsize CÓDIGO:} \\
\textbf{CCMP3016}
\end{minipage}}

\noindent \fbox{
\begin{minipage}[t]{0.293\textwidth}
\textsf{\scriptsize PERÍODO A SER OFERTADO:} \\
\textbf{3}
\end{minipage}
%
\begin{minipage}[t]{0.7\textwidth}
\textsf{\scriptsize NÚCLEO DE FORMAÇÃO:} \\
\textbf{CICLO GERAL OU CICLO BÁSICO}
\end{minipage}}

\noindent \fbox{
\begin{minipage}[t]{0.493\textwidth}
\textsf{\scriptsize TIPO:} \\
\textbf{OBRIGATÓRIO}
\end{minipage}
%
\begin{minipage}[t]{0.5\textwidth}
\textsf{\scriptsize CRÉDITOS:} \\
\textbf{4}
\end{minipage}}

\noindent \fbox{
\begin{minipage}[t]{0.3\textwidth}
\textsf{\scriptsize CARGA HORÁRIA TOTAL:} \\
\textbf{60}
\end{minipage}
%
\begin{minipage}[t]{0.19\textwidth}
\textsf{\scriptsize TEÓRICA:} \\
\textbf{60}
\end{minipage}
%
\begin{minipage}[t]{0.19\textwidth}
\textsf{\scriptsize PRÁTICA:} \\
\textbf{0}
\end{minipage}
%
\begin{minipage}[t]{0.30\textwidth}
\textsf{\scriptsize EAD-SEMIPRESENCIAL:} \\
\textbf{0}
\end{minipage}}

\noindent \fbox{
\begin{minipage}[t]{\textwidth}
\textsf{\scriptsize PRÉ-REQUISITOS:}
\begin{itemize}
\item
  CCMP3006 ALGORITMOS E ESTRUTURA DE DADOS I
\item
  CCMP3057 INTRODUÇÃO À PROGRAMAÇÃO
\end{itemize}
\end{minipage}}

\noindent \fbox{
\begin{minipage}[t]{\textwidth}
\textsf{\scriptsize CORREQUISITOS:} 
Não há.
\end{minipage}}

\noindent \fbox{
\begin{minipage}[t]{\textwidth}
\textsf{\scriptsize REQUISITO DE CARGA HORÁRIA:} 
Não há.
\end{minipage}}

\noindent \fbox{
\begin{minipage}[t]{\textwidth}
\textsf{\scriptsize EMENTA:} \\
Formas de representação e abstração de dados em memória. Dispositivos e
técnicas para armazenamento de dados. Estruturas Abstratas de Dados: -
Lista, Pilha, Fila. Métodos de busca e classificação de dados em memória
secundária de computadores Parâmetros físicos e lógicos dos dispositivos
para armazenagem de dados. Métodos de representação e abstração de
dados. Métodos de acesso a arquivos: sequenciais, sequenciais indexadas,
indexadas e diretas. Estruturas árvores-B e hashing, Árvore binária,
Árvore Balanceada, Heap Grafo.
\end{minipage}}

\noindent \fbox{
\begin{minipage}[t]{\textwidth}
\textsf{\scriptsize BIBLIOGRAFIA BÁSICA:}
\begin{enumerate}
\def\labelenumi{\arabic{enumi}.}
\item
  SZWARCFITER, J. L.; MARKENZON, L. Estrutura de dados e seus
  algoritmos. 2. ed.~revista. Rio de Janeiro, RJ: LTC - Livros Técnicos
  e Científicos, 2009.
\item
  CORMEN, T. H., LEISERSON, C. E., RIVEST, R. L., STEIN, C. Algoritmos:
  Teoria e prática. Editora Campus, tradução da 2a edição Americana,
  2002.
\item
  BOAVENTURA NETTO, P. O. Grafos: teoria, modelos, algoritmos. 4.
  ed.~rev. ampl. São Paulo, SP: Edgard Blücher, 2006.
\end{enumerate}
\end{minipage}}

\noindent \fbox{
\begin{minipage}[t]{\textwidth}
\textsf{\scriptsize BIBLIOGRAFIA COMPLEMENTAR:}
\begin{enumerate}
\def\labelenumi{\arabic{enumi}.}
\item
  GOODRICH, M. T.; TAMASSIA, R. Estruturas de dados e algoritmos em
  Java. 4. ed.~Porto Alegre: Bookman, 2007.
\item
  GUIMARÃES, A. de M.; LAGES, N. A. de C. Algoritmos e estruturas de
  dados. Rio de Janeiro: LTC, 2008.
\item
  PAPADIMITRIOU, C. H., VAZIRANI, U. V., DASGUPTA, S. Algoritmos.
  McGraw-Hill, 2006.
\end{enumerate}
\end{minipage}}
\end{scriptsize}

\newpage