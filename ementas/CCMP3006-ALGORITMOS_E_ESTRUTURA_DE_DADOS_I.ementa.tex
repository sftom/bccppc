\begin{figure*}
    \centering
    \includegraphics[width=2.57cm]{./images/brasao_da_republica.jpeg}
    
    \textsf{Ministério da Educação}\\
    \textsf{Universidade Federal do Agreste de Pernambuco}\\
    \textsf{Bacharelado em Ciência da Computação}
\end{figure*}

\vspace{2cm}

\begin{scriptsize}
\noindent \fbox{
\begin{minipage}[t]{\textwidth}
\textsf{\scriptsize COMPONENTE CURRICULAR:} \\
\textbf{ALGORITMOS E ESTRUTURA DE DADOS I} \\
\textsf{\scriptsize CÓDIGO:} \\
\textbf{CCMP3006}
\end{minipage}}

\noindent \fbox{
\begin{minipage}[t]{0.293\textwidth}
\textsf{\scriptsize PERÍODO A SER OFERTADO:} \\
\textbf{2}
\end{minipage}
%
\begin{minipage}[t]{0.7\textwidth}
\textsf{\scriptsize NÚCLEO DE FORMAÇÃO:} \\
\textbf{CICLO GERAL OU CICLO BÁSICO}
\end{minipage}}

\noindent \fbox{
\begin{minipage}[t]{0.493\textwidth}
\textsf{\scriptsize TIPO:} \\
\textbf{OBRIGATÓRIO}
\end{minipage}
%
\begin{minipage}[t]{0.5\textwidth}
\textsf{\scriptsize CRÉDITOS:} \\
\textbf{4}
\end{minipage}}

\noindent \fbox{
\begin{minipage}[t]{0.3\textwidth}
\textsf{\scriptsize CARGA HORÁRIA TOTAL:} \\
\textbf{60}
\end{minipage}
%
\begin{minipage}[t]{0.19\textwidth}
\textsf{\scriptsize TEÓRICA:} \\
\textbf{60}
\end{minipage}
%
\begin{minipage}[t]{0.19\textwidth}
\textsf{\scriptsize PRÁTICA:} \\
\textbf{0}
\end{minipage}
%
\begin{minipage}[t]{0.30\textwidth}
\textsf{\scriptsize EAD-SEMIPRESENCIAL:} \\
\textbf{0}
\end{minipage}}

\noindent \fbox{
\begin{minipage}[t]{\textwidth}
\textsf{\scriptsize PRÉ-REQUISITOS:}
CCMP3057 INTRODUÇÃO À PROGRAMAÇÃO
\end{minipage}}

\noindent \fbox{
\begin{minipage}[t]{\textwidth}
\textsf{\scriptsize CORREQUISITOS:} 
Não há.
\end{minipage}}

\noindent \fbox{
\begin{minipage}[t]{\textwidth}
\textsf{\scriptsize REQUISITO DE CARGA HORÁRIA:} 
Não há.
\end{minipage}}

\noindent \fbox{
\begin{minipage}[t]{\textwidth}
\textsf{\scriptsize EMENTA:} \\
Resolução de problemas e desenvolvimento de algoritmos Conceitos de C:
sintaxe e semântica de comandos de I/O, decisão, repetição e
modularização de programas. Tipo de dado abstrato. Estruturas de dados
seqüencial (vetor) e seus algoritmos. Algoritmos de Ordenação e Busca.
Estruturas de dados elaboradas (lista, fila e pilha) e seus algoritmos.
Análise do problema, estratégias de solução, representação e
documentação.
\end{minipage}}

\noindent \fbox{
\begin{minipage}[t]{\textwidth}
\textsf{\scriptsize BIBLIOGRAFIA BÁSICA:}
\begin{enumerate}
\def\labelenumi{\arabic{enumi}.}
\item
  TENENBAUM, A.; LANGSAM, Y.; AUGENSTEIN, M. J.-Data Structures Using C
  And C++, 2nd Edition, Prentice-Hall, 1996.
\item
  CELES ,W.; CERQUEIRA, R.; RANGEL, J.L.- Introdução a Estrutura de
  Dados Uma Introdução com Tecnicas de Programação em C, Coleção:
  Campus/SBC, Rio de Janeiro: Campus, 2004.
\item
  GUIMARÃES, A. de M.; LAGES, N. A. de C. Algoritmos e Estruturas de
  Dados. Editora LTC, 1994.
\end{enumerate}
\end{minipage}}

\noindent \fbox{
\begin{minipage}[t]{\textwidth}
\textsf{\scriptsize BIBLIOGRAFIA COMPLEMENTAR:}
\begin{enumerate}
\def\labelenumi{\arabic{enumi}.}
\item
  Estruturas de Dados e seus Algoritmos. J. L. Szwarcfiter e L.
  Markenzion. Segunda Edição. LTC, 1994.
\item
  Desenvolvimento de Algoritmos e Estruturas de Dados. R. Terada, Makron
  Books, 1991.
\item
  ZIVIANI, N., Projeto de Algoritmos: com implementação em Pascal e C,
  2a Ed., Editora Thomson Pioneira, 2004.
\end{enumerate}
\end{minipage}}
\end{scriptsize}

\newpage