\begin{figure*}
    \centering
    \includegraphics[width=2.57cm]{./images/brasao_da_republica.jpeg}
    
    \textsf{Ministério da Educação}\\
    \textsf{Universidade Federal do Agreste de Pernambuco}\\
    \textsf{Bacharelado em Ciência da Computação}
\end{figure*}

\vspace{2cm}

\begin{scriptsize}
\noindent \fbox{
\begin{minipage}[t]{\textwidth}
\textsf{\scriptsize COMPONENTE CURRICULAR:} \\
\textbf{SISTEMAS DE INFORMAÇÃO E TECNOLOGIAS} \\
\textsf{\scriptsize CÓDIGO:} \\
\textbf{CCMP3067}
\end{minipage}}

\noindent \fbox{
\begin{minipage}[t]{0.293\textwidth}
\textsf{\scriptsize PERÍODO A SER OFERTADO:} \\
\textbf{5}
\end{minipage}
%
\begin{minipage}[t]{0.7\textwidth}
\textsf{\scriptsize NÚCLEO DE FORMAÇÃO:} \\
\textbf{CICLO GERAL OU CICLO BÁSICO}
\end{minipage}}

\noindent \fbox{
\begin{minipage}[t]{0.493\textwidth}
\textsf{\scriptsize TIPO:} \\
\textbf{OBRIGATÓRIO}
\end{minipage}
%
\begin{minipage}[t]{0.5\textwidth}
\textsf{\scriptsize CRÉDITOS:} \\
\textbf{4}
\end{minipage}}

\noindent \fbox{
\begin{minipage}[t]{0.3\textwidth}
\textsf{\scriptsize CARGA HORÁRIA TOTAL:} \\
\textbf{60}
\end{minipage}
%
\begin{minipage}[t]{0.19\textwidth}
\textsf{\scriptsize TEÓRICA:} \\
\textbf{60}
\end{minipage}
%
\begin{minipage}[t]{0.19\textwidth}
\textsf{\scriptsize PRÁTICA:} \\
\textbf{0}
\end{minipage}
%
\begin{minipage}[t]{0.30\textwidth}
\textsf{\scriptsize EAD-SEMIPRESENCIAL:} \\
\textbf{0}
\end{minipage}}

\noindent \fbox{
\begin{minipage}[t]{\textwidth}
\textsf{\scriptsize PRÉ-REQUISITOS:}
\begin{itemize}
\item
  CCMP3017 PROGRAMAÇÃO ORIENTADA AO OBJETO
\item
  CCMP3018 ENGENHARIA DE SOFTWARE
\item
  CCMP3057 INTRODUÇÃO À PROGRAMAÇÃO
\end{itemize}
\end{minipage}}

\noindent \fbox{
\begin{minipage}[t]{\textwidth}
\textsf{\scriptsize CORREQUISITOS:} 
Não há.
\end{minipage}}

\noindent \fbox{
\begin{minipage}[t]{\textwidth}
\textsf{\scriptsize REQUISITO DE CARGA HORÁRIA:} 
Não há.
\end{minipage}}

\noindent \fbox{
\begin{minipage}[t]{\textwidth}
\textsf{\scriptsize EMENTA:} \\
A organização. Configuração estrutural. A TI na empresa e a Revolução da
Web.Visão sistêmica de estratégias integradoras de áreas e informação
como apoio ao processo decisório. Aplicações organizacionais;
Planejamento. Elementos da Tomada de decisão numa organização. Decisão e
controle. Sistemas de Informação Transacionais, Gerenciais e de Apoio às
Operações e à Decisão. ERPs. CRMs. SCMs. Business intellligence. Gestão
do conhecimento. A importância do planejamento em TI. Tendências em TI
nas organizações.
\end{minipage}}

\noindent \fbox{
\begin{minipage}[t]{\textwidth}
\textsf{\scriptsize BIBLIOGRAFIA BÁSICA:}
\begin{enumerate}
\def\labelenumi{\arabic{enumi}.}
\item
  TURBAN, E.; MCLEAN, E; WETHERBE, J. Tecnologia da Informação para
  Gestão. Transformando os Negócios na Economia Digital. Tradução de
  Renate Schinke. Revisão técnica de ngela F. Brodbeck. Porto Alegre:
  Bookman, 2010.
\item
  O'BRIEN, James. Sistemas de Informação e as decisões gerenciais na era
  da Internet. São Paulo: Editora Saraiva, 2006.
\item
  REZENDE, D. A E; ABREU, A F de . Tecnologia Da Informação Aplicada A
  Sistemas de Informação Empresariais: O Papel Estratégico Da Informação
  E Dos Sistemas de Informação Nas Empresas. Editora Atlas - São Paulo
  -- 2013.
\end{enumerate}
\end{minipage}}

\noindent \fbox{
\begin{minipage}[t]{\textwidth}
\textsf{\scriptsize BIBLIOGRAFIA COMPLEMENTAR:}
\begin{enumerate}
\def\labelenumi{\arabic{enumi}.}
\item
  LAUDON, Kenneth; LAUDON, Jane. Sistemas de Informação Gerenciais. 11
  Ed. Pearson. 2014.
\item
  GORDON, S.R.; GORDON, J.R. Sistema de informação. 3. ed.~Rio de
  Janeiro: LTC -- Livros Técnicos e Científicos Editora S.A., 2006.
\item
  O'BRIEN,James; MARAKAS,George. Administração de Sistemas de Informação
  - 15ª Ed. Amgh Editora. 2013
\item
  KROENKE ,David M. Sistemas de Informação Gerenciais Editora saraiva.
  2013.
\item
  CRUZ, Tadeu. Sistemas de Informações Gerenciais - 4ª Ed. Atlas. 2014
\end{enumerate}
\end{minipage}}
\end{scriptsize}

\newpage