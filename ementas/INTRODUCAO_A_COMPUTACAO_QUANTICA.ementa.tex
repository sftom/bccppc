\begin{figure*}
    \centering
    \includegraphics[width=2.57cm]{./images/brasao_da_republica.jpeg}
    
    \textsf{Ministério da Educação}\\
    \textsf{Universidade Federal do Agreste de Pernambuco}\\
    \textsf{Bacharelado em Ciência da Computação}
\end{figure*}

\vspace{0.5cm}

\begin{scriptsize}
\noindent \fbox{
\begin{minipage}[t]{\textwidth}
\textsf{\scriptsize COMPONENTE CURRICULAR:} \\
\textbf{INTRODUÇÃO À COMPUTAÇÃO QUÂNTICA} \\
\textsf{\scriptsize CÓDIGO:} \textbf{}
\end{minipage}}

\noindent \fbox{
\begin{minipage}[t]{0.293\textwidth}
\textsf{\scriptsize PERÍODO A SER OFERTADO:} \\
\textbf{0}
\end{minipage}
%
\begin{minipage}[t]{0.7\textwidth}
\textsf{\scriptsize NÚCLEO DE FORMAÇÃO:} \\
\textbf{COMPONENTES OPTATIVOS ÁREA TEMÁTICA INTELIGÊNCIA COMPUTACIONAL}
\end{minipage}}

\noindent \fbox{
\begin{minipage}[t]{0.493\textwidth}
\textsf{\scriptsize TIPO:} \textbf{OPTATIVO}
\end{minipage}
%
\begin{minipage}[t]{0.5\textwidth}
\textsf{\scriptsize CRÉDITOS:} \textbf{4}
\end{minipage}}

\noindent \fbox{
\begin{minipage}[t]{0.3\textwidth}
\textsf{\scriptsize CARGA HORÁRIA TOTAL:} 
\textbf{60}
\end{minipage}
%
\begin{minipage}[t]{0.19\textwidth}
\textsf{\scriptsize TEÓRICA:} 
\textbf{45}
\end{minipage}
%
\begin{minipage}[t]{0.19\textwidth}
\textsf{\scriptsize PRÁTICA:} 
\textbf{15}
\end{minipage}
%
\begin{minipage}[t]{0.30\textwidth}
\textsf{\scriptsize EAD-SEMIPRESENCIAL:} 
\textbf{0}
\end{minipage}}

\noindent \fbox{
\begin{minipage}[t]{\textwidth}
\textsf{\scriptsize PRÉ-REQUISITOS:}
Não há.
\end{minipage}}

\noindent \fbox{
\begin{minipage}[t]{\textwidth}
\textsf{\scriptsize CORREQUISITOS:} 
Não há.
\end{minipage}}

\noindent \fbox{
\begin{minipage}[t]{\textwidth}
\textsf{\scriptsize REQUISITO DE CARGA HORÁRIA:} 
Não há.
\end{minipage}}

\noindent \fbox{
\begin{minipage}[t]{\textwidth}
\textsf{\scriptsize EMENTA:} \\
Espaços de Hilbert sobre corpo Complexo. Elementos da Teoria da
Computação Clássica contendo Circuitos Booleanos. Elementos da Teoria
Quântica. Elementos da Computação quântica: modelos teóricos, portas
lógicas quânticas. Algoritmos quânticos do tipo Oráculo (Deutsch-Josza,
Grover). Algorítmos quânticos do tipo Transformada de Fourier (Simon,
Shor). Simuladores e Linguagens de Programação Quânticas. Noções de
complexidade de computação: Classe NP, Algoritmos Probabilísticos e a
Classe BPP.
\end{minipage}}

\noindent \fbox{
\begin{minipage}[t]{\textwidth}
\textsf{\scriptsize BIBLIOGRAFIA BÁSICA:}
\begin{enumerate}
\def\labelenumi{\arabic{enumi}.}
\item
  Noson S. Yanofsky; Mirco A. Mannucci: Quantum Computing for Computer
  Scientists. Cambridge University Press, 2008, ISBN 978-0-521-87996-5.
\item
  David McMahon: Quantum Computing Explained. Wiley-Interscience,
  Hoboken, New Jersey, USA, 2008, ISBN 978-0-470-09699-4.
\item
  N. David Mermin: Quantum Computer Science An Introduc-tion. Cambridge
  University Press, New York, USA, 2007, ISBN 978-0-521-87658-2.
\end{enumerate}
\end{minipage}}

\noindent \fbox{
\begin{minipage}[t]{\textwidth}
\textsf{\scriptsize BIBLIOGRAFIA COMPLEMENTAR:}
\begin{enumerate}
\def\labelenumi{\arabic{enumi}.}
\item
  Alexei Yu. Kitaev, Alexander H. Shen e Mikhail N. Vyalyi: Classical
  and Quantum Computation. Graduate Studies in Mathematics, vol 47, AMS,
  2002. ISBN 0-8218-3229-8.
\item
  Michael A. Nielsen e Isaac L. Chuang: Computação Quântica e Informação
  Quântica. 1a. Edição, Editora Bookman, 2005, ISBN 8536305541.
\item
  Artigos recentes da área.
\end{enumerate}
\end{minipage}}
\end{scriptsize}

\newpage