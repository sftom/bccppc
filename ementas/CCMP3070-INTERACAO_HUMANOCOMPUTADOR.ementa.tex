\begin{figure*}
    \centering
    \includegraphics[width=2.57cm]{./images/brasao_da_republica.jpeg}
    
    \textsf{Ministério da Educação}\\
    \textsf{Universidade Federal do Agreste de Pernambuco}\\
    \textsf{Bacharelado em Ciência da Computação}
\end{figure*}

\vspace{2cm}

\begin{scriptsize}
\noindent \fbox{
\begin{minipage}[t]{\textwidth}
\textsf{\scriptsize COMPONENTE CURRICULAR:} \\
\textbf{INTERAÇÃO HUMANO-COMPUTADOR} \\
\textsf{\scriptsize CÓDIGO:} \\
\textbf{CCMP3070}
\end{minipage}}

\noindent \fbox{
\begin{minipage}[t]{0.293\textwidth}
\textsf{\scriptsize PERÍODO A SER OFERTADO:} \\
\textbf{7}
\end{minipage}
%
\begin{minipage}[t]{0.7\textwidth}
\textsf{\scriptsize NÚCLEO DE FORMAÇÃO:} \\
\textbf{CICLO PROFISSIONAL OU TRONCO COMUM}
\end{minipage}}

\noindent \fbox{
\begin{minipage}[t]{0.493\textwidth}
\textsf{\scriptsize TIPO:} \\
\textbf{OBRIGATÓRIO}
\end{minipage}
%
\begin{minipage}[t]{0.5\textwidth}
\textsf{\scriptsize CRÉDITOS:} \\
\textbf{4}
\end{minipage}}

\noindent \fbox{
\begin{minipage}[t]{0.3\textwidth}
\textsf{\scriptsize CARGA HORÁRIA TOTAL:} \\
\textbf{60}
\end{minipage}
%
\begin{minipage}[t]{0.19\textwidth}
\textsf{\scriptsize TEÓRICA:} \\
\textbf{30}
\end{minipage}
%
\begin{minipage}[t]{0.19\textwidth}
\textsf{\scriptsize PRÁTICA:} \\
\textbf{30}
\end{minipage}
%
\begin{minipage}[t]{0.30\textwidth}
\textsf{\scriptsize EAD-SEMIPRESENCIAL:} \\
\textbf{0}
\end{minipage}}

\noindent \fbox{
\begin{minipage}[t]{\textwidth}
\textsf{\scriptsize PRÉ-REQUISITOS:}
\begin{itemize}
\item
  CCMP3017 PROGRAMAÇÃO ORIENTADA AO OBJETO
\item
  CCMP3018 ENGENHARIA DE SOFTWARE
\item
  CCMP3057 INTRODUÇÃO À PROGRAMAÇÃO
\end{itemize}
\end{minipage}}

\noindent \fbox{
\begin{minipage}[t]{\textwidth}
\textsf{\scriptsize CORREQUISITOS:} 
Não há.
\end{minipage}}

\noindent \fbox{
\begin{minipage}[t]{\textwidth}
\textsf{\scriptsize REQUISITO DE CARGA HORÁRIA:} 
Não há.
\end{minipage}}

\noindent \fbox{
\begin{minipage}[t]{\textwidth}
\textsf{\scriptsize EMENTA:} \\
Conhecer os fundamentos de fatores humanos em IHC, os modelos mentais
(metáforas), os paradigmas de IHC (engenharia semiótica e cognitiva) e
os métodos, técnicas, suporte e avaliação de design de interação.
\end{minipage}}

\noindent \fbox{
\begin{minipage}[t]{\textwidth}
\textsf{\scriptsize BIBLIOGRAFIA BÁSICA:}
\begin{enumerate}
\def\labelenumi{\arabic{enumi}.}
\item
  Barbosa, S. D. J.; Silva, B. S. Interação Humano-Computador. Rio de
  Janeiro: Elservier, 2010.
\item
  Lidwell, W.; Holden, K.; Butler, J. Universal principles of design,
  revised and updated: 125 ways to enhance usability, influence
  perception, increase appeal, make better design decisions, and teach
  through design. Rockport Pub, 2010.
\item
  Heloísa Rocha, Maria Baranauskas. Design e Avaliação de Interfaces
  Humano-Computador. IC, UNICAMP, 2003.
\end{enumerate}
\end{minipage}}

\noindent \fbox{
\begin{minipage}[t]{\textwidth}
\textsf{\scriptsize BIBLIOGRAFIA COMPLEMENTAR:}
\begin{enumerate}
\def\labelenumi{\arabic{enumi}.}
\item
  Norman, D. The design of everyday things: Revised and expanded
  edition. Basic books, 2013.
\item
  Andrew Sears, Julie Jacko. Human-Computer Interaction Fundamentals.
  CRC Press, 2009.
\item
  Bootcamp Bootleg. Disponível em:
  https://dschool.stanford.edu/resources/the-bootcamp-bootleg. Acesso
  em: 22 Jul.~2022.
\item
  Nielsen Norman Group. Disponível em: https://www.nngroup.com/. Acesso
  em: 5 Ago. 2022.
\item
  Julie A. Jacko. Human Computer Interaction Handbook: Fundamentals,
  Evolving Technologies, and Emerging Applications, Third Edition (Human
  Factors and Ergonomics). 3rd Edition. ISBN-10: 1439829438
\end{enumerate}
\end{minipage}}
\end{scriptsize}

\newpage