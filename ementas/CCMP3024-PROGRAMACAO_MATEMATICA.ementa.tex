\begin{figure*}
    \centering
    \includegraphics[width=2.57cm]{./images/brasao_da_republica.jpeg}
    
    \textsf{Ministério da Educação}\\
    \textsf{Universidade Federal do Agreste de Pernambuco}\\
    \textsf{Bacharelado em Ciência da Computação}
\end{figure*}

\vspace{2cm}

\begin{scriptsize}
\noindent \fbox{
\begin{minipage}[t]{\textwidth}
\textsf{\scriptsize COMPONENTE CURRICULAR:} \\
\textbf{PROGRAMAÇÃO MATEMÁTICA} \\
\textsf{\scriptsize CÓDIGO:} \\
\textbf{CCMP3024}
\end{minipage}}

\noindent \fbox{
\begin{minipage}[t]{0.293\textwidth}
\textsf{\scriptsize PERÍODO A SER OFERTADO:} \\
\textbf{0}
\end{minipage}
%
\begin{minipage}[t]{0.7\textwidth}
\textsf{\scriptsize NÚCLEO DE FORMAÇÃO:} \\
\textbf{COMPONENTES OPTATIVOS ÁREA TEMÁTICA MATEMÁTICA E SIMULAÇÃO
COMPUTACIONAL}
\end{minipage}}

\noindent \fbox{
\begin{minipage}[t]{0.493\textwidth}
\textsf{\scriptsize TIPO:} \\
\textbf{OPTATIVO}
\end{minipage}
%
\begin{minipage}[t]{0.5\textwidth}
\textsf{\scriptsize CRÉDITOS:} \\
\textbf{4}
\end{minipage}}

\noindent \fbox{
\begin{minipage}[t]{0.3\textwidth}
\textsf{\scriptsize CARGA HORÁRIA TOTAL:} \\
\textbf{60}
\end{minipage}
%
\begin{minipage}[t]{0.19\textwidth}
\textsf{\scriptsize TEÓRICA:} \\
\textbf{60}
\end{minipage}
%
\begin{minipage}[t]{0.19\textwidth}
\textsf{\scriptsize PRÁTICA:} \\
\textbf{0}
\end{minipage}
%
\begin{minipage}[t]{0.30\textwidth}
\textsf{\scriptsize EAD-SEMIPRESENCIAL:} \\
\textbf{0}
\end{minipage}}

\noindent \fbox{
\begin{minipage}[t]{\textwidth}
\textsf{\scriptsize PRÉ-REQUISITOS:}
\begin{itemize}
\item
  MATM3019 ÁLGEBRA LINEAR I
\item
  MATM3021 GEOMETRIA ANALÍTICA A
\end{itemize}
\end{minipage}}

\noindent \fbox{
\begin{minipage}[t]{\textwidth}
\textsf{\scriptsize CORREQUISITOS:} 
Não há.
\end{minipage}}

\noindent \fbox{
\begin{minipage}[t]{\textwidth}
\textsf{\scriptsize REQUISITO DE CARGA HORÁRIA:} 
Não há.
\end{minipage}}

\noindent \fbox{
\begin{minipage}[t]{\textwidth}
\textsf{\scriptsize EMENTA:} \\
Modelagem sistêmica de problemas industriais. Modelos de Programação
Linear Inteira Mista (PLIM) para apoio à tomada de decisão. Programação
Linear (PL). Método primal simplex. Problema de transporte. Problema de
designação. Dualidade. Método dual simplex. Análise de sensibilidade.
Interpretação econômica da PL. Programação inteira. Programação inteira
mista. Resolução de problemas de grande porte. Decomposição em PL e
PLIM. Aplicações em sistemas produtivos.
\end{minipage}}

\noindent \fbox{
\begin{minipage}[t]{\textwidth}
\textsf{\scriptsize BIBLIOGRAFIA BÁSICA:}
\begin{enumerate}
\def\labelenumi{\arabic{enumi}.}
\item
  J. F. Benders. Partitioning procedures for solving mixed-variables
  programming problems. Numerisch Mathematik, v. 4, p.~238-252, 1962.
\item
  G. B. Dantzig and Wolfe. Decomposition principle for linear programs.
  Operations Research, v. 8, p.~101-111, 1960.
\item
  C. R. V. de Carvalho. Une Proposition d'Integration de la
  Planification et l'Ordonancement de Production: Application de la
  Métode de Benders. PhD thesis, Université Blaise Pascal,
  Clermont-Ferrand, França, 1998.
\end{enumerate}
\end{minipage}}

\noindent \fbox{
\begin{minipage}[t]{\textwidth}
\textsf{\scriptsize BIBLIOGRAFIA COMPLEMENTAR:}
\begin{enumerate}
\def\labelenumi{\arabic{enumi}.}
\item
  M. T. P. de Carvalho. Confecção de horários de aulas em instituições
  de ensino privadas. Master's thesis, Programa de Pós-Graduação em
  Engenharia de Produção da UFMG, 2002.
\item
  C. R. V. de Carvalho. Notas de Aula.
\item
  M. C. Goldbarg and H. P. L. Luna. Otimização Combinatória e
  Programação Linear Modelos e Algoritmos. Ed. Campus, 2000.
\item
  F. S. Hiller and G. J. Liberman. Introdução à Pesquisa Operacional.
  Ed. Campus Ltda, Rio de Janeiro, 1989.
\item
  L. S. Lasdon. Optimization Theory for Large Systems. The Macmillan
  Company, New York, 1972.
\item
  T. R. Neto. Uma metodologia para elaboração de planos de compras de
  carvão em empresas siderúrgicas brasileiras. Dissertação de Mestrado.
  Programa de Pós-Graduação em Engenharia de Produção da UFMG, 2003.
\item
  C. R. Oliveira. Planejamento da distribuição de produtos siderúrgicos
  utilizando modelos de localização. Dissertação de Mestrado. Programa
  de Pós-Graduação em Engenharia de Produção da UFMG, 2003.
\item
  H. M. Wagner. Pesquisa Operacional. Prentice-Hall do Brasil, Rio de
  Janeiro, 1986.
\item
  -. R. Baker. Introduction to Sequencing and Scheduling. Wiley, 1974.
\item
  M. S. Bazaraa and J. J. Jarvis. Linear Programming and Network Flows.
  John Wiley \& Sons, New York, 1977.
\end{enumerate}
\end{minipage}}
\end{scriptsize}

\newpage