\begin{figure*}
    \centering
    \includegraphics[width=2.57cm]{./images/brasao_da_republica.jpeg}
    
    \textsf{Ministério da Educação}\\
    \textsf{Universidade Federal do Agreste de Pernambuco}\\
    \textsf{Bacharelado em Ciência da Computação}
\end{figure*}

\vspace{0.5cm}

\begin{scriptsize}
\noindent \fbox{
\begin{minipage}[t]{\textwidth}
\textsf{\scriptsize COMPONENTE CURRICULAR:} \\
\textbf{REDES NEURAIS} \\
\textsf{\scriptsize CÓDIGO:} \textbf{}
\end{minipage}}

\noindent \fbox{
\begin{minipage}[t]{0.293\textwidth}
\textsf{\scriptsize PERÍODO A SER OFERTADO:} \\
\textbf{0}
\end{minipage}
%
\begin{minipage}[t]{0.7\textwidth}
\textsf{\scriptsize NÚCLEO DE FORMAÇÃO:} \\
\textbf{COMPONENTES OPTATIVOS ÁREA TEMÁTICA INTELIGÊNCIA COMPUTACIONAL}
\end{minipage}}

\noindent \fbox{
\begin{minipage}[t]{0.493\textwidth}
\textsf{\scriptsize TIPO:} \textbf{OPTATIVO}
\end{minipage}
%
\begin{minipage}[t]{0.5\textwidth}
\textsf{\scriptsize CRÉDITOS:} \textbf{4}
\end{minipage}}

\noindent \fbox{
\begin{minipage}[t]{0.3\textwidth}
\textsf{\scriptsize CARGA HORÁRIA TOTAL:} 
\textbf{60}
\end{minipage}
%
\begin{minipage}[t]{0.19\textwidth}
\textsf{\scriptsize TEÓRICA:} 
\textbf{60}
\end{minipage}
%
\begin{minipage}[t]{0.19\textwidth}
\textsf{\scriptsize PRÁTICA:} 
\textbf{0}
\end{minipage}
%
\begin{minipage}[t]{0.30\textwidth}
\textsf{\scriptsize EAD-SEMIPRESENCIAL:} 
\textbf{}
\end{minipage}}

\noindent \fbox{
\begin{minipage}[t]{\textwidth}
\textsf{\scriptsize PRÉ-REQUISITOS:}
RECONHECIMENTO DE PADRÕES
\end{minipage}}

\noindent \fbox{
\begin{minipage}[t]{\textwidth}
\textsf{\scriptsize CORREQUISITOS:} 
Não há.
\end{minipage}}

\noindent \fbox{
\begin{minipage}[t]{\textwidth}
\textsf{\scriptsize REQUISITO DE CARGA HORÁRIA:} 
Não há.
\end{minipage}}

\noindent \fbox{
\begin{minipage}[t]{\textwidth}
\textsf{\scriptsize EMENTA:} \\
Processos de aprendizagem de máquina. Perceptrons de camada única.
Perceptrons de múltiplas camadas. Redes de função de base radial. Mapas
auto-organizáveis. Tópicos avançados em Redes Neurais: máquinas de vetor
de suporte, análise de componentes principais, outros tópicos.
\end{minipage}}

\noindent \fbox{
\begin{minipage}[t]{\textwidth}
\textsf{\scriptsize BIBLIOGRAFIA BÁSICA:}
\begin{enumerate}
\def\labelenumi{\arabic{enumi}.}
\item
  BRAGA, A. P., LUDEMIR, T. B., CARVALHO, A. P. L. F. A Redes Neurais
  Artificiais Teoria e Aplicações. LTC, 2007.
\item
  HAYKIN, Simon. Redes Neurais: Princípios e Prática, 3. edição.
  Bookman. 2008.
\item
  COPPIN, B. Inteligência Artificial. LTC, 2010.
\end{enumerate}
\end{minipage}}

\noindent \fbox{
\begin{minipage}[t]{\textwidth}
\textsf{\scriptsize BIBLIOGRAFIA COMPLEMENTAR:}
\begin{enumerate}
\def\labelenumi{\arabic{enumi}.}
\item
  MARQUES, J. S.. Reconhecimento de Padrões Métodos Estatísticos e
  Neurais. IST Press, 2005.
\item
  WITTEN, I. H., FRANK, E. Data mining : practical machine learning
  tools and techniques. Second Edition. Elsevier, 2005.
\item
  THEODORIDIS, S., KOUTROUMBAS, K. Pattern Recognition. Fourth Edition,
  Academic Press, 2009.
\item
  DUDA, R.O., HART, P.E., STORK, D.G. Pattern Classification. Second
  Edition. Wiley, 2001.
\item
  BISHOP, C. M. Pattern Recognition and Machine Learning. Springer,
  2006.
\end{enumerate}
\end{minipage}}
\end{scriptsize}

\newpage