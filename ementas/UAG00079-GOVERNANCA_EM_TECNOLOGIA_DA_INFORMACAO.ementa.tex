\begin{figure*}
    \centering
    \includegraphics[width=2.57cm]{./images/brasao_da_republica.jpeg}
    
    \textsf{Ministério da Educação}\\
    \textsf{Universidade Federal do Agreste de Pernambuco}\\
    \textsf{Bacharelado em Ciência da Computação}
\end{figure*}

\vspace{2cm}

\begin{scriptsize}
\noindent \fbox{
\begin{minipage}[t]{\textwidth}
\textsf{\scriptsize COMPONENTE CURRICULAR:} \\
\textbf{GOVERNANÇA EM TECNOLOGIA DA INFORMAÇÃO} \\
\textsf{\scriptsize CÓDIGO:} \\
\textbf{UAG00079}
\end{minipage}}

\noindent \fbox{
\begin{minipage}[t]{0.293\textwidth}
\textsf{\scriptsize PERÍODO A SER OFERTADO:} \\
\textbf{0}
\end{minipage}
%
\begin{minipage}[t]{0.7\textwidth}
\textsf{\scriptsize NÚCLEO DE FORMAÇÃO:} \\
\textbf{COMPONENTES OPTATIVOS ÁREA TEMÁTICA TECNOLOGIAS DA INFORMAÇÃO}
\end{minipage}}

\noindent \fbox{
\begin{minipage}[t]{0.493\textwidth}
\textsf{\scriptsize TIPO:} \\
\textbf{OPTATIVO}
\end{minipage}
%
\begin{minipage}[t]{0.5\textwidth}
\textsf{\scriptsize CRÉDITOS:} \\
\textbf{4}
\end{minipage}}

\noindent \fbox{
\begin{minipage}[t]{0.3\textwidth}
\textsf{\scriptsize CARGA HORÁRIA TOTAL:} \\
\textbf{60}
\end{minipage}
%
\begin{minipage}[t]{0.19\textwidth}
\textsf{\scriptsize TEÓRICA:} \\
\textbf{60}
\end{minipage}
%
\begin{minipage}[t]{0.19\textwidth}
\textsf{\scriptsize PRÁTICA:} \\
\textbf{0}
\end{minipage}
%
\begin{minipage}[t]{0.30\textwidth}
\textsf{\scriptsize EAD-SEMIPRESENCIAL:} \\
\textbf{0}
\end{minipage}}

\noindent \fbox{
\begin{minipage}[t]{\textwidth}
\textsf{\scriptsize PRÉ-REQUISITOS:}
CCMP3067 SISTEMAS DE INFORMAÇÃO E TECNOLOGIAS
\end{minipage}}

\noindent \fbox{
\begin{minipage}[t]{\textwidth}
\textsf{\scriptsize CORREQUISITOS:} 
Não há.
\end{minipage}}

\noindent \fbox{
\begin{minipage}[t]{\textwidth}
\textsf{\scriptsize REQUISITO DE CARGA HORÁRIA:} 
Não há.
\end{minipage}}

\noindent \fbox{
\begin{minipage}[t]{\textwidth}
\textsf{\scriptsize EMENTA:} \\
Planejamento Estratégico; Alinhamento estratégico; Decisões Estratégicas
de TI; Governança Corporativa e Governança de TI. Arquétipos de TI para
alocação de direitos decisórios; Mecanismos para implantar a Governança
de TI. Tipos de governança. Associação da Estratégia, da Governança e o
Desempenho. Princípios de Liderança para governança de TI. Normas,
processos e indicadores de desempenho para área de TI. Modelos de apoio
para Governança de TI: COBIT (Control Objectives for Information and
Related Tecnhonology); ITIL (Informática Tecnology Infrasture Library);
BSC (Balanced Socorecard); Estruturação de um plano de implantação de um
modelo de governança de TI; A norma ISO 20000.
\end{minipage}}

\noindent \fbox{
\begin{minipage}[t]{\textwidth}
\textsf{\scriptsize BIBLIOGRAFIA BÁSICA:}
\begin{enumerate}
\def\labelenumi{\arabic{enumi}.}
\item
  WEILL, Peter; ROSS, Jeanne, W. Governança de TI: Tecnologia da
  Informação. São Paulo: Makron Books, 2006.
\item
  FERNANDES, Aguinaldo Aragon; ABREU, Vladimir Ferraz. Implantando a
  Governança de TI -- da Estratégia à Gestão dos Processos e Serviços.
  Rio de Janeiro: Brasport, 2012.
\item
  ISACA. COBIT 5.0 - Controls Objectives for Information and related
  Technology. Disponível em:
  http://www.isaca.org/COBIT/Pages/COBIT-5-portuguese.aspx. Acesso em:
  12 de junho de 2017.
\end{enumerate}
\end{minipage}}

\noindent \fbox{
\begin{minipage}[t]{\textwidth}
\textsf{\scriptsize BIBLIOGRAFIA COMPLEMENTAR:}
\begin{enumerate}
\def\labelenumi{\arabic{enumi}.}
\item
  MAGALHÃES, Ivan Luizio; PINHEIRO, Walfrido Brito. Gerenciamento de
  Serviços de TI na Prática - Uma abordagem com base na ITIL. São Paulo:
  Novatec, 2007.
\item
  CASSARRO, A. C.; Sistemas de informações para tomadas de decisões. 4ª
  Edição. Editora Cengage Learning, 2010.
\item
  MARQUES, A. S.; MARQUES, E. V. JOÃO, B. Sistemas de informação
  gerenciais: administrando a empresa digital. 5. ed.~São Paulo: Pearson
  Prentice Hall, 2004.
\item
  MOLINARO, L. F. R., RAMOS, K. H. C.; Gestão de tecnologia da
  informação: governança de TI: arquitetura e alinhamento entre sistemas
  de informação e o negócio. Editora LTC, 2011.
\end{enumerate}
\end{minipage}}
\end{scriptsize}

\newpage