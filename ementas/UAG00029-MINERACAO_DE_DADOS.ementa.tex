\begin{figure*}
    \centering
    \includegraphics[width=2.57cm]{./images/brasao_da_republica.jpeg}
    
    \textsf{Ministério da Educação}\\
    \textsf{Universidade Federal do Agreste de Pernambuco}\\
    \textsf{Bacharelado em Ciência da Computação}
\end{figure*}

\vspace{2cm}

\begin{scriptsize}
\noindent \fbox{
\begin{minipage}[t]{\textwidth}
\textsf{\scriptsize COMPONENTE CURRICULAR:} \\
\textbf{MINERAÇÃO DE DADOS} \\
\textsf{\scriptsize CÓDIGO:} \\
\textbf{UAG00029}
\end{minipage}}

\noindent \fbox{
\begin{minipage}[t]{0.293\textwidth}
\textsf{\scriptsize PERÍODO A SER OFERTADO:} \\
\textbf{0}
\end{minipage}
%
\begin{minipage}[t]{0.7\textwidth}
\textsf{\scriptsize NÚCLEO DE FORMAÇÃO:} \\
\textbf{COMPONENTES OPTATIVOS ÁREA TEMÁTICA INTELIGÊNCIA COMPUTACIONAL}
\end{minipage}}

\noindent \fbox{
\begin{minipage}[t]{0.493\textwidth}
\textsf{\scriptsize TIPO:} \\
\textbf{OPTATIVO}
\end{minipage}
%
\begin{minipage}[t]{0.5\textwidth}
\textsf{\scriptsize CRÉDITOS:} \\
\textbf{4}
\end{minipage}}

\noindent \fbox{
\begin{minipage}[t]{0.3\textwidth}
\textsf{\scriptsize CARGA HORÁRIA TOTAL:} \\
\textbf{60}
\end{minipage}
%
\begin{minipage}[t]{0.19\textwidth}
\textsf{\scriptsize TEÓRICA:} \\
\textbf{60}
\end{minipage}
%
\begin{minipage}[t]{0.19\textwidth}
\textsf{\scriptsize PRÁTICA:} \\
\textbf{0}
\end{minipage}
%
\begin{minipage}[t]{0.30\textwidth}
\textsf{\scriptsize EAD-SEMIPRESENCIAL:} \\
\textbf{0}
\end{minipage}}

\noindent \fbox{
\begin{minipage}[t]{\textwidth}
\textsf{\scriptsize PRÉ-REQUISITOS:}
\begin{itemize}
\item
  CCMP3014 INTELIGÊNCIA ARTIFICIAL
\item
  CCMP3043 RECONHECIMENTO DE PADRÕES
\end{itemize}
\end{minipage}}

\noindent \fbox{
\begin{minipage}[t]{\textwidth}
\textsf{\scriptsize CORREQUISITOS:} 
Não há.
\end{minipage}}

\noindent \fbox{
\begin{minipage}[t]{\textwidth}
\textsf{\scriptsize REQUISITO DE CARGA HORÁRIA:} 
Não há.
\end{minipage}}

\noindent \fbox{
\begin{minipage}[t]{\textwidth}
\textsf{\scriptsize EMENTA:} \\
Introdução e Motivação ao Processo de Descoberta de Conhecimento em
Bases de Dados (KDD). Etapas do Processo de KDD. Conceitos e Tecnologias
de Suporte à Mineração de Dados. Pré-processamento dos Dados. Extração
de Padrões: Tarefas, Algoritmos e Paradigmas de Mineração de Dados.
Pós-processamento de Resultados. Métricas de Avaliação: Complexidade,
Eficiência e Escalabilidade. Tópicos Avançados: Metaheurísticas,
Paralelismo e Distribuição, Visualização, Privacidade e Segurança,
Representações e Estruturas de Dados Nãoconvencionais,Mineração
Multimodal (Textos e Multimídia),Mineração de Dados Espaciais e
Temporais. Técnicas, Ferramentas e Aplicações.
\end{minipage}}

\noindent \fbox{
\begin{minipage}[t]{\textwidth}
\textsf{\scriptsize BIBLIOGRAFIA BÁSICA:}
\begin{enumerate}
\def\labelenumi{\arabic{enumi}.}
\item
  WITTEN, I. H.; FRANK, E. Data Mining: Practical Machine Learning Tools
  and Techniques. Morgan Kaufmann, 2011.
\item
  TAN, P. N.; STEINBACH, M.; KUMAR, V. Introduction to Data Mining.
  Pearson, 2005.
\item
  HAN, J.; KAMBER, M. Data Mining: Concepts and Techniques. Morgan
  Kaufmann, 2011.
\end{enumerate}
\end{minipage}}

\noindent \fbox{
\begin{minipage}[t]{\textwidth}
\textsf{\scriptsize BIBLIOGRAFIA COMPLEMENTAR:}
\begin{enumerate}
\def\labelenumi{\arabic{enumi}.}
\item
  HAND D.; MANNILA, H.; SMITH, P. Principles of Data Mining. Bradford
  Book, 2001.
\item
  MAIMON, O.; ROKACH, L. Data Mining and Knowledge Discovery Handbook.
  Springer, 2010.
\item
  LAROSE, D. T. Discovering Knowledge in Data: An Introduction to Data
  Mining. John Wiley, 2005.
\end{enumerate}
\end{minipage}}
\end{scriptsize}

\newpage