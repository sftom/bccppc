\begin{figure*}
    \centering
    \includegraphics[width=2.57cm]{./images/brasao_da_republica.jpeg}
    
    \textsf{Ministério da Educação}\\
    \textsf{Universidade Federal do Agreste de Pernambuco}\\
    \textsf{Bacharelado em Ciência da Computação}
\end{figure*}

\vspace{2cm}

\begin{scriptsize}
\noindent \fbox{
\begin{minipage}[t]{\textwidth}
\textsf{\scriptsize COMPONENTE CURRICULAR:} \\
\textbf{REDE DE COMPUTADORES} \\
\textsf{\scriptsize CÓDIGO:} \\
\textbf{CCMP3023}
\end{minipage}}

\noindent \fbox{
\begin{minipage}[t]{0.293\textwidth}
\textsf{\scriptsize PERÍODO A SER OFERTADO:} \\
\textbf{5}
\end{minipage}
%
\begin{minipage}[t]{0.7\textwidth}
\textsf{\scriptsize NÚCLEO DE FORMAÇÃO:} \\
\textbf{CICLO PROFISSIONAL OU TRONCO COMUM}
\end{minipage}}

\noindent \fbox{
\begin{minipage}[t]{0.493\textwidth}
\textsf{\scriptsize TIPO:} \\
\textbf{OBRIGATÓRIO}
\end{minipage}
%
\begin{minipage}[t]{0.5\textwidth}
\textsf{\scriptsize CRÉDITOS:} \\
\textbf{4}
\end{minipage}}

\noindent \fbox{
\begin{minipage}[t]{0.3\textwidth}
\textsf{\scriptsize CARGA HORÁRIA TOTAL:} \\
\textbf{60}
\end{minipage}
%
\begin{minipage}[t]{0.19\textwidth}
\textsf{\scriptsize TEÓRICA:} \\
\textbf{60}
\end{minipage}
%
\begin{minipage}[t]{0.19\textwidth}
\textsf{\scriptsize PRÁTICA:} \\
\textbf{0}
\end{minipage}
%
\begin{minipage}[t]{0.30\textwidth}
\textsf{\scriptsize EAD-SEMIPRESENCIAL:} \\
\textbf{0}
\end{minipage}}

\noindent \fbox{
\begin{minipage}[t]{\textwidth}
\textsf{\scriptsize PRÉ-REQUISITOS:}
\begin{itemize}
\item
  CCMP3006 ALGORITMOS E ESTRUTURA DE DADOS I
\item
  CCMP3016 ALGORITMOS E ESTRUTURA DE DADOS II
\item
  CCMP3056 INTRODUÇÃO À COMPUTAÇÃO C
\item
  CCMP3057 INTRODUÇÃO À PROGRAMAÇÃO
\end{itemize}
\end{minipage}}

\noindent \fbox{
\begin{minipage}[t]{\textwidth}
\textsf{\scriptsize CORREQUISITOS:} 
Não há.
\end{minipage}}

\noindent \fbox{
\begin{minipage}[t]{\textwidth}
\textsf{\scriptsize REQUISITO DE CARGA HORÁRIA:} 
Não há.
\end{minipage}}

\noindent \fbox{
\begin{minipage}[t]{\textwidth}
\textsf{\scriptsize EMENTA:} \\
Conceitos básicos de redes de computadores: definições; terminologia;
classificação; topologias; modelos de arquitetura e aplicações.
Protocolos e modelos de referência: o modelo ISO/OSI e o modelo TCP/IP;
conceitos básicos de cada camada; protocolos das camadas de Rede, de
Transporte e de Aplicação. Conceitos de segurança. Conceitos de
gerenciamento. Conceitos de avaliação de desempenho.
\end{minipage}}

\noindent \fbox{
\begin{minipage}[t]{\textwidth}
\textsf{\scriptsize BIBLIOGRAFIA BÁSICA:}
\begin{enumerate}
\def\labelenumi{\arabic{enumi}.}
\item
  Kusore, James.; Ross, Keith w. Redes de Computadores e a Internet: uma
  abordagem top-down. 3ª Edição. São Paulo: Peason Addison Wesleey,
  2008.
\item
  Tenenbaum, Andrew S. Redes de Computadores. 4ª Edição. Rio de
  Jeaneiro: Editora Campus, 2002.
\item
  Comer, Douglas E. Interligação de Redes com TCP/IP, Volumes I e II. 5ª
  Edição. Prentice Hall, 2006.
\end{enumerate}
\end{minipage}}

\noindent \fbox{
\begin{minipage}[t]{\textwidth}
\textsf{\scriptsize BIBLIOGRAFIA COMPLEMENTAR:}
\begin{enumerate}
\def\labelenumi{\arabic{enumi}.}
\item
  SOARES, LEMOS e COLCHER -- Redes Locais -- Das LANs, MANs e WANs às
  redes ATM. 2° Edição. Ed. Campus, 1995.
\item
  Anderson, Al; Benedetti, Ryan. Redes de Computadores - Use a Cabeça!
  Alta Books.
\item
  Torres, Gabriel. Redes De Computadores. Novaterra.
\item
  Cardoso, Fernanda Caetano. Servidores De Internet Embarcada. Ciência
  Moderna.
\item
  Gast, Matthew S. 802.11 Wireless Networks - The Definitive Guide.
  Oreilly \& Assoc
\end{enumerate}
\end{minipage}}
\end{scriptsize}

\newpage