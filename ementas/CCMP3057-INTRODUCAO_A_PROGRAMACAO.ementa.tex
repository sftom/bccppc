\begin{figure*}
    \centering
    \includegraphics[width=2.57cm]{./images/brasao_da_republica.jpeg}
    
    \textsf{Ministério da Educação}\\
    \textsf{Universidade Federal do Agreste de Pernambuco}\\
    \textsf{Bacharelado em Ciência da Computação}
\end{figure*}

\vspace{2cm}

\begin{scriptsize}
\noindent \fbox{
\begin{minipage}[t]{\textwidth}
\textsf{\scriptsize COMPONENTE CURRICULAR:} \\
\textbf{INTRODUÇÃO À PROGRAMAÇÃO} \\
\textsf{\scriptsize CÓDIGO:} \\
\textbf{CCMP3057}
\end{minipage}}

\noindent \fbox{
\begin{minipage}[t]{0.293\textwidth}
\textsf{\scriptsize PERÍODO A SER OFERTADO:} \\
\textbf{1}
\end{minipage}
%
\begin{minipage}[t]{0.7\textwidth}
\textsf{\scriptsize NÚCLEO DE FORMAÇÃO:} \\
\textbf{CICLO GERAL OU CICLO BÁSICO}
\end{minipage}}

\noindent \fbox{
\begin{minipage}[t]{0.493\textwidth}
\textsf{\scriptsize TIPO:} \\
\textbf{OBRIGATÓRIO}
\end{minipage}
%
\begin{minipage}[t]{0.5\textwidth}
\textsf{\scriptsize CRÉDITOS:} \\
\textbf{6}
\end{minipage}}

\noindent \fbox{
\begin{minipage}[t]{0.3\textwidth}
\textsf{\scriptsize CARGA HORÁRIA TOTAL:} \\
\textbf{90}
\end{minipage}
%
\begin{minipage}[t]{0.19\textwidth}
\textsf{\scriptsize TEÓRICA:} \\
\textbf{45}
\end{minipage}
%
\begin{minipage}[t]{0.19\textwidth}
\textsf{\scriptsize PRÁTICA:} \\
\textbf{45}
\end{minipage}
%
\begin{minipage}[t]{0.30\textwidth}
\textsf{\scriptsize EAD-SEMIPRESENCIAL:} \\
\textbf{0}
\end{minipage}}

\noindent \fbox{
\begin{minipage}[t]{\textwidth}
\textsf{\scriptsize PRÉ-REQUISITOS:}
Não há.
\end{minipage}}

\noindent \fbox{
\begin{minipage}[t]{\textwidth}
\textsf{\scriptsize CORREQUISITOS:} 
Não há.
\end{minipage}}

\noindent \fbox{
\begin{minipage}[t]{\textwidth}
\textsf{\scriptsize REQUISITO DE CARGA HORÁRIA:} 
Não há.
\end{minipage}}

\noindent \fbox{
\begin{minipage}[t]{\textwidth}
\textsf{\scriptsize EMENTA:} \\
Introdução a algoritmos e pseudocódigos. Introdução à programação
imperativa: variáveis, constantes e expressões. Controle de fluxo de
execução e repetição. Estruturas triviais de dados: vetores, matrizes e
registros. Noções de funções. Comandos de atribuição e declaração de
constantes, variáveis e tipos de dados. Expressões. Ponteiros.
Instruções condicionais de controle de fluxo. Bibliotecas definidas pelo
usuário. Recursividade. Alocação dinâmica de memória.
\end{minipage}}

\noindent \fbox{
\begin{minipage}[t]{\textwidth}
\textsf{\scriptsize BIBLIOGRAFIA BÁSICA:}
\begin{enumerate}
\def\labelenumi{\arabic{enumi}.}
\item
  Introduction to Computation and Programming Using Python. John V.
  Guttag. Spring 2013 Edition.
\item
  Schildt, Herbert, and Roberto Carlos Mayer. C completo e total. 1997.
\item
  Mizrahi, Victorine Viviane. Treinamento em linguagem C. McGraw-Hill,
  1990.
\end{enumerate}
\end{minipage}}

\noindent \fbox{
\begin{minipage}[t]{\textwidth}
\textsf{\scriptsize BIBLIOGRAFIA COMPLEMENTAR:}
\begin{enumerate}
\def\labelenumi{\arabic{enumi}.}
\item
  Fundamentos de programação de computadores. Ana F. G. Ascencio, Edilne
  A. V. de Campo. Pearson 2ª Edição 2012
\item
  Lógica de Programação. A construção de Algoritmos e estruturas de
  dados. André L. V. Forbellone e Henri Frederico Eberspäscher. Pearson
  3ª edição
\item
  Programming Python. Mark Luiz. O'reilly 4th Edition 2010
\item
  Introdução à programação: 500 algoritmos resolvidos. Anita Lopes e
  Guto Garcia. Campus 2002
\item
  Introdução à Programação com Python. Algoritmos e Lógica de
  Programação para iniciantes. Nilo Ney C. Menezes. Novatec 2ª edição
\end{enumerate}
\end{minipage}}
\end{scriptsize}

\newpage