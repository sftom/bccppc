\begin{figure*}
    \centering
    \includegraphics[width=2.57cm]{./images/brasao_da_republica.jpeg}
    
    \textsf{Ministério da Educação}\\
    \textsf{Universidade Federal do Agreste de Pernambuco}\\
    \textsf{Bacharelado em Ciência da Computação}
\end{figure*}

\vspace{0.5cm}

\begin{scriptsize}
\noindent \fbox{
\begin{minipage}[t]{\textwidth}
\textsf{\scriptsize COMPONENTE CURRICULAR:} \\
\textbf{GESTÃO DE SERVIÇOS EM TI} \\
\textsf{\scriptsize CÓDIGO:} \textbf{CCMP3077}
\end{minipage}}

\noindent \fbox{
\begin{minipage}[t]{0.293\textwidth}
\textsf{\scriptsize PERÍODO A SER OFERTADO:} \\
\textbf{0}
\end{minipage}
%
\begin{minipage}[t]{0.7\textwidth}
\textsf{\scriptsize NÚCLEO DE FORMAÇÃO:} \\
\textbf{COMPONENTES OPTATIVOS ÁREA TEMÁTICA TECNOLOGIAS DA INFORMAÇÃO}
\end{minipage}}

\noindent \fbox{
\begin{minipage}[t]{0.493\textwidth}
\textsf{\scriptsize TIPO:} \textbf{OPTATIVO}
\end{minipage}
%
\begin{minipage}[t]{0.5\textwidth}
\textsf{\scriptsize CRÉDITOS:} \textbf{4}
\end{minipage}}

\noindent \fbox{
\begin{minipage}[t]{0.3\textwidth}
\textsf{\scriptsize CARGA HORÁRIA TOTAL:} 
\textbf{60}
\end{minipage}
%
\begin{minipage}[t]{0.19\textwidth}
\textsf{\scriptsize TEÓRICA:} 
\textbf{60}
\end{minipage}
%
\begin{minipage}[t]{0.19\textwidth}
\textsf{\scriptsize PRÁTICA:} 
\textbf{0}
\end{minipage}
%
\begin{minipage}[t]{0.30\textwidth}
\textsf{\scriptsize EAD-SEMIPRESENCIAL:} 
\textbf{0}
\end{minipage}}

\noindent \fbox{
\begin{minipage}[t]{\textwidth}
\textsf{\scriptsize PRÉ-REQUISITOS:}
Não há.
\end{minipage}}

\noindent \fbox{
\begin{minipage}[t]{\textwidth}
\textsf{\scriptsize CORREQUISITOS:} 
Não há.
\end{minipage}}

\noindent \fbox{
\begin{minipage}[t]{\textwidth}
\textsf{\scriptsize REQUISITO DE CARGA HORÁRIA:} 
Não há.
\end{minipage}}

\noindent \fbox{
\begin{minipage}[t]{\textwidth}
\textsf{\scriptsize EMENTA:} \\
Conceitos do que é um Serviço. Características do Serviço. Governança de
TI e Gerenciamento de Serviços. Gestão estratégica e tática de serviços
de TI. Acordos de nível de serviço. Gerenciamento de serviços com base
no conjunto de melhores práticas baseado no ITIL (Information Technology
Infrastructure Library Biblioteca de Infra-estrutura de Tecnologia da
Informação) que identifica o relacionamento das diversas atividades
necessárias para entrega e suporte dos serviços de TI. Ferramentas de
apoio ao gerenciamento de serviços. Elaboração de Projeto.
\end{minipage}}

\noindent \fbox{
\begin{minipage}[t]{\textwidth}
\textsf{\scriptsize BIBLIOGRAFIA BÁSICA:}
\begin{enumerate}
\def\labelenumi{\arabic{enumi}.}
\item
  MAGALHÃES, Ivan Luizio; PINHEIRO, Walfrido Brito. Gerenciamento de
  Serviços de TI na Prática - Uma abordagem com base na ITIL. São Paulo:
  Novatec, 2007.
\item
  FREITAS, Marcos André dos Santos. Fundamentos do Gerenciamento de
  Serviço de TI. 2ª Ed. Brasport. 2013.
\item
  COUGO, Paulo Sérgio. ITIL - Guia de Implantação. Elsevier Campus.
  2013.
\end{enumerate}
\end{minipage}}

\noindent \fbox{
\begin{minipage}[t]{\textwidth}
\textsf{\scriptsize BIBLIOGRAFIA COMPLEMENTAR:}
\begin{enumerate}
\def\labelenumi{\arabic{enumi}.}
\item
  BON, Jan Van. Guia de Referência ITIL. Elsevier. 2012.
\item
  SILVA, Marcelo Gaspar Rodrigues; GOMEZ, Thierry Albert M;
  MIRANDA,Zailton Cardoso.TI. - Mudar E Inovar - ResolvENDO Conflitos
  Com ITIL. Ed. Senac Nacional. 2014.
\item
  ABREU, Vladimir Ferraz de; FERNANDES, Aguinaldo Aragon. Implantando A
  Governança de Ti - da Estratégia A Gestão Dos Processos e Serviços -
  4ª Ed. Brasport. 2014
\end{enumerate}
\end{minipage}}
\end{scriptsize}

\newpage